\documentclass[a4paper,twoside]{article}
\usepackage[T1]{fontenc}
\usepackage[bahasa]{babel}
\usepackage{graphicx}
\usepackage{graphics}
\usepackage{float}
\usepackage[cm]{fullpage}
\pagestyle{myheadings}
\usepackage{etoolbox}
\usepackage{setspace} 
\usepackage{lipsum} 
\setlength{\headsep}{30pt}
\usepackage[inner=2cm,outer=2.5cm,top=2.5cm,bottom=2cm]{geometry} %margin
% \pagestyle{empty}

\makeatletter
\renewcommand{\@maketitle} {\begin{center} {\LARGE \textbf{ \textsc{\@title}} \par} \bigskip {\large \textbf{\textsc{\@author}} }\end{center} }
\renewcommand{\thispagestyle}[1]{}
\markright{\textbf{\textsc{AIF401/AIF402 \textemdash Rencana Kerja Skripsi \textemdash Sem. Ganjil 2017/2018}}}

\onehalfspacing
 
\begin{document}

\title{\@judultopik}
\author{\nama \textendash \@npm} 

%tulis nama dan NPM anda di sini:
\newcommand{\nama}{Priambodo Pangestu}
\newcommand{\@npm}{2013730055}
\newcommand{\@judultopik}{Pemanfaatan Smartphone Sebagai Pengendali Permainan Berbasis Web} % Judul/topik anda
\newcommand{\jumpemb}{1} % Jumlah pembimbing, 1 atau 2
\newcommand{\tanggal}{07/09/2017}
\maketitle

\pagenumbering{arabic}

\section{Deskripsi} 
\textit{WebSockets} adalah teknologi yang memungkinkan \textit{web browser} pengguna dan \textit{web server} membuka sesi komunikasi interaktif satu sama lain. Teknologi \textit{WebSockets}  didesain untuk diimplementasikan pada \textit{web browser} dan \textit{web server}, tetapi dapat juga digunakan oleh setiap aplikasi \textit{client} maupun \textit{server}. \textit{WebSockets} memiliki standar yang menyediakan cara agar \textit{web server} dapat mengirim konten ke \textit{web browser} tanpa diminta oleh \textit{client}, dan memungkinkan agar pesan dikirimkan berulang-ulang dengan tetap menjaga koneksi yang terbuka. Oleh karena itu, protokol \textit{WebSockets} memungkinkan interaksi antara \textit{web browser} dan \textit{web server} dengan \textit{overhead} yang rendah, dan juga memfasilitasi transfer data \textit{realtime} dari \textit{server} maupun menuju \textit{server}.

Salah satu teknologi yang memanfaatkan protokol \textit{WebSockets} adalah \textit{Socket.io}. Teknologi ini memungkinkan untuk melakukan komunikasi secara \textit{realtime}, dan dua arah antara \textit{client} dan \textit{server}. Socket.io memiliki dua bagian: \textit{client-side library} yang berjalan didalam \textit{web browser}, dan \textit{server-side library} yang berjalan pada \textit{Node.js}. \textit{Socket.io} memiliki fitur-fitur yang beragam, seperti melakukan broadcast ke beberapa \textit{sockets}, dan menyimpan data yang berhubungan dengan masing-masing \textit{client}. Teknologi ini sangat berguna untuk membantu membangun sebuah aplikasi yang membutuhkan koneksi \textit{realtime} seperti dalam aplikasi \textit{chatting} maupun \textit{game}.

Pada skripsi ini, akan dibuat sebuah aplikasi permainan yang memanfaatkan protokol \textit{WebSockets}, dimana dalam penggunaan protokol tersebut akan dibantu dengan teknologi \textit{Socket.io}. Selain itu, aplikasi yang dibuat akan memanfaatkan \textit{personal computer (PC)} dan \textit{smartphone} untuk pengembangan aplikasinya. Kedua teknologi tersebut merupakan teknologi yang sudah dimiliki oleh banyak orang. Oleh karena itu, aplikasi permainan yang akan dibangun akan memanfaatkan \textit{PC} dan \textit{smartphone}.

Nama permainan yang akan dibangun adalah \textit{Finger For Life}. Permainan tersebut merupakan adu balap lari yang dapat dilakukan oleh dua orang pemain, dimana para pemain akan memiliki karakter untuk dimainkan pada trek lari yang berbentuk huruf S di layar \textit{PC}. Agar dapat memainkan permainan tersebut, para pemain harus memiliki \textit{smartphone} dan \textit{PC} beserta koneksi internet yang stabil. Apabila hal-hal tersebut terpenuhi, pemain dapat membuka \textit{web browser} pada \textit{PC} untuk mengakses alamat web yang akan menuju ke permainan \textit{Finger For Life}. Para pemain akan diminta untuk melakukan dua hal agar dapat memainkan permainan tersebut bersama seorang rekan yang akan menjadi lawan mainnya, yaitu : 

\begin{itemize}
	\item Membuka \textit{web browser} pada \textit{PC} untuk mengakses alamat web permainan \textit{Finger For Life}.
	\item Mengakses alamat web permainan dan memasukan kode tertentu pada \textit{web browser} di \textit{smartphone} untuk sesi permainan saat ini.
\end{itemize}

Kedua hal tersebut bertujuan untuk melakukan koneksi antara \textit{smartphone} dan \textit{PC}, dimana \textit{smartphone} akan berfungsi sebagai \textit{controller} dalam permainan. Apabila kedua hal diatas telah dilakukan, maka kedua pemain akan dapat mulai memainkan permainannya.  Permainan akan diawali dengan pemilihan karakter oleh kedua pemain, dimana karakter tersebut akan berfungsi sebagai representasi masing-masing pemain dalam permainan \textit{Finger For Life}. Setelah pemilihan karakter selesai, maka para pemain akan dibawa ke halaman selanjutnya yang berupa halaman \textit{test drive}. Pada halaman ini, para pemain diminta untuk memegang \textit{smartphone} masing-masing untuk mencoba memainkan permainannya dengan cara menekan tombol-tombol yang muncul pada \textit{smartphone}. Hal tersebut bertujuan agar para pemain terbiasa terlebih dahulu dengan cara bermainnya. Setelah hal itu dilakukan, maka para pemain dapat memulai memainkan permainannya.

Para pemain akan mengkoneksikan \textit{smartphone} miliknya pada suatu \textit{PC}, dimana \textit{smartphone} tersebut akan berfungsi sebagai \textit{controller} untuk memainkan permainannya. Oleh karena itu, protokol \textit{WebSockets} akan digunakan sebagai koneksi antara \textit{smartphone} dan \textit{PC} dalam aplikasi permainan yang akan dibangun. Aplikasi permainan akan menggunakan teknologi berbasis web, sehingga untuk memainkannya, \textit{client} bisa mengakses melalui \textit{web browser} tanpa harus berada di satu jaringan lokal yang sama.

\section{Rumusan Masalah}

\begin{itemize}
	\item Bagaimana membangun aplikasi permainan berbasis web dengan memanfaatkan protokol \textit{WebSockets} untuk penggunaan \textit{smartphone} sebagai pengendali permainan berbasis web ?
	\item Berapa \textit{latency} yang dihasilkan berdasarkan penggunaan protokol \textit{WebSockets} ? 
\end{itemize}

\section{Tujuan}

\begin{itemize}
	\item Mengetahui cara membangun aplikasi permainan berbasis web dengan memanfaatkan protokol \textit{WebSockets} untuk penggunaan \textit{smartphone} sebagai pengendali permainan berbasis web.
	\item Mengetahui jumlah \textit{latency} yang dihasilkan berdasarkan pemanfaatan protokol \textit{WebSockets}.
\end{itemize}

\section{Deskripsi Perangkat Lunak}

Pada skripsi ini akan dibuat aplikasi permainan berbasis web yang akan memanfaatkan \textit{smartphone} dan \textit{PC}.

Aplikasi yang akan dibuat memiliki fitur minimal sebagai berikut:
\begin{itemize}
	\item Aplikasi dijalankan dengan cara mengakses alamat web pada \textit{web browser} yang ada di \textit{smartphone} dan \textit{PC}, yang kemudian \textit{smartphone} akan dikoneksikan dengan \textit{PC}.
	\item \textit{Smartphone} digunakan sebagai \textit{controller} dalam permainan.
	\item \textit{PC} digunakan sebagai layar yang memperlihatkan permainan yang sedang dimainkan.
	\item Aplikasi dapat dimainkan oleh dua orang dalam satu waktu.
\end{itemize}

\section{Detail Pengerjaan Skripsi}


Bagian-bagian pekerjaan skripsi ini adalah sebagai berikut :
	
	\begin{enumerate}
		\item Melakukan studi literatur mengenai \textit{WebSockets}, \textit{Socket.io}, \textit{Node.js}, \textit{HTMLCanvas}.
		\item Menganalisis aplikasi sejenis.
		\item Merancang antarmuka permainan pada \textit{PC} dan \textit{smartphone}.
		\item Menyusun cara bermain aplikasi permainan yang dibangun.
		\item Mengimplementasi program aplikasi permainan berbasis web.
		\item Menganalisis \textit{latency} yang dihasilkan pada aplikasi.
		\item Melakukan eksperimen dan pengujian yang melibatkan responden untuk menilai hasil simulasi secara kualitatif
		\item Menulis dokumen skripsi
	\end{enumerate}

\section{Rencana Kerja}

\begin{center}
  \begin{tabular}{ | c | c | c | c | l |}
    \hline
    1*  & 2*(\%) & 3*(\%) & 4*(\%) &5*\\ \hline \hline
    1 & 5 & 5 & & \\ \hline
    2 & 5 & 5 & & \\ \hline
    3 & 15 & 10 & 5 & \\ \hline
    4 & 15 & 5 & 10 & \\ \hline
    5 & 20 & 5 & 15 & \\ \hline
    6 & 10 &   & 10 & \\ \hline
    7 & 10 &  & 10 & \\ \hline
    8 & 20 & 10 & 10 & {\footnotesize menulis dokumen skripsi dari bab1 hingga bab3 di S1} \\ \hline
    Total & 100 & 40 & 60 & \\ \hline
                          \end{tabular}
\end{center}

Keterangan (*)\\
1 : Bagian pengerjaan Skripsi (nomor disesuaikan dengan detail pengerjaan di bagian 5)\\
2 : Persentase total \\
3 : Persentase yang akan diselesaikan di Skripsi 1 \\
4 : Persentase yang akan diselesaikan di Skripsi 2 \\
5 : Penjelasan singkat apa yang dilakukan di S1 (Skripsi 1) atau S2 (Skripsi 2)

\vspace{1cm}
\centering Bandung, \tanggal\\
\vspace{2cm} \nama \\ 
\vspace{1cm}

Menyetujui, \\
\ifdefstring{\jumpemb}{2}{
\vspace{1.5cm}
\begin{centering} Menyetujui,\\ \end{centering} \vspace{0.75cm}
\begin{minipage}[b]{0.45\linewidth}
% \centering Bandung, \makebox[0.5cm]{\hrulefill}/\makebox[0.5cm]{\hrulefill}/2013 \\
\vspace{2cm} Nama: \makebox[3cm]{\hrulefill}\\ Pembimbing Utama
\end{minipage} \hspace{0.5cm}
\begin{minipage}[b]{0.45\linewidth}
% \centering Bandung, \makebox[0.5cm]{\hrulefill}/\makebox[0.5cm]{\hrulefill}/2013\\
\vspace{2cm} Nama: \makebox[3cm]{\hrulefill}\\ Pembimbing Pendamping
\end{minipage}
\vspace{0.5cm}
}{
% \centering Bandung, \makebox[0.5cm]{\hrulefill}/\makebox[0.5cm]{\hrulefill}/2013\\
\vspace{2cm} Nama: \makebox[3cm]{\hrulefill}\\ Pembimbing Tunggal
}

\end{document}

