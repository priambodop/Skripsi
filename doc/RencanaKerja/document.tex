\documentclass[a4paper,twoside]{article}
\usepackage[T1]{fontenc}
\usepackage[bahasa]{babel}
\usepackage{graphicx}
\usepackage{graphics}
\usepackage{float}
\usepackage[cm]{fullpage}
\pagestyle{myheadings}
\usepackage{etoolbox}
\usepackage{setspace} 
\usepackage{lipsum} 
\setlength{\headsep}{30pt}
\usepackage[inner=2cm,outer=2.5cm,top=2.5cm,bottom=2cm]{geometry} %margin
% \pagestyle{empty}

\makeatletter
\renewcommand{\@maketitle} {\begin{center} {\LARGE \textbf{ \textsc{\@title}} \par} \bigskip {\large \textbf{\textsc{\@author}} }\end{center} }
\renewcommand{\thispagestyle}[1]{}
\markright{\textbf{\textsc{AIF401/AIF402 \textemdash Rencana Kerja Skripsi \textemdash Sem. Ganjil 2017/2018}}}

\onehalfspacing
 
\begin{document}

\title{\@judultopik}
\author{\nama \textendash \@npm} 

%tulis nama dan NPM anda di sini:
\newcommand{\nama}{Priambodo Pangestu}
\newcommand{\@npm}{2013730055}
\newcommand{\@judultopik}{Pemanfaatan Smartphone Sebagai Pengendali Permainan Berbasis Web} % Judul/topik anda
\newcommand{\jumpemb}{1} % Jumlah pembimbing, 1 atau 2
\newcommand{\tanggal}{01/01/1900}
\maketitle

\pagenumbering{arabic}

\section{Deskripsi}
%Tuliskan deskripsi dari topik skripsi yang akan anda ajukan. Di sini dapat dituliskan latar belakang, seperti apa penelitian yang sudah ada sebelumnya dan apa yang akan anda kerjakan. Sertakan gambar agar penjelasan anda menjadi lebih baik.

%Pada skripsi ini, akan dibuat sebuah perangkat lunak yang dapat menampilkan visualisasi dan simulasi kerumunan orang yang berkunjung ke sebuah museum. Dengan menggunakan perangkat lunak tersebut, pengelola museum dapat mengatur tempat peletakan objek sehingga tidak terjadi kerumunan yang terlalu padat.

%Dari berbagai macam teknik yang dapat digunakan untuk melakukan simulasi kerumunan, dipilih dua buah teknik yaitu teknik {\it flow tiles} dan {\it social force model (steering behaviour)}.

%Dst, dst, dst, \ldots\ldots\ldots 

%Perangkat lunak akan dibuat dengan bantuan {\it framework} OpenSteer. Sebagai studi kasus, museum yang digunakan untuk melakukan simulasi adalah Museum Geologi Bandung.

%Dst, dst, dst, \ldots\ldots\ldots 
Websocket adalah teknologi yang memungkinkan \textit{browser} pengguna dan \textit{server} membuka sesi komunikasi interaktif satu sama lain. Teknologi websocket didesain untuk diimplementasikan pada web browser dan web server, tetapi dapat juga digunakan oleh setiap aplikasi \textit{client} maupun \textit{server}. Websocket memiliki standar yang menyediakan cara agar \textit{server} dapat mengirim konten ke \textit{browser} tanpa diminta oleh \textit{client}, dan memungkinkan agar pesan dikirimkan berulang-ulang dengan tetap menjaga koneksi yang terbuka. Oleh karena itu, protokol websocket memungkinkan interaksi antara web \textit{browser} dan web \textit{server} dengan biaya \textit{overhead} yang rendah, dan juga memfasilitasi transfer data \textit{realtime} dari \textit{server} maupun menuju \textit{server}.

Salah satu teknologi yang memanfaatkan protokol Websockets adalah Socket.io. Teknologi ini memungkinkan untuk melakukan komunikasi secara realtime, dan dua arah antara web client dan server. Socket.io memiliki dua bagian: \textit{library client-side} yang berjalan didalam browser, dan \textit{library server-side} yang berjalan pada Node.js. Socket.io memiliki fitur-fitur yang beragam, seperti melakukan broadcast ke beberapa sockets, dan menyimpan data yang berhubungan dengan masing-masing client. Teknologi ini sangat berguna untuk membantu membangun sebuah aplikasi yang membutuhkan koneksi realtime seperti dalam aplikasi chatting maupun \textit{game}.

Pada skripsi ini, akan dibuat sebuah aplikasi permainan yang memanfaatkan protokol Websockets, dimana dalam penggunaan protokol tersebut akan dibantu dengan teknologi Socket.io. Aplikasi permainan yang dibuat akan menggunakan \textit{smartphone} dan Personal Computer (PC). Oleh karena itu, protokol Websockets akan digunakan sebagai koneksi antara smartphone dan PC dalam aplikasi permainan yang akan dibangun. Aplikasi permainan akan menggunakan teknologi berbasis web, sehingga untuk memainkannya, \textit{client} bisa mengakses melalui \textit{browser} dimanapun dan kapanpun.

\section{Rumusan Masalah}
%Tuliskan rumusan dari masalah yang akan anda bahas pada skripsi ini. Rumusan masalah biasanya berupa kalimat pertanyaan. Gunakan itemize seperti contoh di bagian Deskripsi Perangkat Lunak.
\begin{itemize}
	\item Bagaimana memanfaatkan protokol Websocket untuk penggunaan \textit{smartphone} sebagai pengendali permainan berbasis web ?
	\item Berapa latency yang dihasilkan berdasarkan penggunaan protokol websocket ? 
\end{itemize}

\section{Tujuan}
%Tuliskan tujuan dari topik skripsi yang anda ajukan. Tujuan penelitian biasanya berkaitan erat dengan pertanyaan yang diajukan di bagian rumusan masalah. Gunakan itemize seperti contoh di bagian Deskripsi Perangkat Lunak.
\begin{itemize}
	\item Mengetahui cara memanfaatkan protokol Websocket untuk penggunaan \textit{smartphone} sebagai pengendali permainan berbasis web.
	\item Mengetahui jumlah latency yang dihasilkan berdasarkan penggunaan protokol websocket.
\end{itemize}

\section{Deskripsi Perangkat Lunak}
%Tuliskan deksripsi dari perangkat lunak yang akan anda hasilkan. Apa saja fitur yang disediakan oleh PL tersebut dan apa saja kemampuan dari PL tersebut. Perhatikan contoh di bawah ini:

%Perangkat lunak akhir yang akan dibuat memiliki fitur minimal sebagai berikut:
%\begin{itemize}
	%\item Pengguna dapat melihat denah Musem Geologi Bandung dalam bidang dua dimensi. Sedangkan pengunjung direpresentasikan menggunakan lingkaran-lingkaran kecil (tidak menggunakan gambar manusia yang diambil dari atas)
	%\item Pengguna dapat memunculkan atau menghilangkan gambar {\it flow tiles} pada denah museum. 
	%\item Pengguna dapat mengatur jalannya simulasi: memulai(start) simulasi, menunda(pause) simulasi, melanjutkan(continue) simulasi, maupun menghentikan(stop) simulasi
	%\item Pengguna dapat mengatur banyaknya pengunjung di dalam museum, baik melalui perubahan frekuensi kedatangan pengunjung maupun menambahkan dan menghapus pengunjung satu-persatu secara manual.
	%\item Posisi kamera dapat diubah (pergerakan di bidang tiga dimensi) sehingga pengguna dapat melihat simulasi di museum dari berbagai arah. 
	%\item Posisi kamera dapat diubah untuk emngikuti perjalanan seorang pengunjung di dalam 
	%\item Pengguna dapat memilih apakah akan menggunakan teknik {\it flow tiles} atau tidak pada saat simulasi berlangsung
	%\item Jenis {\it flow tiles} yang digunakan dapat diubah-ubah pada saat simulasi sedang berlangsung
		
%\end{itemize}
Pada skripsi ini akan dibuat aplikasi permainan berbasis web yang akan memanfaatkan \textit{smartphone} dan PC.

Aplikasi yang akan dibuat memiliki fitur minimal sebagai berikut:
\begin{itemize}
	\item Aplikasi dapat diakses melalui web \textit{browser} milik pengguna yang ada di \textit{smartphone} dan \textit{PC}.
	\item Pengguna dapat menggunakan \textit{smartphone} miliknya sebagai pengendali permainan.
	\item  Pengguna dapat mengakses permainan melalui web \textit{browser} miliknya.
	\item Aplikasi dapat dimainkan lebih dari satu orang dalam satu waktu, dengan jumlah maksimal pemain sebanyak tiga orang.
	\item Pengguna-pengguna yang memainkan permainan ini dapat berada ditempat yang berbeda satu sama lain selama ada koneksi internet yang stabil. 
\end{itemize}

\section{Detail Pengerjaan Skripsi}
%Tuliskan bagian-bagian pengerjaan skripsi secara detail. Bagian pekerjaan tersebut mencakup awal hingga akhir skripsi, termasuk di dalamnya pengerjaan dokumentasi skripsi, pengujian, survei, dll.

Bagian-bagian pekerjaan skripsi ini adalah sebagai berikut :
	%\begin{enumerate}
		%\item Melakukan survei ke Museum Geologi Bandung untuk mendapatkan denah serta mengetahui perilaku pengunjung museum secara umum (arah perjalanan, kecepatan, lama melihat objek, dll)
		%\item Melakukan analisis pada hasil survei terhadap pergerakan pengunjung di museum dan membuat rancangan denah di komputer yang dilengkapi dengan penghalang dan objek di museum.
		%\item Melakukan studi literatur mengenai sifat kolektif suatu kerumunan, teknik {\it social force model} dan teknik {\it flow tiles}
		%\item Mempelajari bahasa pemrograman C++ dan cara menggunakan framework OpenSteer
		%\item Merancang pergerakan kerumunan di dalam museum menggunakan teknik {\it social force model} dan {\it flow tiles} serta menggunakan teknik lainnya seperti konsep pathway dan waypoints. Selain itu, dirancang pula adanya waktu tunggu (pada saat pengunjung melihat objek di museum) dan cara pembuatan jalur bagi setiap individu pengunjung
		%\item Melakukan analisa dan merancang struktur data yang cocok untuk menyimpan penghalang (obstacle)
		%\item Mengimplementasikan keseluruhan algoritma dan struktur data yang dirancang, dengan menggunakan framework OpenSteer 
		%\item Melakukan pengujian (dan eksperimen) yang melibatkan responde untuk menilai hasil simulasi secara kualitatif
		%\item Menulis dokumen skripsi
	%\end{enumerate}
	\begin{enumerate}
		\item Melakukan studi literatur mengenai Websocket, Socket.io, Node.js, HTMLCanvas.
		\item Mempelajari cara kerja program Websocket.
		\item Menganalisis aplikasi sejenis.
		\item Merancang antarmuka permainan pada \textit{PC} dan \textit{smartphone}.
		\item Menyusun cara bermain aplikasi permainan yang dibangun.
		\item Menganalisis latency yang dihasilkan pada aplikasi.
		\item Melakukan eksperimen dan pengujian yang melibatkan responden untuk menilai hasil simulasi secara kualitatif
		\item Menulis dokumen skripsi
	\end{enumerate}

\section{Rencana Kerja}
Tuliskan rencana anda untuk menyelesaikan skripsi. Rencana kerja dibagi menjadi dua bagian yaitu yang akan dilakukan pada saat mengambil kuliah AIF401 Skripsi 1 dan pada saat mengambil kuliah AIF402 Skripsi 2. Perhatikan contoh berikut ini :


\begin{center}
  \begin{tabular}{ | c | c | c | c | l |}
    \hline
    1*  & 2*(\%) & 3*(\%) & 4*(\%) &5*\\ \hline \hline
    %1   & 5  & 5  &  &  \\ \hline
    %2   & 5 & 5  &   & \\ \hline
    %3   & 10  & 7  & 3 & {\footnotesize sebagian kecil teknik {\it flow tiles} di S2}  \\ \hline
    %4   & 15  & 10  &  5 & {\footnotesize teknik lanjut OOP di C++ di S2} \\ \hline
    %5   & 20  & 5  & 15 & {\footnotesize perancangan awal SFM, pathway dan waypoint di S1} \\ \hline
    %6   & 5 &   & 5  & \\ \hline
    %7   & 20  & 5  & 15 &  {\footnotesize implementasi denah dan rancangan awal SFM di S1}\\ \hline
    %8   & 5  &   &  5  & \\ \hline
    %9   & 15  & 3  & 12  & {\footnotesize menulis dokumen skripsi dari bab1 hingga bab3 di S1}\\ \hline
    %Total  & 100  & 40  & 60 &  \\ \hline
    1 & 5 & 5 & & \\ \hline
    2 & 10 & 7 & 3 & \\ \hline
    3 & 5 & 5 & & \\ \hline
    4 & 15 & 5 & 10 & \\ \hline
    5 & 15 & 7 & 8 & \\ \hline
    6 & 15 & 5 & 10 & \\ \hline
    7 & 15 & 5 & 10 & \\ \hline
    8 & 20 & 10 & 10 & {\footnotesize menulis dokumen skripsi dari bab1 hingga bab3 di S1} \\ \hline
    Total & 100 & 49 & 51 & \\ \hline
                          \end{tabular}
\end{center}

Keterangan (*)\\
1 : Bagian pengerjaan Skripsi (nomor disesuaikan dengan detail pengerjaan di bagian 5)\\
2 : Persentase total \\
3 : Persentase yang akan diselesaikan di Skripsi 1 \\
4 : Persentase yang akan diselesaikan di Skripsi 2 \\
5 : Penjelasan singkat apa yang dilakukan di S1 (Skripsi 1) atau S2 (skripsi 2)

\vspace{1cm}
\centering Bandung, \tanggal\\
\vspace{2cm} \nama \\ 
\vspace{1cm}

Menyetujui, \\
\ifdefstring{\jumpemb}{2}{
\vspace{1.5cm}
\begin{centering} Menyetujui,\\ \end{centering} \vspace{0.75cm}
\begin{minipage}[b]{0.45\linewidth}
% \centering Bandung, \makebox[0.5cm]{\hrulefill}/\makebox[0.5cm]{\hrulefill}/2013 \\
\vspace{2cm} Nama: \makebox[3cm]{\hrulefill}\\ Pembimbing Utama
\end{minipage} \hspace{0.5cm}
\begin{minipage}[b]{0.45\linewidth}
% \centering Bandung, \makebox[0.5cm]{\hrulefill}/\makebox[0.5cm]{\hrulefill}/2013\\
\vspace{2cm} Nama: \makebox[3cm]{\hrulefill}\\ Pembimbing Pendamping
\end{minipage}
\vspace{0.5cm}
}{
% \centering Bandung, \makebox[0.5cm]{\hrulefill}/\makebox[0.5cm]{\hrulefill}/2013\\
\vspace{2cm} Nama: \makebox[3cm]{\hrulefill}\\ Pembimbing Tunggal
}

\end{document}

