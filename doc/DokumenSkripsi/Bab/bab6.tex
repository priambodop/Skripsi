\chapter{Kesimpulan dan Saran}
\label{chap:kesdansaran}

\section{Kesimpulan}
\label{sec:kesimpulan}
Dari hasil pembangunan aplikasi Finger For Life, didapatkanlah kesimpulan-kesimpulan sebagai berikut:

\begin{enumerate}
	\item Aplikasi Finger For Life telah berhasil dibangun dengan menggunakan Socket.io. Selain itu, aplikasi ini dibangun berdasarkan Node.js, dan juga menggunakan Express.js. Untuk proses animasi, Finger For Life menggunakan CanvasAPI dalam proses pengembangannya. Dalam proses pengolahan halaman HTML pada aplikasi, proses tersebut dibantu dengan menggunakan jQuery dan The Content Template Element dari HTML.
	
	\item Telah berhasil mendapatkan ukuran \textit{latency} yang dihasilkan oleh aplikasi Finger For Life dengan pengimplementasian Socket.io. Rata-rata ukuran \textit{latency} yang dihasilkan adalah 244 milidetik, atau sekitar 0,244 detik. Rata-rata ukuran \textit{latency} tersebut masih mendekati 100 milidetik. Dengan begitu, aplikasi Finger For Life telah memberikan respon yang cepat pada saat aplikasi dijalankan.
\end{enumerate}

\section{Saran}
\label{sec:saran}

Dari hasil penelitian termasuk kesimpulan yang didapat, berikut adalah saran untuk pengembangan lebih lanjut:

\begin{enumerate}
	\item Penelitian ini menggunakan metode berbasis \textit{event} yang diimplementasi menggunakan Socket.io untuk menghitung jumlah \textit{latency} yang dihasilkan oleh aplikasi Finger For Life pada saat dimainkan. Proses menghitung \textit{latency} digunakan dengan menggunakan waktu yang terdapat pada \textit{PC} sebagai acuan waktunya. Dengan demikian, hasil \textit{latency} yang didapat terkadang berjumlah negatif. Hal tersebut terjadi karena waktu yang terdapat pada \textit{PC} dan \textit{smartphone} tidak sinkron. Oleh karena itu, dibutuhkan metode yang lebih baik agar waktu yang terdapat pada \textit{PC} dan \textit{smartphone} akan selalu sinkron, sehingga tidak menghasilkan \textit{latency} dengan bilangan negatif.
\end{enumerate}