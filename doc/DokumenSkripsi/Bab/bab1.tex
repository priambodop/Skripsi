%versi 2 (8-10-2016) 
\chapter{Pendahuluan}
\label{chap:intro}
   
\section{Latar Belakang}
\label{sec:label}

\textit{WebSockets} adalah teknologi yang memungkinkan \textit{web browser} pengguna dan \textit{web server} membuka sesi komunikasi interaktif satu sama lain. Teknologi \textit{WebSockets}  didesain untuk diimplementasikan pada \textit{web browser} dan \textit{web server}, tetapi dapat juga digunakan oleh setiap aplikasi \textit{client} maupun \textit{server}. \textit{WebSockets} memiliki standar yang menyediakan cara agar \textit{web server} dapat mengirim konten ke \textit{web browser} tanpa diminta oleh \textit{client}, dan memungkinkan agar pesan dikirimkan berulang-ulang dengan tetap menjaga koneksi yang terbuka. Oleh karena itu, protokol \textit{WebSockets} memungkinkan interaksi antara \textit{web browser} dan \textit{web server} dengan \textit{overhead} yang rendah, dan juga memfasilitasi transfer data \textit{realtime} dari \textit{server} maupun menuju \textit{server}.

Salah satu teknologi yang memanfaatkan protokol \textit{WebSockets} adalah \textit{Socket.io}. Teknologi ini memungkinkan untuk melakukan komunikasi secara \textit{realtime}, dan dua arah antara \textit{client} dan \textit{server}. Socket.io memiliki dua bagian: \textit{client-side library} yang berjalan didalam \textit{web browser}, dan \textit{server-side library} yang berjalan pada \textit{Node.js}. \textit{Socket.io} memiliki fitur-fitur yang beragam, seperti melakukan broadcast ke beberapa \textit{sockets}, dan menyimpan data yang berhubungan dengan masing-masing \textit{client}. Teknologi ini sangat berguna untuk membantu membangun sebuah aplikasi yang membutuhkan koneksi \textit{realtime} seperti dalam aplikasi \textit{chatting} maupun \textit{game}.

Pada skripsi ini, akan dibuat sebuah aplikasi permainan yang memanfaatkan protokol \textit{WebSockets}, dimana dalam penggunaan protokol tersebut akan dibantu dengan teknologi \textit{Socket.io}. Selain itu, aplikasi yang dibuat akan memanfaatkan \textit{personal computer (PC)} dan \textit{smartphone} untuk pengembangan aplikasinya. Kedua teknologi tersebut merupakan teknologi yang sudah dimiliki oleh banyak orang. Oleh karena itu, aplikasi permainan yang akan dibangun akan memanfaatkan \textit{PC} dan \textit{smartphone}.

Para pemain akan mengkoneksikan \textit{smartphone} miliknya pada suatu \textit{PC}, dimana \textit{smartphone} tersebut akan berfungsi sebagai \textit{controller} untuk memainkan permainannya. Oleh karena itu, protokol \textit{WebSockets} akan digunakan sebagai koneksi antara \textit{smartphone} dan \textit{PC} dalam aplikasi permainan yang akan dibangun. Aplikasi permainan akan menggunakan teknologi berbasis web, sehingga untuk memainkannya, \textit{client} bisa mengakses melalui \textit{web browser} tanpa harus berada di satu jaringan lokal yang sama.

\section{Rumusan Masalah}
\label{sec:rumusan}

\begin{itemize}
	\item Bagaimana membangun aplikasi permainan berbasis web dengan memanfaatkan protokol \textit{WebSockets} untuk penggunaan \textit{smartphone} sebagai pengendali permainan berbasis web ?
	\item Berapa \textit{latency} yang dihasilkan berdasarkan penggunaan protokol \textit{WebSockets} ? 
\end{itemize}


\section{Tujuan}
\label{sec:tujuan}
\begin{itemize}
	\item Mengetahui cara membangun aplikasi permainan berbasis web dengan memanfaatkan protokol \textit{WebSockets} untuk penggunaan \textit{smartphone} sebagai pengendali permainan berbasis web.
	\item Mengetahui jumlah \textit{latency} yang dihasilkan berdasarkan pemanfaatan protokol \textit{WebSockets}.
\end{itemize}


\section{Batasan Masalah}
\label{sec:batasan}

Batasan masalah yang dibuat terkait dengan pengerjaan skripsi ini adalah sebagai berikut:

\begin{itemize}
	\item Aplikasi permainan yang dibuat merupakan permainan \textit{multiplayer} yang hanya bisa dimainkan oleh dua orang saja.
\end{itemize}


\section{Metodologi}
\label{sec:metlit}
Metodologi yang dilakukan dalam pengerjaan skripsi ini adalah sebagai berikut:

\begin{itemize}
	\item Studi literatur mengenai :
		\begin{itemize}
			\item \textit{WebSockets} yang akan digunakan untuk koneksi antara \textit{smartphone} dan \textit{PC}.
			\item \textit{Socket.io} sebagai teknologi yang akan menggunakan \textit{WebSockets} dalam pembangunan aplikasi.
			\item \textit{HTML5 Canvas} yang akan digunakan untuk antarmuka permainan.
			\item \textit{NodeJs} sebagai \textit{server} dalam pembangunan aplikasi 	
		\end{itemize}
\end{itemize}

\dtext{9}

\section{Sistematika Pembahasan}
\label{sec:sispem}
Rencananya Bab 2 akan berisi petunjuk penggunaan template dan dasar-dasar \LaTeX.
Mungkin bab 3,4,5 dapt diisi oleh ketiga jurusan, misalnya peraturan dasar skripsi atau pedoman penulisan, tentu jika berkenan.
Bab 6 akan diisi dengan kesimpulan, bahwa membuat template ini ternyata sungguh menghabiskan banyak waktu.

\dtext{10}