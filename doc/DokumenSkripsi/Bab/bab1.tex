%versi 2 (8-10-2016) 
\chapter{Pendahuluan}
\label{chap:intro}
   
\section{Latar Belakang}
\label{sec:label}

%\textit{WebSockets} adalah teknologi yang memungkinkan \textit{web browser} pengguna dan \textit{web server} membuka sesi komunikasi interaktif satu sama lain. Teknologi \textit{WebSockets}  didesain untuk diimplementasikan pada \textit{web browser} dan \textit{web server}, tetapi dapat juga digunakan oleh setiap aplikasi \textit{client} maupun \textit{server}. \textit{WebSockets} memiliki standar yang menyediakan cara agar \textit{web server} dapat mengirim konten ke \textit{web browser} tanpa diminta oleh \textit{client}, dan memungkinkan agar pesan dikirimkan berulang-ulang dengan tetap menjaga koneksi yang terbuka. Oleh karena itu, protokol \textit{WebSockets} memungkinkan interaksi antara \textit{web browser} dan \textit{web server} dengan \textit{overhead} yang rendah, dan juga memfasilitasi transfer data \textit{realtime} dari \textit{server} maupun menuju \textit{server}.

%Salah satu teknologi yang memanfaatkan protokol \textit{WebSockets} adalah \textit{Socket.io}. Teknologi ini 

Teknologi \textit{smartphone} dan \textit{Personal Computer (PC)} dimanfaatkan untuk mengakses berbagai macam layanan aplikasi yang tersedia. Beberapa aplikasi yang dapat diakses adalah \textit{web browser} dan aplikasi permainan. Kedua jenis aplikasi tersebut dapat diakses melalui \textit{smartphone} maupun \textit{PC}. Salah satu aplikasi yang dapat memanfaatkan kedua gawai tersebut adalah permainan berbasis web yang memanfaatkan pengendali.

Permainan berbasis web adalah aplikasi permainan yang diakses melalui browser yang membutuhkan jaringan internet. Teknologi \textit{smartphone} dapat dimanfaatkan sebagai pengendali yang memainkan permainan berbasis web melalui browser. \textit{PC} berperan sebagai \textit{console} yang menyediakan permainan sehingga \textit{smartphone} dapat menjadi pengendali permainan. Untuk dapat memanfaatkan \textit{smartphone} sebagai pengendali permainan berbasis web, \textit{PC} dan \textit{smartphone} harus terhubung satu sama lain. 

\textit{Smartphone} harus membuka \textit{browser} untuk mengakses permainan berbasis web yang sama dengan yang ada di \textit{browser} pada \textit{PC}. Dengan memanfaatkan \textit{smartphone} dan \textit{PC} di dalam aplikasi permainan berbasis web maka dibutuhkan ukuran \textit{latency} yang kecil. \textit{Latency} adalah jeda waktu yang dihasilkan pada saat \textit{client} mengirimkan data kepada \textit{server} maupun sebaliknya. \textit{Latency} yang dapat memberikan respon cepat pada aplikasi permainan berbasis web adalah sekitar 100 milidetik, atau sekitar 0,100 detik \footnote{\url{https://forum.unity.com/threads/question-about-acceptable-levels-of-latency-in-online-gaming.261271/}, diakses 2 Januari 2019}. Semakin kecil ukuran \textit{latency} yang dihasilkan maka semakin cepat respon yang diterima.

Agar \textit{smartphone} dan \textit{PC} dapat terhubung satu sama lain, maka dibutuhkan suatu pustaka yang dapat memenuhi hal tersebut. Pustaka yang dapat digunakan adalah Socket.io.

Socket.io adalah pustaka yang memungkinkan \textit{client} dan \textit{server} untuk melakukan komunikasi dua arah secara \textit{real-time} yang artinya jeda waktu yang sedikit antara permintaan yang dilakukan oleh \textit{client} dengan respon yang diberikan oleh \textit{server} maupun sebaliknya  \cite{damien:11:socketiodocs}. Socket.io memiliki dua bagian: \textit{client-side library}, atau pustaka pada bagian \textit{client} yang berjalan di dalam \textit{web browser}, dan \textit{server-side library}, atau pustaka pada bagian \textit{server} yang berjalan pada bagian \textit{server}. Socket.io memiliki fitur untuk melakukan komunikasi dari satu \textit{server} ke beberapa \textit{client} di dalam proses implementasinya. Teknologi ini sangat berguna untuk membantu membangun sebuah aplikasi yang membutuhkan koneksi \textit{real-time} seperti di dalam aplikasi permainan berbasis web.

Untuk memanfaatkan teknologi Socket.io dalam membangun aplikasi permainan dibutuhkan beberapa teknologi yang dapat membantu pembangunan aplikasinya. Salah satu teknologi tersebut adalah Canvas API. Teknologi ini merupakan bagian dari elemen HTML5 yang dapat digunakan untuk mengolah objek grafis dengan menggunakan JavaScript \cite{moz:04:canvasapi}. \textit{Canvas API} dapat juga digunakan untuk mengolah komposisi foto dan membuat animasi. Oleh karena itu, fungsi-fungsi yang ada pada \textit{Canvas API} membantu pembangunan aplikasi permainan terutama pada bagian pengembangan grafis.

Teknologi lain yang dapat membantu membangun aplikasi permainan dalam menggunakan teknologi Socket.io adalah Node.js. Teknologi ini merupakan sebuah \textit{platform} atau lingkungan yang didesain untuk mengembangkan aplikasi berbasis web pada bagian \textit{server} \cite{nodeFound:09:nodejsdocs}. Node.js ditulis dalam sintaks bahasa pemrograman JavaScript dan menggunakan \textit{V8} yang merupakan \textit{engine} JavaScript milik perusahaan \textit{Google} untuk mengeksekusi JavaScript pada \textit{web server}. Node.js memiliki sifat \textit{non-blocking} yang artinya adalah Node.js tidak akan menunggu untuk mengerjakan permintaan selanjutnya. Oleh karena itu, fitur-fitur yang dimiliki oleh Node.js akan sangat membantu untuk membangun aplikasi permainan yang membutuhkan koneksi \textit{real-time}.

Salah satu teknologi yang akan membantu dalam mengimpementasi Node.js adalah Express.js \cite{nodeFound:10:expressjsapi}. Teknologi ini menyediakan kumpulan fitur untuk mengatur penyimpanan data secara lokal dalam membangun aplikasi web maupun \textit{mobile}. Pada proses implementasinya Express.js akan mengolah data lokal sedemikian rupa sehingga dapat dengan mudah diakses apabila diperlukan. Express.js hanya dapat digunakan untuk membangun aplikasi apabila aplikasi tersebut berjalan di dalam lingkungan Node.js. Oleh karena itu, fitur-fitur yang dimiliki oleh Express.js akan membantu di dalam pembangunan aplikasi berbasis Node.js.

Penelitian yang dilakukan merupakan aplikasi permainan berbasis web yang memanfaatkan teknologi \textit{PC} dan \textit{smartphone}. Oleh karena itu dibutuhkan teknologi yang dapat mengatur tampilan halaman pada layar \textit{PC} maupun \textit{smartphone}. Teknologi yang digunakan adalah HTML Content Template (<template>) \cite{moz:05:template}. Teknologi ini merupakan bagian dari elemen HTML5, yang berfungsi untuk menyimpan seluruh elemen-elemen HTML untuk ditampilkan ke layar \textit{browser} pada \textit{PC} maupun \textit{smartphone}. Di dalam satu berkas HTML, elemen <template> dapat berjumlah lebih dari satu. Dengan begitu, beberapa halaman dapat dipilih untuk ditampilkan dalam satu waktu tertentu, sebelum menampilkan halaman lain yang disimpan oleh <template>. Proses menampilkan halaman ke layar \textit{PC} maupun \textit{smartphone} akan menggunakan JavaScript.

Teknologi yang akan membantu dalam penggunaan <template> adalah jQuery \cite{jqFound:06:jQueryAPI}. Teknologi ini merupakan pustaka JavaScript yang menyediakan fitur-fitur untuk mengatur berbagai elemen HTML. Pustaka ini memiliki fitur untuk memanipulasi berkas HTML. Dengan begitu, jQuery dapat mengatur untuk menampilkan <template> mana yang akan ditampilkan ke layar \textit{PC} maupun \textit{smartphone}.

Pada skripsi ini akan dibuat sebuah aplikasi permainan berbasis web yang memanfaatkan Socket.io. Aplikasi yang dibuat akan memanfaatkan \textit{personal computer (PC)} dan \textit{smartphone} untuk pengembangan aplikasinya. Para pemain akan mengkoneksikan \textit{smartphone} pada suatu \textit{PC} yang akan berfungsi sebagai \textit{console} dan \textit{smartphone} tersebut akan berfungsi sebagai \textit{controller} untuk memainkan permainannya. Socket.io akan digunakan sebagai koneksi antara \textit{smartphone} dan \textit{PC} dalam aplikasi permainan yang akan dibangun. Aplikasi permainan akan dibangun berdasarkan Node.js sehingga proses eksekusi JavaScript dapat dilakukan pada \textit{server}. Pengaturan struktur direktori di dalam pengembangan aplikasi akan menggunakan Express.js. Di dalam proses pengaturan elemen grafis yang dibutuhkan di dalam aplikasi, teknologi Canvas API akan digunakan di dalam pengembangannya. Elemen <template> akan digunakan untuk menampilkan setiap halaman-halaman web yang dibutuhkan di dalam pengembangan aplikasi permainan. Teknologi jQuery akan digunakan untuk pengaturan berbagai elemen HTML di dalam aplikasi. Aplikasi permainan akan menggunakan teknologi berbasis web sehingga untuk memainkannya, \textit{client} harus memiliki akses internet dan mengakses alamat aplikasi permainan menggunakan \textit{browser}.


%Hampir setiap \textit{game} menggunakan animasi yang ditujukan untuk memperjelas dan memperindah jalannya permainan tersebut. Pada skripsi ini, animasi akan digunakan dalam pembangungan aplikasinya. Oleh karena itu, akan digunakan teknologi bernama \textit{HTML5Canvas}. Teknologi ini merupakan bagian dari \textit{HTML element} yang dapat digunakan untuk menggambar suatu grafis melalui \textit{JavaScript} secara \textit{on the fly}. \textit{HTML5Canvas} dapat juga digunakan untuk membuat komposisi foto, membuat animasi, dan membuat \textit{real-time video processing} atau \textit{rendering}. Dengan fungsi-fungsi tersebut, skripsi ini akan memanfaatkan \textit{HTML5Canvas} dalam pembangunan aplikasi untuk membuat animasi.

%Aplikasi yang akan dibangun merupakan suatu aplikasi permainan berbasis web. Oleh karena itu, dibutuhkan suatu \textit{web server} yang akan mengatasi \textit{request-request} dari pengguna yang akan menggunakan aplikasi tersebut. Untuk pembangunan \textit{web server}, akan digunakan teknologi yang bernama \textit{NodeJs}. Teknologi ini merupakan \textit{open source}, \textit{cross-platform runtime environment} yang ditulis dalam \textit{JavaScript} yang digunakan untuk mengembangkan \textit{web server} dan jaringan pada sebuah aplikasi. Hal ini berarti bahwa \textit{NodeJs} menyediakan \textit{library} modul \textit{JavaScript} yang dapat digunakan secara bebas oleh pengguna. \textit{NodeJs} pun dapat dijalankan pada beberapa sistem operasi seperti \textit{OS X}, \textit{Microsoft Windows}, dan \textit{Linux}. Salah satu fitur dari \textit{NodeJs} yaitu seluruh \textit{API} yang ada dalam \textit{NodeJs} merupakan \textit{asynchronous}. Hal tersebut berarti \textit{server} berbasis \textit{NodeJs} tidak pernah menunggu \textit{API} untuk mengembalikan suatu data tertentu. Fitur tersebut sangat membantu sebuah aplikasi yang membutuhkan koneksi yang \textit{real-time} seperti dalam aplikasi \textit{game}. Oleh karena itu, skripsi ini akan memanfaatkan \textit{NodeJs} dalam pembangunan aplikasinya.


\section{Rumusan Masalah}
\label{sec:rumusan}

\begin{enumerate}
	\item Bagaimana membangun aplikasi permainan berbasis web dengan memanfaatkan Socket.io untuk penggunaan \textit{smartphone} sebagai pengendali permainan berbasis web ?
	\item Berapa \textit{latency} yang dihasilkan berdasarkan penggunaan Socket.io ? 
\end{enumerate}


\section{Tujuan}
\label{sec:tujuan}
\begin{enumerate}
	\item Mengetahui cara membangun aplikasi permainan berbasis web dengan memanfaatkan Socket.io untuk penggunaan \textit{smartphone} sebagai pengendali permainan berbasis web.
	\item Mengetahui jumlah \textit{latency} yang dihasilkan berdasarkan pemanfaatan Socket.io.
\end{enumerate}


\section{Batasan Masalah}
\label{sec:batasan}

Batasan masalah yang dibuat terkait dengan pengerjaan skripsi ini adalah sebagai berikut:

\begin{itemize}
	\item Aplikasi permainan yang dibuat merupakan permainan \textit{multiplayer} yang hanya bisa dimainkan oleh dua orang saja.
\end{itemize}


\section{Metodologi}
\label{sec:metlit}
Metodologi yang dilakukan dalam pengerjaan skripsi ini adalah sebagai berikut:

\begin{enumerate}
	\item Studi literatur mengenai :
		\begin{itemize}
			\item \textit{Socket.io} sebagai teknologi yang akan menghubungkan \textit{smartphone} dan \textit{PC}.
			\item \textit{Canvas API} yang akan digunakan untuk antarmuka permainan.
			\item \textit{Node.js} sebagai \textit{web server} dalam pembangunan aplikasi. 	
			\item \textit{Express.js} sebagai \textit{Node.js framework} yang akan digunakan untuk mengatur penyimpanan data.
			\item \textit{jQuery} yang akan digunakan dalam pengaturan elemen HTML.
			\item \textit{The Content Template element} yang akan digunakan untuk menampilkan halaman-halaman HTML.
		\end{itemize}
	\item Menganalisis aplikasi sejenis.
	\item Merancang antarmuka permainan pada \textit{PC} dan \textit{smartphone}. Antarmuka pada \textit{PC} akan berbeda dengan yang ada di \textit{smartphone}, karena \textit{smartphone} akan bekerja sebagai \textit{controller} dan \textit{PC} akan bekerja sebagai \textit{console}.
	\item Menyusun cara bermain aplikasi permainan yang dibangun.
	\item Mengimplementasi program aplikasi permainan berbasis web.
	\item Menganalisis \textit{latency} yang dihasilkan pada aplikasi.
	\item Melakukan eksperimen dan pengujian yang melibatkan responden.
\end{enumerate}

\section{Sistematika Pembahasan}
\label{sec:sispem}
Setiap bab dalam skripsi ini memiliki sistematika penulisan yang dijelaskan ke dalam poin-poin sebagai berikut:

\begin{enumerate}
	\item Bab 1 : Pendahuluan \\
	Membahas mengenai gambaran umum penelitian ini. Berisi tentang latar belakang, rumusan masalah, tujuan, batasan masalah, metode penelitian, dan sistematika penulisan.
	
	\item Bab 2 : Dasar Teori \\
	Membahas mengenai teori-teori yang mendukung berjalannya penelitian ini. Berisi tentang Socket.io, Node.js, Express.js, Canvas API, jQuery, dan The Content Template element.
	
	\item Bab 3 : Analisis \\
	Membahas mengenai analisa masalah.
	
	\item Bab 4 : Perancangan \\
	Membahas mengenai perancangan yang dilakukan sebelum melakukan tahapan implementasi.
	
	\item Bab 5 : Implementasi dan Pengujian \\
	Membahas mengenai implementasi dan pengujian yang telah dilakukan.
	
	\item Bab 6 : Kesimpulan dan Saran \\
	Membahas hasil kesimpulan dari keseluruhan penelitian ini dan saran-saran yang dapat diberikan untuk penelitian berikutnya.
\end{enumerate}