%versi 2 (8-10-2016)
\chapter{Landasan Teori}
\label{chap:teori}

Pada bab ini akan dijelaskan landasan teori mengenai \textit{WebSockets}, \textit{Socket.io}, \textit{Node.js}, \textit{Express.js}, dan \textit{Canvas API}.

\section{WebSockets}
\label{sec:WebSockets} 

\textit{WebSockets} merupakan \textit{Application Programming Interface (API)} yang memungkinkan sebuah aplikasi membuka sesi komunikasi interaktif antara \textit{browser} pengguna dan \textit{server}. Dengan API ini, pengguna dapat mengirim pesan ke \textit{server} dan menerima respon tanpa harus melakukan \textit{polling} pada \textit{server} terlebih dahulu.

Subbab-subbab berikut menjelaskan beberapa kelas dari \textit{WebSockets}.

\subsection{WebSocket}
Sebuah objek dari kelas \textit{WebSocket} menyediakan \textit{API} untuk membuat dan mengelola koneksi \textit{WebSocket} ke \textit{server}, dan juga untuk mengirim dan menerima data pada koneksi. Konstruktor pada kelas \textit{WebSocket} menerima satu parameter wajib dan satu parameter pilihan:

\textbf{WebSocket WebSocket(in DOMString url, in optional DOMString protocols);}

\textbf{url}, merupakan parameter wajib yang menunjukan URL mana yang akan direspon oleh \textit{WebSocket server}.

\textbf{protocols}, merupakan parameter pilihan yang dapat berupa satu protokol dengan tipe \textit{string}, atau beberapa protokol dengan tipe \textit{array of strings}. Apabila protokol spesifik tidak dimasukan pada parameter, maka akan diasumsikan sebagai \textit{string} kosong.

Beberapa atribut yang dimiliki oleh kelas \textit{WebSocket} yaitu sebagai berikut :

\begin{itemize}
	\item \textbf{readyState} \\ Atribut ini menunjukan status dari sebuah koneksi.
	\item \textbf{onclose} \\ Atribut ini merupakan \textit{event listener} yang akan dipanggil pada saat status koneksi \textit{WebSocket} berubah menjadi CLOSED.
	\item \textbf{onerror} \\ Atribut ini merupakan \textit{event listener} yang akan dipanggil apabila terjadi \textit{error}.
	\item \textbf{onmessage} \\ Atribut ini merupakan \textit{event listener} yang akan dipanggil apabila pesan dari server telah diterima.
	\item \textbf{onopen} \\ Atribut ini merupakan \textit{event listener} yang akan dipanggil pada saat status koneksi \textit{WebSocket} berubah menjadi OPEN.
\end{itemize}

Kelas \textit{WebSocket} memiliki dua buah \textit{method}, yaitu : 

\begin{itemize}
	\item close() \\ Berfungsi untuk menutup koneksi \textit{WebSocket} atau menghentikan apabila sedang ada proses koneksi. \textit{Method} ini memiliki tipe kembalian \textit{void}, sehingga tidak akan mengembalikan apapun.
	\item send() \\ Berfungsi untuk mengirim data ke \textit{server} melalui koneksi \textit{WebSocket}. \textit{Method} ini memiliki parameter \textbf{data} yang merupakan sebuah \textit{string} yang akan dikirimkan ke \textit{server}.
\end{itemize}

\subsection{CloseEvent}
\textit{CloseEvent} akan dikirim ke \textit{client} menggunakan protokol \textit{WebSockets} ketika koneksi sudah tertutup. \textit{Constructor} dari kelas ini yaitu : \\

\textbf{CloseEvent()} \\

Properti yang dimiliki oleh kelas ini yaitu : 

\begin{itemize}
	\item \textbf{CloseEvent.code} \\ Mengembalikan sebuah kode untuk menutup koneksi yang dikirimkan oleh \textit{server}.
	\item \textbf{CloseEvent.reason} \\ Mengembalikan alasan dari koneksi yang telah ditutup oleh \textit{server}
	\item \textbf{CloseEvent.wasClean} \\ Mengembalikan \textit{boolean} yang mengindikasi apakah sebuah koneksi sudah tertutup sepenuhnya atau belum.
\end{itemize}

\subsection{MessageEvent}
Kelas ini merepresentasikan pesan yang diterima oleh suatu objek tertentu. \textit{Constructor} dari kelas ini yaitu : \\

\textbf{MessageEvent()} \\

Beberapa properti yang dimiliki oleh kelas ini yaitu : 

\begin{itemize}
	\item \textbf{MessageEvent.data} \\ Merupakan data yang telah dikirimkan oleh pengirim.
	\item \textbf{MessageEvent.lastEventId} \\ Merepresentasikan \textit{ID} yang unik untuk sebuah \textit{Event}.
\end{itemize}


%-------------REVISED LIMIT OF WEBSOCKETS---------%

\section{Socket.io}
\label{sec:Socket.io}

\textit{Socket.io} merupakan \textit{JavaScript library} yang digunakan pada aplikasi web untuk melakukan koneksi secara \textit{realtime}. Teknologi ini memiliki dua bagian \textit{library}: bagian \textit{client} yang dijalankan pada \textit{web browser}, dan bagian \textit{server} yang jalankan untuk \textit{Node.js}. \textit{Socket.io} memiliki fitur-fitur yang beragam, seperti melakukan \textit{broadcast} ke beberapa \textit{sockets}, dan menyimpan data yang berhubungan dengan masing-masing \textit{client}.

\subsection{Connection}
Koneksi yang dimulai oleh \textit{Socket.io} dilakukan dengan cara \textit{handshake}. Hal tersebut merupakan bagian yang penting dalam protokol. \textit{Handshake} hanya dilakukan hanya pada saat memulai koneksi, pesan-pesan atau hal lain dalam protokol akan dikirimkan melalui \textit{socket}. 


\subsection{Messages}
Apabila koneksi telah dilakukan, seluruh komunikasi antara \textit{client} dan \textit{server} akan menggunakan pesan(\textit{message}) melalui \textit{socket}. Pesan yang akan dikirimkan harus diubah kedalam format yang sudah dispesifikasi oleh \textit{socket.io}.

Format yang sudah dispesifikasi oleh \textit{socket.io} bertujuan untuk menentukan jenis pesan dan data yang dikirimkan dalam pesan tersebut. Format pesan yang sudah dispesifikasi yaitu seperti berikut : 

[type] : [id] : [endpoint] (: [data])

\begin{itemize}
	\item \textit{type}, merupakan satu digit angka integer yang menunjukan jenis pesan yang akan dikirim.
	\item \textit{id}, merupakan identitas pesan yang terdiri dari beberapa digit angka.
	\item \textit{endpoint}, merupakan \textit{socket} tujuan yang akan menerima pesan yang sedang dikirim. Apabila tidak ada \textit{endpoint}, maka pesan akan dikirimkan ke \textit{default socket}.
	\item \textit{data}, merupakan data yang akan dikirim ke \textit{socket} tertentu. Pada kasus \textit{messages}, data akan dikirimkan dalam bentuk \textit{plain text}, sementara pada kasus \textit{events} akan dikirimkan dalam bentuk \textit{JSON}.
\end{itemize}

\section{Node.js}
\label{sec:Node.js}

\textit{Node.js} merupakan sebuah platform yang didesain untuk mengembangkan aplikasi berbasis web pada bagian \textit{web server}. \textit{Node.js} ditulis dalam sintaks bahasa pemrograman \textit{JavaScript} dan memiliki sifat \textit{non-blocking} yang berarti \textit{Node.js} tidak akan menunggu untuk mengerjakan \textit{request} selanjutnya. \textit{Node.js} dapat berjalan pada berbagai sistem operasi, seperti \textit{OS X}, \textit{Windows}, dan \textit{Linux}. Dengan begitu, tidak ada batasan dan perbedaan dalam menjalankan fungsi-fungsi yang ada pada berbagai sistem operasi.




\section{Express.js}
\label{sec:Express.js}

Express.js merupakan sebuah \textit{framework} aplikasi web untuk \textit{Node.js}. Express.js menyediakan fitur-fitur yang membuat pengembangan aplikasi web dapat bertahan lama. Teknologi ini pun merupakan modul \textit{node package manager (npm)} yang menjadi ketergantungan dalam suatu aplikasi. Agar dapat berjalan, seluruh \textit{file} yang dimiliki oleh \textit{framework} ini harus berada pada \textit{node\_modules} lokal dalam suatu projek tertentu. 


\section{Canvas API}
\label{sec:Canvas API}
 
Canvas API merupakan salah satu elemen \textit{HTML5} yang digunakan untuk membuat gambar grafis dalam aplikasi web. Teknologi ini memiliki fitur untuk membuat komposisi foto, membuat animasi, dan membuat \textit{real-time video processiong} atau \textit{rendering}. Untuk dapat menggunakan elemen \textit{canvas} harus menambahkan \textit{tag} <canvas> pada suatu halaman \textit{HTML}. \textit{Tag} <canvas> memiliki tiga atribut utama dimana atribut tersebut terdapat didalam kurung lancip pada \textit{HTML tag}. Atribut-atribut tersebut yaitu : 

\begin{itemize}
	\item id \\
	Merupakan nama yang akan digunakan sebagai referensi dalam kode JavaScript. Dimana nantinya nama tersebut akan merujuk ke \textit{tag} <canvas> yang memiliki nama yang sama.
	\item width \\
	Merupakan lebar dari \textit{canvas} yang dibuat.
	\item height \\
	Merupakan tinggi dari \textit{canvas} yang dibuat.
\end{itemize}

Menggunakan \textit{Canvas API} membutuhkan dasar yang kuat dalam menggambar , dan merubah bentuk-bentuk dasar dua dimensi. Berikut merupakan bentuk-bentuk dasar dua dimensi yang dapat digambar pada \textit{canvas}.

\begin{itemize}
	\item Rectangle \\
	Untuk menggambar suatu \textit{rectangle} (persegi), \textit{canvas} menyediakan tiga \textit{method} yaitu sebagai berikut:
	\begin{itemize}
		\item fillRect(x,y,width,height) \\
		Menggambar persegi dengan warna yang penuh mengisi bagian dalam persegi pada posisi x,y dengan ukuran persegi \textit{width} dan \textit{height}.
		\item strokeRect(x,y,width,height) \\
		Menggambar garis luar persegi pada posisi x,y dengan ukuran persegi \textit{width} dan \textit{height}.
		\item clearRect(x,y,width,height) \\
		Mengosongkan area tertentu dan membuat area tersebut transparan pada posisi x,y dengan ukuran persegi \textit{width} dan \textit{height}.
	\end{itemize}
	
	\item Paths\\
	Paths merupakan \textit{method} yang digunakan untuk menggambar seluruh bentuk pada \textit{canvas}. Path merupakan kumpulan titik, dan garis yang digambar diantara titik-titik tersebut. Untuk menggunakan path pada \textit{canvas}, dibutuhkan dua fungsi utama. Fungsi tersebut yaitu beginPath(), yang akan mulai membuka suatu path pada canvas, fungsi lainnya yaitu closePath(), yang akan menutup suatu path pada canvas.
	
	\item Arcs \\
	Sebuah \textit{arc} (garis lengkung) dapat berupa suatu lingkaran utuh atau bagian dari lingkaran tertentu. Untuk menggambar sebuah garis lengkung, \textit{Canvas API} menyediakan beberapa fungsi yang dapat digunakan. Salah satu fungsi tersebut yaitu: 
	
	arc(x, y, radius, startAngle, endAngle, anticlockwise).
	
	Nilai x dan y merupakan titik pusat dari lingkaran, dan radius merupakan jarak dari titik pusat ke suatu titik tertentu dimana garis lengkung akan digambar.\textit{startAngle} dan \textit{endAngle} ada dalam satuan radian, bukan derajat. \textit{anticlockwise} merupakan suatu \textit{boolean} yang menandakan apakah garis lengkung tersebut akan searah jarum jam atau tidak.
\end{itemize}

Selain menggambar suatu bentuk tertentu, pada \textit{canvas} pun dapat memberi warna pada bentuk yang sudah dibuat. \textit{Canvas API} memiliki properti yang digunakan untuk memberi warna dasar pada bagian dalam suatu bentuk di \textit{canvas} yang bernama fillStyle. Contoh penggunaan properti tersebut sebagai berikut:

context.fillStyle = ''red'';

Langkah tersebut akan memberikan warna merah pada suatu bentuk tertentu. Selain itu, ada beberapa cara yang dapat dilakukan untuk memberikan warna dasar pada suatu bentuk. Cara tersebut dijelaskan sebagai berikut:

context.fillStyle = ''rgb(255,0,0)'';\\
\textit{Method} rgb() akan menggunakan nilai RGB 24-bit pada saat memberikan warna pada suatu bentuk tertentu.\\

context.fillStyle = ''\#ff0000'';\\
Properti ini dapat menerima bilangan hex dalam bentuk \textit{string}.\\

context.fillStyle = ''rgba(255,0,0,1)'';\\
\textit{Method} rgba() akan menggunakan nilai 32-bit dengan nilai 8 bit di akhir yang merepresentasikan nilai \textit{alpha} pada suatu warna. 
