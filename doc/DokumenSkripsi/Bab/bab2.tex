%versi 2 (8-10-2016)
\chapter{Landasan Teori}
\label{chap:teori}

Pada bab ini akan dijelaskan landasan teori mengenai \textit{Node.js}, \textit{Express.js}, \textit{WebSockets}, \textit{Socket.io}, dan \textit{Canvas API}.

\section{Node.js}


\label{sec:Node.js}

\textit{Node.js} adalah \textit{JavaScript runtime} yang dibangun berdasarkan \textit{V8} yang merupakan \textit{JavaScript engine} milik perusahaan \textit{Google} \cite{dahl:09:nodejs}. \textit{Node.js} memiliki model \textit{event-driven}, dan \textit{non-blocking I/O} yang membuat teknologi tersebut efisien dalam implementasinya. Teknologi ini menyediakan beberapa kelas yang berfungsi untuk mengimplementasi fitur-fitur yang dimiliki.

Beberapa kelas yang terdapat pada \textit{Node.js} adalah sebagai berikut: 

\subsection{HTTP}
\textit{Interfaces} \textit{HTTP} pada \textit{Node.js} digunakan untuk menangani \textit{request} dari protokol \textit{HTTP} yang secara \textit{native} sulit untuk digunakan \cite{dahl:09:nodejsdocs}. \textit{Interface} ini akan menangani protokol \textit{HTTP} dengan tidak melakukan \textit{buffer} pada seluruh \textit{request} atau \textit{responses}.

%Berikut akan dijelaskan kelas-kelas yang ada pada \textit{interface} \textit{HTTP}.

%\begin{enumerate}
%	\item \textbf{http.IncomingMessage} \\ 
%	Objek dari kelas ini akan dibuat oleh kelas \textit{http.Server} atau \textit{http.ClientRequest} dan memasukannya sebagai argumen suatu \textit{event} \textit{'request'} dan \textit{'response'}. Objek tersebut dapat digunakan untuk mengakses status \textit{response}, \textit{headers}, dan data. Kelas ini mengimplementasi \textit{interface Readable Stream}, beserta \textit{method, events,} dan properti yang ada didalamnya.
%	
%	\textbf{Properti:}
%	\begin{itemize}
%		\item \textbf{message.headers} \\ \textbf{Kembalian:} \textit{headers} milik objek \textit{request/response.}
%		\item \textbf{message.rawHeaders} \\ \textbf{Kembalian:} bentuk \textit{raw} dari \textit{headers} milik objek \textit{request/response}.
%		\item \textbf{message.statusCode} \\ \textbf{Kembalian:} tiga dijit kode status \textit{HTTP response}. Contoh: 404.
%		\item \textbf{message.statusMessage} \\ \textbf{Kembalian:} pesan status \textit{HTTP response} Contoh: \textit{OK} atau \textit{Internal Server Error.}
%		\item \textbf{message.url} \\ \textbf{Kembalian:} \textit{URL string} yang muncul pada permintaan \textit{HTTP}.
%	\end{itemize}
	
%	\item \textbf{http.ClientRequest} \\ 
%	Objek dari kelas ini dibuat dalam kelas ini sendiri dan dikembalikan dari \textit{method http.request()}. Objek ini merepresentasikan permintaan yang sedang berlangsung dimana \textit{header} objek tersebut sudah berada dalam antrian. \textit{Header} masih dapat diubah dengan menggunakan \textit{setHeader(name, value)} dan \textit{removeHeader(name)}. \textit{Header} yang asli akan dikirim bersamaan dengan \textit{chunk} pertama dari suatu data atau saat memanggil \textit{request.end()}.
%	
%	Beberapa \textbf{event} yang dimiliki oleh kelas ini yaitu sebagai berikut:
%	\begin{itemize}
%		\item \textbf{'connect'} \\ Dipancarkan saat \textit{server} merespon kepada permintaan.
%		\item \textbf{'response'} \\ Dipancarkan saat suatu respon diterima atas permintaan saat ini.
%		\item \textbf{'timeout'} \\ Dipancarkan saat suatu \textit{socket} telah mencapai batas waktu untuk tidak aktif.
%	\end{itemize}
%	
%	Beberapa \textit{method} yang dimiliki oleh kelas ini yaitu sebagai berikut:
%	\begin{itemize}
%		\item \textbf{request.end([data[,encoding]][,callback])} \\ 
%		\textbf{Parameter:} 
%		\begin{itemize}
%			\item \textbf{data} \\tipe: \textbf{string} atau \textbf{Buffer} \\ Data yang akan dikirim.
%			\item \textbf{encoding} \\tipe: \textbf{string} \\ Bersifat opsional dan akan bernilai \textit{utf8} apabila tipe parameter \textit{data} berupa \textit{string}.
%			\item \textbf{callback} \\tipe:	\textbf{Function} \\ Fungsi callback
%		\end{itemize}
%		
%		\textit{Method} ini akan mengakhiri proses pengiriman permintaan.
		
%		\item \textbf{request.getHeader(name)} \\
%		\textbf{Parameter:} 
%		\begin{itemize}
%			\item \textbf{name} \\tipe: \textbf{string} \\ Nama dari \textit{header} yang dibutuhkan.
%		\end{itemize}
%		\textbf{Kembalian:} suatu \textit{string} yang sesuai dengan parameter.
%		
%		\textit{Method} ini akan membaca seluruh \textit{header} dalam permintaan dan mengembalikan bagian yang sesuai dengan parameter. Berikut contoh implementasi dari \textit{method} ini:
%\begin{lstlisting}
%const contentType = request.getHeader('Content-Type');
%\end{lstlisting}
		
%		\item \textbf{request.removeHeader(name)} \\ 
%		\textbf{Parameter:}
%		\begin{itemize}
%			\item \textbf{name} \\tipe: \textbf{string} \\ Nama dari \textit{header} yang dibutuhkan.
%		\end{itemize}
%		\textit{Method} ini akan menghapus \textit{header} yang sudah ada pada objek \textit{header}. Berikut contoh implementasi dari \textit{method} ini:
%\begin{lstlisting}
%request.removeHeader('Content-Type');
%\end{lstlisting}
%		
%		\item \textbf{request.setHeader(name, value)} \\ 
%		\textbf{Parameter:}
%		\begin{itemize}
%			\item \textbf{name} \\tipe: \textbf{string} \\ Nama dari \textit{header} yang dibutuhkan.
%			\item \textbf{value} \\tipe: \textbf{value} \\ Nilai yang akan dimasukan pada objek \textit{header}
%		\end{itemize}
%		
%		\textit{Method} ini akan menetapkan suatu nilai kepada objek \textit{header}. Berikut contoh implementasi \textit{method} ini:
%\begin{lstlisting}
%request.setHeader('Content-Type', 'application/json');
%\end{lstlisting}
%	\end{itemize}
	
%	\item \textbf{http.Server} \\ 
%	Kelas ini merupakan turunan dari \textit{net.Server}. \textit{Event} yang dimiliki kelas ini yaitu sebagai berikut:
%	\begin{itemize}
%		\item \textbf{'close'} \\ Dipancarkan apabila \textit{server} sudah ditutup.
%		
%	\end{itemize}
%	
%	Beberapa properti yang dimiliki oleh kelas ini yaitu:
%	\begin{itemize}
%		\item \textbf{server.listening} \\ Mengembalikan \textit{boolean} yang menandakan apakah \textit{server} melakukan proses \textit{listening} untuk suatu koneksi atau tidak.
%		
%		\item \textbf{server.maxHeadersCount} \\ Mengembalikan \textit{number} yang menandakan batas maksimum suatu \textit{headers} yang masuk. Nilai default dari properti ini yaitu 2000.
%		
%		\item \textbf{server.timeout} \\ Mengembalikan \textit{number} yang menandakan \textit{timeout} dalam milidetik.
%	\end{itemize}
%	
%	Beberapa \textit{Method} yang dimiliki oleh kelas ini yaitu:
%	\begin{itemize}
%		\item \textbf{server.listen()} \\ Memulai \textit{server HTTP} melakukan proses \textit{listening} untuk suatu koneksi.
%		
%		\item \textbf{server.setTimeout([msecs][,callback])} \\ \textbf{Parameter:}
%		\begin{itemize}
%			\item \textbf{msec} nilai \textit{timeout} dalam milidetik. secara default bernilai 120000 (2 menit).
%			\item \textbf{callback} fungsi \textit{callback}.
%		\end{itemize}
%		\textbf{Kembalian:} objek \textit{server}.
%		
%		\textit{Method} ini menetapkan nilai \textit{timeout} untuk \textit{sockets} dan memancarkan \textit{event 'timeout'} pada objek \textit{Server}.
%		
%		\item \textbf{server.close([callback])} \\ \textbf{Parameter:}
%		\begin{itemize}
%			\item \textbf{callback} fungsi \textit{callback}.
%		\end{itemize} 
%		
%		\textit{Method} ini menghentikan \textit{server} untuk menerima koneksi baru.
%	\end{itemize}
%\end{enumerate} 

Salah satu \textit{Method} yang dimiliki oleh \textit{HTTP} yaitu sebagai berikut:
\begin{itemize}
	\item \textbf{http.createServer([options][,requestListener])} \\ \textbf{Parameter:}
	\begin{itemize}
		\item \textbf{options} objek-objek sebagai berikut:
			\begin{itemize}
				\item \textbf{IncomingMessage} menentukan objek dari kelas \textit{IncomingMessage} yang akan digunakan.
				\item \textbf{ServerResponse} menentukan objek dari kelas \textit{ServerResponse} yang akan digunakan.
			\end{itemize}
			
		\item \textbf{requestListener} fungsi yang akan secara otomatis ditambahkan pada \textit{event} \textit{'request'} milik kelas \textit{http.Server}.
	\end{itemize}
	\textbf{Kembalian:} objek \textit{http.Server}
	
	\textit{Method} ini akan membuat objek \textit{http.Server} untuk menangani \textit{request} dari \textit{client} dan memberikan \textit{response} kepada \textit{client}. Fungsi yang diberikan pada \textit{method} ini akan dipanggil satu kali setiap \textit{request} dibuat kepada \textit{server}.
	
%	\item \textbf{http.request(options[,callback])} \\ 
%	\textbf{Parameter:}
%	\begin{itemize}
%		\item \textbf{options} \\ Dapat berupa \textit{Object, string} atau \textit{URL}. Berikut jenis-jenis \textit{options} yang dapat menjadi parameter:
%		\begin{itemize}
%			\item \textbf{protocol} \\tipe: \textbf{string} \\ Protokol yang digunakan.
%			\item \textbf{host} \\tipe: \textbf{string} \\ Nama domain atau alamat \textit{IP} milik server.
%			\item \textbf{hostname} \\tipe: \textbf{string} \\ Nama lain untuk \textit{host}.
%			\item \textbf{port} \\tipe: \textbf{number} \\ \textit{Port} untuk \textit{server}.
%			\item \textbf{path} \\tipe: \textbf{string} \\ \textit{Path} untuk permintaan.
%			\item \textbf{headers} \\tipe: \textbf{Object} \\ Objek yang berisi permintaan \textit{headers}.
%			\item \textbf{timeout} \\tipe: \textbf{number} \\ Nomor yang menentukan \textit{timeout} dari suatu \textit{socket} dalam milidetik.
%		\end{itemize}
%		
%		\item \textbf{callback} \\tipe: \textbf{Function} \\ Fungsi \textit{callback}.
%	\end{itemize}
%	
%	\textbf{Kembalian:} objek dari kelas \textit{http.ClientRequest}.
%	
%	\textit{Method} ini digunakan untuk menangani permintaan \textit{HTTP} pada \textit{server}. Berikut contoh implementasi dari \textit{method} ini:
%	\begin{lstlisting}
%	const options = {
%	hostname: 'www.google.com',
%	port: 80,
%	path: '/upload',
%	method: 'POST',
%	headers: {
%	'Content-Type': 'application/x-www-form-urlencoded',
%	'Content-Length': Buffer.byteLength(postData)
%	}
%	};
%	
%	const req = http.request(options, (res) => {
%	console.log(`STATUS: ${res.statusCode}`);
%	console.log(`HEADERS: ${JSON.stringify(res.headers)}`);
%	res.setEncoding('utf8');
%	res.on('data', (chunk) => {
%	console.log(`BODY: ${chunk}`);
%	});
%	res.on('end', () => {
%	console.log('No more data in response.');
%	});
%	});
%	
%	req.end();
%	\end{lstlisting}
	
\end{itemize}

\subsection{Path}
Modul \textit{Path} menyediakan fungsi untuk mengatur akses suatu \textit{file} dan direktori \cite{dahl:09:nodejsdocs}. Modul tersebut dapat diakses dengan cara sebagai berikut:

\begin{lstlisting}
const path = require('path');
\end{lstlisting}

Salah satu method yang dimiliki oleh modul \textit{Path} adalah : 

\begin{itemize}
	\item \textbf{path.join([...paths])} \\ 
	\textbf{parameter:} 
	\begin{itemize}
		\item \textbf{...paths} \\ tipe: \textbf{String} \\ Urutan suatu lokasi \textit{file} yang akan digunakan.
	\end{itemize}

	\textbf{kembalian:} String
	
	\textit{Method} ini akan menggabungkan seluruh bagian-bagian \textit{path} dengan menormalisasinya dan mengembalikan bentuk \textit{path} yang menyeluruh.
\end{itemize}

%\subsection{Events}
%\textit{Node.js} dibangun berdasarkan arsitektur \textit{event-driven} dengan sifat \textit{asynchronous}, dimana jenis-jenis objek tertentu akan memancarkan suatu \textit{events} secara berkala dan akan memanggil objek \textit{Function} ("\textit{listeners}").
%
%Semua objek yang memancarkan \textit{events} merupakan turunan dari kelas \textit{EventEmitter}. Objek tersebut akan memanggil \textit{method} \textit{eventEmitter.on()} yang memungkinkan satu atau beberapa fungsi dapat ditangani dalam suatu \textit{event} yang dipancarkan oleh objek saat ini.
%
%Kelas \textit{EventEmitter} dapat didefinisikan dengan memanggil modul \textit{events} seperti berikut :
%
%\begin{lstlisting}
%const EventEmitter = require('events');
%\end{lstlisting}
%
%Sebuah \textit{EventEmitter} akan memancarkan \textit{'event'} \textit{'newListener'} pada saat \textit{listeners} baru akan ditambahkan, dan \textit{'removeListener'} akan dipancarkan saat \textit{listeners} saat ini akan dihapus.
%
%Berikut merupakan beberapa \textit{method} yang dimiliki oleh kelas \textit{EventEmitter}:
%
%\begin{itemize}
%	\item \textbf{eventEmitter.on(eventName, listener)} \\ \textbf{Parameter:}
%	\begin{itemize}
%		\item \textit{eventName}, nama dari suatu \textit{event} yang akan dipancarkan.
%		\item \textit{listener}, suatu fungsi \textit{callback} yang akan menangani \textit{event} dari \textit{eventName}. 
%	\end{itemize} 
%	\textbf{Kembalian:} referensi kepada \textit{EventEmitter}.
%	
%	\textit{Method} ini berfungsi untuk mencatat suatu listener yang akan digunakan. Fungsi \textit{listener} yang menjadi parameter \textit{method} ini akan ditambahkan ke \textit{index} terakhir dari \textit{array of listeners} pada \textit{eventName}. Tidak akan ada pengecekan apakah fungsi \textit{listener} sudah dimasukan sebelumnya. Oleh karena itu, pemanggilan \textit{eventName} dan \textit{listener} secara berulang akan menyebabkan fungsi \textit{listener} dimasukan kedalam \textit{array} dan dipanggil secara berulang. \textit{Method} ini juga akan mengembalikan \textit{reference} kepada \textit{EventEmitter}, sehingga pemanggilan dapat saling menyambung dengan pemanggilan lainnya.
%	
%	Berikut merupakan contoh implementasi dari \textit{method} ini:
%\begin{lstlisting}
%const EventEmitter = require('events');
%	
%class MyEmmit extends EventEmitter {}
%	
%const myEmmit = new MyEmitter();
%	
%myEmmit.on('event', () => {
%console.log('suatu event telah terjadi');	
%});
%\end{lstlisting}
%	
%	\item \textbf{eventEmitter.emit(eventName)} \\ \textit{Method} ini berfungsi untuk memicu suatu \textit{event} yang akan dipancarkan. \textbf{Parameter:}
%	\begin{itemize}
%		\item \textit{eventName}, nama dari sebuah \textit{event} yang akan dipancarkan.
%		\item \textit{...args}, argumen tambahan yang akan diberikan pada \textit{eventName}. 
%	\end{itemize}
%	\textbf{Kembalian:} \textit{true} apabila \textit{event} memiliki \textit{listener}, false jika tidak.
%	
%	\textit{Method} ini akan memanggil masing-masing \textit{listener} yang sudah dicatat oleh \textit{eventEmitter.on()} dalam \textit{array of listeners} secara sinkronis, dimana beberapa \textit{listener} tersebut mengacu pada \textit{eventName} yang sama. Argumen yang diterima dari parameter akan diberikan pada masing-masing \textit{listener}.
%	
%	Berikut merupakan contoh implementasi dari \textit{method} ini:
%\begin{lstlisting}
%const EventEmitter = require('events');
%	
%class MyEmmit extends EventEmitter {}
%	
%const myEmmit = new MyEmitter();
%	
%myEmmit.on('event', () => {
%console.log('suatu event telah terjadi');	
%});
%myEmitter.emit('event');
%\end{lstlisting} 
%\end{itemize}

%\subsection{Stream}
%Kelas ini digunakan untuk menangani aliran data yang terjadi pada \textit{Node.js}. Data yang ditangani dapat berjumlah banyak dan akan menghabiskan banyak memori apabila tidak ditangani dengan baik. Oleh karena itu, modul \textit{stream} menyediakan fitur-fitur yang memudahkan penanganan aliran data.
%
%Ada empat tipe dasar \textit{stream} dalam \textit{Node.js}:
%\begin{itemize}
%	\item \textbf{Readable} \\ \textit{Streams} yang dapat membaca data dari sumber eksternal tertentu.
%	\item \textbf{Writable} \\ \textit{Streams} yang dapat menulis data dan mengirimkannya ke sumber external tertentu.
%	\item \textbf{Duplex} \\ \textit{Streams} yang dapat membaca dan menulis data sekaligus.
%	\item \textbf{Transform} \\ \textit{Duplex streams} yang dapat memodifikasi atau mengubah data dimana data tersebut dapat dilihat langsung hasil perubahannya.
%\end{itemize}
%
%\begin{enumerate}
%	\item \textbf{Readable Stream} \\
%	Merupakan abstraksi untuk sumber data yang digunakan. Berikut merupakan contoh dari \textit{Readable Stream} pada \textit{Node.js}:
%	\begin{itemize}
%		\item \textbf{HTTP responses} pada \textit{client}
%		\item \textbf{HTTP requests} pada \textit{server}
%		\item \textbf{fs read streams}
%	\end{itemize}
%	Seluruh \textit{Readable streams} mengimplementasi \textit{interface} yang didefinisikan oleh kelas \textit{stream.Readable}.
%	
%	\begin{itemize}
%		\item \textbf{stream.Readable} \\
%		\textbf{Events:}
%		\begin{itemize}
%			\item \textbf{'close'} \\ \textit{Event} ini dipancarkan saat suatu \textit{stream} atau sumber lain telah ditutup. \textit{Event} ini menandakan tidak akan ada \textit{event} lagi yang akan dipancarkan, dan tidak ada komputasi lain yang akan dilakukan.
%\begin{lstlisting}
%const readable = getReadableStreamSomehow();
%readable.on('close', (chunk) => {
%console.log(`Stream telah ditutup`);
%});
%			\end{lstlisting}
%			
%			\item \textbf{'data'} \\ \textit{Event} ini akan dipancarkan setiap kali suatu \textit{stream} melepas kepemilikan sebuah data kepada pemakai. Hal tersebut dapat terjadi setiap suatu \textit{stream} berganti menjadi mode \textit{flowing} dengan memanggil \textit{readable.pipe()}, \textit{readable.resume()}, atau dengan menghubungkan \textit{listener callback} pada \textit{'data' event}. Contoh implementasi:
%\begin{lstlisting}
%const readable = getReadableStreamSomehow();
%readable.on('data', (chunk) => {
%console.log(`Menerima data sebesar ${chunk.length} bytes.`);
%});
%			\end{lstlisting}
%			
%			\item \textbf{'end'} \\ \textit{Event} ini dipancarkan saat tidak ada lagi data yang akan digunakan dari \textit{stream}. Contoh implementasi:
%\begin{lstlisting}
%const readable = getReadableStreamSomehow();
%readable.on('end', () => {
%console.log('Tidak akan ada data yang dikirimkan.');
%});
%\end{lstlisting}
%			
%			\item \textbf{'error'} \\ \textit{Event} ini akan dipancarkan oleh \textit{Readable} setiap saat. \textit{Event} ini dapat terjadi apabila \textit{stream} tidak dapat menyediakan data dikarenakan kesalahan internal, atau ketika implementasi suatu \textit{stream} mencoba mengirimkan \textit{chunk} dari data yang tidak sesuai. Contoh implementasi:
%\begin{lstlisting}
%const readable = getReadableStreamSomehow();
%readable.on('error', () => {
%console.error('Terjadi kesalahan');
%});
%\end{lstlisting}
%			
%			
%		\end{itemize}
%		
%		\textbf{Method:}
%		\begin{itemize}
%			\item \textbf{readable.pipe(destination[, options])} \\ \textbf{Parameter:}
%			\begin{itemize}
%				\item \textbf{destination} \\tipe: \textit{stream.Writable} \\ Destinasi untuk menulis suatu data.
%				\item \textbf{options} \\tipe: \textit{Object} \\ Bersifat opsional, dapat berupa objek sebagai berikut:
%				\begin{itemize}
%					\item \textbf{end} \\tipe: \textit{boolean} \\ Mengakhiri \textit{writer} yang melakukan proses menulis saat \textit{reader} telah selesai. Nilai \textit{default} parameter ini yaitu \textit{true}.
%				\end{itemize}
%			\end{itemize}
%			\textbf{Kembalian:} referensi yang menuju \textit{destination}.
%			
%			\textit{Method} ini akan menghubungkan \textit{Writable stream} pada objek \textit{readable}, sehingga dapat berubah menjadi mode \textit{flowing} secara otomatis dan akan menaruh seluruh data pada \textit{Writable} yang sudah terhubung. \textit{Method} ini mengembalikan referensi yang menuju \textit{destination}, sehingga \textit{pipe streams} dapat saling menyambung dengan \textit{pipe streams} lainnya.
%			
%			Berikut merupakan contoh implementasi dari \textit{method} ini:
%\begin{lstlisting}
%const fs = require('fs');
%			
%const readable = getReadableStreamSomehow();
%const writable = fs.createWriteStream('file.txt');
%			
%//seluruh data dari objek readable masuk ke 'file.txt'
%readable.pipe(writable); 
%\end{lstlisting}
%			
%		\end{itemize}
%	\end{itemize}
%\end{enumerate}

\subsection{Module}
Pada aplikasi berbasis \textit{Node.js}, setiap \textit{file} yang terdapat dalam pembangunan aplikasi dianggap sebagai modul-modul yang terpisah satu sama lain \cite{dahl:09:nodejsdocs}. Variabel dan fungsi yang terdapat pada satu \textit{file}, atau modul, hanya dapat digunakan pada satu lingkup modul tersebut. Suatu modul tidak dapat menggunakan variabel atau fungsi yang terdapat pada modul lainnya. Oleh karena itu, apabila variabel dan fungsi yang terdapat pada satu modul dapat digunakan oleh modul yang lain, diperlukan cara tertentu. Cara tersebut dapat dilakukan seperti berikut:

\begin{lstlisting}
module.exports = functions

\\OR

module.exports = object
\end{lstlisting}

\textit{module.exports} merupakan objek yang dibentuk oleh sistem \textit{Module}. Dengan menggunakan cara ini, suatu modul dapat berubah menjadi global dan dapat diakses oleh modul-modul lain.

%------------- END OF NODE.JS --------------

\section{Express.js}
\label{sec:Express.js}

\textit{Express.js} merupakan \textit{framework} aplikasi web untuk \textit{Node.js} \cite{tj:10:expressjs}. \textit{Express.js} menyediakan fitur-fitur untuk web dan aplikasi \textit{mobile} agar dapat bertahan lama. \textit{Framework} ini digunakan untuk mengatur struktur direktori dalam pengembangan aplikasi permainan berbasis web. Untuk dapat menggunakan \textit{Express.js}, dapat dilakukan langkah sebagai berikut: 
\begin{lstlisting}
var express = require('express');
var app = express();
\end{lstlisting}

Dengan begitu, fitur-fitur yang terdapat pada \textit{Express.js} dapat digunakan untuk pengembangan aplikasi tertentu.

subbab-subbab berikut akan menjelaskan kelas-kelas yang terdapat pada \textit{Express.js}.

\subsection{express()}
Untuk membuat aplikasi \textit{Express}, langkah yang dilakukan adalah sebagai berikut:
\begin{lstlisting}
const express = require('express');
const app = express();
\end{lstlisting}

Beberapa \textit{method} yang dimiliki oleh fungsi \textit{express()} adalah sebagai berikut:

\begin{itemize}
%	\item \textbf{express.Router([options])} \\ \textbf{Parameter:} 
%	\begin{itemize}
%		\item \textbf{options} bersifat opsional dan akan menentukan sifat dari objek \textit{router}. Parameter ini dapat berupa beberapa jenis seperti berikut:
%		\begin{itemize}
%			\item \textit{caseSensitive} memungkinkan \textit{case-sensitive}. Dapat bernilai \textit{true} atau \textit{false}. Secara default akan bernilai \textit{false}.
%			
%			\item \textit{strict} memungkinkan \textit{strict routing}. Dapat bernilai \textit{true} atau \textit{false}. Apabila bernilai \textit{false}, maka parameter \textit{'/foo'} dan \textit{'/foo/'} akan dianggap sama oleh \textit{router}.
%		\end{itemize}
%	\end{itemize}

	\item \textbf{express.static(root, [,options])} \\
	\textbf{Parameter:}
	\begin{itemize}
		\item \textbf{root} \\ tipe: \textbf{String} \\ Menentukan direktori \textit{root} yang akan digunakan untuk menyediakan \textit{static file}.
		\item \textbf{options} Objek-objek seperti berikut:
			\begin{itemize}
				\item \textbf{dotfiles} menentukan bagaimana mengatasi suatu \textit{dotfiles} (suatu \textit{file} atau direktori yang dimulai dengan tanda ".").
			\end{itemize}
	\end{itemize}
	
	\textit{Method} ini akan menyediakan cara agar dapat menggunakan \textit{static file} yang ada.
	
	\item \textbf{express.Router([options])} \\
	\textbf{Parameter:}
	\begin{itemize}
		\item \textbf{options} merupakan parameter pilihan yang akan menentukan perilaku dari \textit{server}. Parameter dapat berupa objek sebagai berikut:
		\begin{itemize}
			\item \textbf{caseSensitive} membedakan huruf besar dan huruf kecil.
		\end{itemize}
	\end{itemize}

	\textit{Method} ini akan membuat objek \textit{router} yang dapat digunakan dengan menambahkan \textit{middleware} dan \textit{method HTTP} seperti \textit{get, post,} dan \textit{put}.
	 
	
\end{itemize}

\subsection{Application}
Kelas ini akan menangani berbagai proses yang terjadi dalam aplikasi \textit{Express} seperti melakukan \textit{routing} terhadap \textit{HTTP requests}, mengatur \textit{middleware}, \textit{rendering} sebuah \textit{HTML views}, dan mendaftarkan \textit{template engine} tertentu \cite{tj:10:expressjsdocs}. Untuk dapat melakukan fungsi-fungsi tersebut dapat dilakukan langkah berikut:

\begin{lstlisting}
const express = require('express');
const app = express();
\end{lstlisting}

Baris pertama dari potongan kode tersebut berarti variabel \textit{express} memanggil modul \textit{'express'} agar dapat mengakses fungsi-fungsi yang ada pada modul tersebut. Sedangkan baris kedua, Objek \textit{app} memanggil fungsi \textit{express()} yang telah didapatkan dari variabel \textit{express}.

Kelas ini memiliki beberapa \textit{method} sebagai berikut:

\begin{itemize}
%	\item \textbf{app.all(path, callback[, callback ...])} \\ \textbf{Parameter:} 
%	\begin{itemize}
%		\item \textbf{path} suatu \textit{path} yang akan ditangani oleh \textit{middleware}. Dapat berupa \textit{string}, \textit{path pattern}, atau \textit{array} dari kombinasi \textit{string} dan \textit{path pattern}.
%		
%		\item \textbf{callback} merupakan fungsi \textit{callback}, dimana fungsi tersebut dapat berupa fungsi \textit{middleware}, kumpulan dari fungsi \textit{middleware} (yang dipisahkan dengan menggunakan koma), fungsi \textit{array of middleware}, atau kombinasi dari seluruh \textit{item} tersebut.
%	\end{itemize}
%	
%	\textit{Method} ini dapat menangani seluruh \textit{HTTP requests} seperti \textit{GET, POST, PUT,} dan \textit{DELETE}. Berikut merupakan contoh implementasi dari \textit{method} ini:
%\begin{lstlisting}
%app.all('/about', function(req, res, next){
%console.log('Mengakses bagian about ...');
%next(); //bagian ini akan menuju ke handler berikutnya
%});
%\end{lstlisting}
%	
%	\item \textbf{app.get(path, callback[, callback ...])} \\ \textbf{Parameter:}
%	\begin{itemize}
%		\item \textbf{path} suatu \textit{path} yang akan ditangani oleh \textit{middleware}. Dapat berupa \textit{string}, \textit{path pattern}, atau \textit{array} dari kombinasi \textit{string} dan \textit{path pattern}.
%		
%		\item \textbf{callback} merupakan fungsi \textit{callback}, dimana fungsi tersebut dapat berupa fungsi \textit{middleware}, kumpulan dari fungsi \textit{middleware} (yang dipisahkan dengan menggunakan koma), fungsi \textit{array of middleware}, atau kombinasi dari seluruh \textit{item} tersebut.
%	\end{itemize}
%	
%	\textit{Method} ini akan mengarahkan \textit{HTTP GET requests} pada \textit{path} dengan fungsi \textit{callback} tertentu. Berikut merupakan contoh implementasi dari \textit{method} ini:
%\begin{lstlisting}
%app.get('/', function(req, res){
%res.send('Mengirimkan GET request pada homepage');
%});
%	\end{lstlisting}
%	
%	\item \textbf{app.post(path, callback[, callback ...])} \\ \textbf{Parameter:}
%	\begin{itemize}
%		\item \textbf{path} suatu \textit{path} yang akan ditangani oleh \textit{middleware}. Dapat berupa \textit{string}, \textit{path pattern}, atau \textit{array} dari kombinasi \textit{string} dan \textit{path pattern}.
%		
%		\item \textbf{callback} merupakan fungsi \textit{callback}, dimana fungsi tersebut dapat berupa fungsi \textit{middleware}, kumpulan dari fungsi \textit{middleware} (yang dipisahkan dengan menggunakan koma), fungsi \textit{array of middleware}, atau kombinasi dari seluruh \textit{item} tersebut.
%	\end{itemize}
%	
%	\textit{Method} ini akan mengarahkan \textit{HTTP POST requests} pada \textit{path} dengan fungsi \textit{callback} tertentu. Berikut merupakan contoh implementasi dari \textit{method} ini:
%\begin{lstlisting}
%app.post('/', function(req, res){
%res.send('Mengirimkan POST requests pada homepage');
%});
%\end{lstlisting}
%	
%	\item \textbf{app.route(path)} \\ \textbf{Parameter:}
%	\begin{itemize}
%		\item \textbf{path} suatu \textit{path} yang akan ditangani oleh \textit{middleware}. Dapat berupa \textit{string}, \textit{path pattern}, atau \textit{array} dari kombinasi \textit{string} dan \textit{path pattern}.
%	\end{itemize}
%	
%	\textit{Method} ini akan mengembalikan instansi dari satu \textit{route}, yang kemudian dapat digunakan untuk menangani \textit{HTTP request} dengan \textit{middleware} tertentu. Berikut merupakan contoh implementasi dari \textit{method} ini:
%\begin{lstlisting}
%app.route('/buku').get(function(req, res){
%res.send('Mendapatkan suatu buku');
%});
%\end{lstlisting}
%	
	\item \textbf{app.set(name, value)} \\
	\textbf{Parameter:}
	\begin{itemize}
		\item \textbf{name} nama tertentu yang dapat digunakan untuk menentukan perilaku dari suatu \textit{server}.
		\item \textbf{value} nilai yang akan ditetapkan pada parameter \textit{name}.
	\end{itemize}
	\textbf{Kembalian:} -
	
	Method ini akan menetapkan suatu \textit{value} tertentu pada parameter \textit{name}.

	\item \textbf{app.use([path,] callback[, callback...])} \\ \textbf{Parameter:} 
	\begin{itemize}
		\item \textbf{path} suatu \textit{path} yang akan ditangani oleh \textit{middleware}. Dapat berupa \textit{string}, \textit{path pattern}, atau \textit{array} dari kombinasi \textit{string} dan \textit{path pattern}.
		
		\item \textbf{callback} merupakan fungsi \textit{callback}, dimana fungsi tersebut dapat berupa fungsi \textit{middleware}, kumpulan dari fungsi \textit{middleware} (yang dipisahkan dengan menggunakan koma), fungsi \textit{array of middleware}, atau kombinasi dari seluruh \textit{item} tersebut.
	\end{itemize}
	
	\textit{Method} ini akan menghubungkan \textit{middleware} atau suatu fungsi tertentu dengan \textit{path} yang sudah ditentukan. Dalam implementasi \textit{method} ini, urutan penempatan pada baris kode sangat berpengaruh. Setelah \textit{app.use()} dieksekusi, maka suatu \textit{request} tidak akan mengeksekusi \textit{middleware} yang ada dibawah baris kode \textit{app.use()}. 
	
%	Berikut merupakan contoh implementasi dari \textit{method} ini:
%\begin{lstlisting}
%//request hanya akan sampai pada middleware ini
%app.use(function(req, res){
%res.send('Hanya sampai sini saja');
%});
%	
%//request tidak akan mengeksekusi baris ini
%app.get('/', function(req, res){
%res.send('Hello World!');
%});
%\end{lstlisting}

	\item \textbf{app.listen(port, [hostname], [backlog], [callback])} \\ \textbf{Parameter:}
	\begin{itemize}
		\item \textbf{port} nomor yang akan dituju oleh server.
		\item \textbf{hostname} \textit{string} yang diberikan pada gawai tertentu agar dapat dikenali. Parameter ini bersifat opsional.
		\item \textbf{backlog} nomor yang menentukan ukuran maksimal dalam antrian koneksi yang tertunda.Parameter ini bersifat opsional.
		\item \textbf{callback} merupakan fungsi \textit{callback}, dimana fungsi tersebut dapat berupa fungsi \textit{middleware}, kumpulan dari fungsi \textit{middleware} (yang dipisahkan dengan menggunakan koma), fungsi \textit{array of middleware}, atau kombinasi dari seluruh \textit{item} tersebut. Parameter ini bersifat opsional.
	\end{itemize}
	
	\textit{Method} ini akan menghubungkan suatu koneksi pada \textit{host} dan \textit{port} yang sudah ditentukan. 
	
%	Berikut merupakan contoh implementasi dari \textit{method} ini:
%\begin{lstlisting}
%const express = require('express');
%const app = express();
%app.listen(3000);
%\end{lstlisting}
	
\end{itemize}

%\subsection{Request}
%Sebuah objek dari kelas \textit{Request} akan merepresentasikan \textit{HTTP request} dan memiliki properti untuk \textit{request query} seperti \textit{body, HTTP headers} dan \textit{parameters}.
%
%Beberapa \textit{method} yang ada pada kelas \textit{Request} yaitu: 
%\begin{itemize}
%	\item \textbf{req.accepts(types)} \\ Berfungsi untuk memeriksa apakah tipe konten tertentu dapat diterima atau tidak.
%	
%	\item \textbf{req.get(field)} \\ Berfungsi untuk mengembalikan \textit{HTTP request header} tertentu.
%	
%	\item \textbf{req.is(type)} \\ Berfungsi untuk mengembalikan apakah benar atau salah \textit{type} pada parameter sama dengan status \textit{Content-Type} pada \textit{HTTP header}.
%\end{itemize}

\subsection{Response}
Sebuah objek dari kelas \textit{Response} akan merepresentasikan respon \textit{HTTP} yang dikirim oleh \textit{Express} pada saat menerima \textit{HTTP request} \cite{tj:10:expressjsdocs}.

Salah satu properti yang dimiliki oleh kelas ini adalah:

\begin{itemize}
	\item \textbf{res.locals} \\
	Objek yang berisi variabel lokal milik \textit{response} yang berada dalam lingkup suatu \textit{request} tertentu. Objek ini hanya tersedia selama waktu \textit{request} atau \textit{response} tertentu.
\end{itemize}

Beberapa \textit{method} yang terdapat pada kelas \textit{Response} adalah: 

\begin{itemize}
%	\item \textbf{res.send([body])} \\ \textbf{Parameter:} \textit{body} dapat berupa berbagai jenis objek seperti \textit{Buffer}, \textit{String}, dan \textit{Array}.
%	
%	\textit{Method} ini akan mengirimkan respon HTTP kepada \textit{client} sesuai dengan parameter yang diterima. Berikut merupakan contoh implementasi dari \textit{method} ini:
%\begin{lstlisting}
%//parameter objek String
%res.send('Hello World!');
%	
%//parameter objek Array
%res.send([1,2,3]);
%	
%//parameter objek Buffer
%res.send(new Buffer('<p>This is a Buffer</p>'));
%\end{lstlisting}
%	
%	\item \textbf{res.end([data][, encoding])} \\ \textbf{Parameter:}
%	\begin{itemize}
%		\item \textbf{data} dapat berupa objek \textit{String} atau \textit{Buffer} yang akan dikirim saat mengakhiri proses respon.
%		\item \textbf{encoding} merubah suatu tipe data menjadi tipe data yang lain. Contoh beberapa tipe data yang tersedia yaitu \textit{utf8, base64, ascii, } dan \textit{hex}.
%	\end{itemize}
%	
%	\textit{Method} ini berfungsi untuk mengakhiri suatu proses respon. Apabila akan mengakhiri suatu respon tanpa memerlukan suatu data, maka dapat menggunakan \textit{method} ini. Berikut merupakan contoh implementasi \textit{method} ini:
%\begin{lstlisting}
%app.get('/', function(req, res){
%res.end(); //apabila tidak memerlukan data.
%	
%res.end('goodbye!'); // apabila memerlukan suatu data untuk mengakhiri proses.
%});
%\end{lstlisting}
	
	\item \textbf{res.render(view[, locals][, callback])} \\ \textbf{Parameter:}
	\begin{itemize}
		\item \textbf{view} suatu \textit{string} yang menunjukan \textit{path} dari suatu \textit{view file}.
		\item \textbf{locals} suatu objek yang memiliki properti yang menunjukan variabel lokal dari \textit{view}.
		\item \textbf{callback} suatu fungsi \textit{callback}. 
	\end{itemize}
	
	\textit{Method} ini berfungsi untuk merubah \textit{view file} dan mengirim \textit{file} tersebut kepada \textit{client}.
	
%	Berikut merupakan contoh implementasi \textit{method} ini:
%\begin{lstlisting}
%app.get('/', function(req, res){
%res.render('about');  //akan merubah(render) halaman about 
%});
%\end{lstlisting}
	
%	\item \textbf{res.sendStatus(statusCode)} \\ \textbf{Parameter:} 
%	\begin{itemize}
%		\item \textbf{statusCode} kode status \textit{HTTP}.
%	\end{itemize}
%	
%	\textit{Method} ini akan menetapkan kode status \textit{HTTP} di parameter, dan akan mengirimkan bentuk \textit{String} sebagai \textit{body} dari respon. Berikut contoh implementasi \textit{method} ini:
%\begin{lstlisting}
%// akan mengirimkan 'OK' pada response body.
%res.sendStatus(200); 
%
%// akan mengirimkan 'Not Found' pada response body.
%res.sendStatus(404); 
%
%// akan mengirimkan 'Internal Server Error' pada response body.
%res.sendStatus(500); 
%	\end{lstlisting}
	
	\item \textbf{res.status(code)} \\ \textbf{Parameter:}
	\begin{itemize}
		\item \textbf{code} kode status \textit{HTTP}.
	\end{itemize}
	
	\textit{Method} ini akan menetapkan kode status \textit{HTTP} untuk respon. 
	
%	Berikut merupakan contoh implementasi \textit{method} ini:
%\begin{lstlisting}
%res.status(403).end();
%res.status(400).send('Bad Request');
%\end{lstlisting}
	
%	\item \textbf{res.json([body])} \\ \textbf{Parameter:}
%	\begin{itemize}
%		\item \textbf{body} dapat berupa tipe \textit{JSON} apapun, seperti \textit{array, String,} dan \textit{Boolean}.
%	\end{itemize}
%	
%	\textit{Method} ini berfungsi untuk mengirimkan respon \textit{JSON}. 
%	
%	Berikut merupakan contoh implementasi \textit{method} ini:
%\begin{lstlisting}
%res.json({ user: 'tobi', age: '27'});
%\end{lstlisting}
	
\end{itemize}

%\subsection{Router}
%Objek dari kelas \textit{Router} merupakan \textit{instance} dari \textit{middleware} dan \textit{routes}. Setiap aplikasi \textit{Express} memiliki \textit{router} secara \textit{built-in}. 
%
%\textit{Method} yang dimiliki oleh \textit{Router} yaitu sebagai berikut:
%
%\begin{itemize}
%	\item \textbf{router.METHOD(path, [callback, ...] callback)} \\ \textbf{Parameter:} \\ 
%	\begin{itemize}
%		\item \textbf{path} suatu \textit{path} yang akan ditangani oleh \textit{middleware}. Dapat berupa \textit{string}, \textit{path pattern}, atau \textit{array} dari kombinasi \textit{string} dan \textit{path pattern}.
%		
%		\item \textbf{callback} merupakan fungsi \textit{callback}, dimana fungsi tersebut dapat berupa fungsi \textit{middleware}, kumpulan dari fungsi \textit{middleware} (yang dipisahkan dengan menggunakan koma), fungsi \textit{array of middleware}, atau kombinasi dari seluruh \textit{item} tersebut.
%	\end{itemize}
%	
%	\textit{Method} ini menyediakan fungsionalitas \textit{routing} dalam aplikasi \textit{Express}, dimana \textit{METHOD} merupakan salah satu \textit{HTTP methods} seperti \textit{GET, PUT, } dan \textit{POST}, dalam huruf kecil. Dengan begitu, \textit{method} ini dapat berupa \textit{router.get(), router.post(),} dan \textit{router.put()}.
%	
%	Berikut merupakan contoh implementasi dari \textit{method} ini:
%\begin{lstlisting}
%//menggunakan HTTP method GET
%router.get('/', function(req, res){
%res.send('hello world');
%});
%	
%//menggunakan HTTP method POST
%router.post('/buku', function(req, res){
%res.send('mendapatkan buku');
%});
%\end{lstlisting}
	
%\end{itemize}

%------------- END OF EXPRESS.JS--------------- 


%\section{WebSockets}
%\label{sec:WebSockets} 

%\textit{WebSockets} merupakan \textit{Application Programming Interface (API)} yang memiliki kemampuan untuk membuka sesi komunikasi interaktif antara \textit{browser} pengguna dan \textit{server} \cite{websockets}. Dengan \textit{API} ini, pengguna dapat mengirim pesan ke \textit{server} dan menerima respon tanpa harus melakukan \textit{polling} pada \textit{server} terlebih dahulu. Protokol \textit{WebSockets} akan menjadi dasar dalam penggunaan teknologi \textit{Socket.io}.

%Subbab-subbab berikut menjelaskan kelas-kelas yang ada pada \textit{WebSockets}.

%\subsection{WebSocket}
%Kelas ini merupakan inti untuk mengakses fungsi yang ada pada \textit{WebSockets}. Sebuah objek \textit{WebSocket} dapat membuat dan mengelola koneksi \textit{WebSocket} ke server, serta dapat mengirim dan menerima data pada koneksi tersebut. 
%
%%Sebuah objek dari kelas \textit{WebSocket} menyediakan \textit{API} untuk membuat dan mengelola koneksi \textit{WebSocket} ke \textit{server}, dan juga untuk mengirim dan menerima data pada koneksi. \textit{Constructor} pada kelas \textit{WebSocket} menerima satu parameter wajib dan satu parameter pilihan. 
%
%Berikut merupakan konstruktor dari kelas \textit{WebSocket}: 
%\begin{lstlisting}
%WebSocket WebSocket(in DOMString url, in optional DOMString protocols);
%\end{lstlisting}
%
%\begin{itemize}
%	\item \textbf{url}, parameter wajib yang menunjukan \textit{URL} mana yang akan direspon oleh \textit{WebSocket server}.
%	
%	\item \textbf{protocols}, parameter pilihan (tidak harus ada pada parameter) yang dapat berupa satu \textit{string} atau \textit{array of strings}. Parameter \textit{protocols} merepresentasikan nama dari subprotokol yang akan digunakan oleh objek \textit{WebSocket}. Apabila subprotokol tersedia pada parameter, maka \textit{server} akan memeriksa apakah subprotokol tersebut dapat diterima atau tidak. \textit{Server} akan memberikan respon apabila subprotokol dapat diterima, dan akan menghasilkan suatu \textit{error} apabila tidak dapat diterima. Contoh subprotokol yang dapat digunakan yaitu:
%	\begin{itemize}
%		\item \textbf{chat}
%		\item \textbf{superchat}
%	\end{itemize}
%	
%\end{itemize}
%
%Konstruktor dari kelas \textit{WebSocket} dapat menampilkan suatu \textit{exception} seperti berikut:
%
%\begin{lstlisting}
%SECURITY_ERR
%\end{lstlisting}
%
%\textit{Exception} tersebut menandakan bahwa \textit{port} yang akan digunakan untuk melakukan koneksi diblokir.
%
%Atribut yang dimiliki oleh kelas \textit{WebSocket} yaitu:
%
%\begin{itemize}
%	\item \textbf{binaryType} \\ tipe: \textbf{DOMString} \\ Sebuah \textit{string} yang menandakan tipe dari data biner yang dikirimkan oleh koneksi tertentu. Nilai dari atribut ini dapat berupa \textit{"ArrayBuffer"} apabila objek dari \textit{ArrayBuffer} digunakan.
%	
%	\item \textbf{bufferedAmount} \\ tipe: \textbf{unsigned long} \\ Jumlah \textit{bytes} dari data yang belum dikirimkan oleh \textit{method} \textbf{send()}. Nilai dari atribut ini akan kembali menjadi nol apabila seluruh data sudah dikirimkan. Apabila koneksi terputus, nilai atribut ini tidak akan kembali menjadi nol dan akan tetap bertambah apabila terus dilakukan pemanggilan pada \textit{method} \textbf{send()}.
%	
%	\item \textbf{onclose} \\ tipe: \textbf{EventListener} \\ \textit{Event listener} yang dipanggil saat atribut \textit{readyState} dalam koneksi \textit{WebSocket} berubah menjadi \textit{CLOSED}. \textit{Listener} akan menerima objek dari \textit{CloseEvent} dengan nilai \textit{"close"}.
%	
%	\item \textbf{onerror} \\ tipe: \textbf{EventListener} \\ \textit{Event listener} yang dipanggil saat terjadi \textit{error}. \textit{Event} tersebut akan bernilai \textit{"error"}.
%	
%	\item \textbf{onmessage} \\ tipe: \textbf{EventListener} \\ \textit{Event listener} yang dipanggil saat atribut \textit{readyState} dalam koneksi \textit{WebSocket} berubah menjadi \textit{OPEN}. Hal tersebut menandakan bahwa koneksi sudah siap untuk mengirim dan menerima data. \textit{Event} tersebut akan bernilai \textit{"open"}.
%	
%	\item \textbf{protocol} \\ tipe: \textbf{DOMString} \\ \textit{String} yang menandakan sebuah nama dari sub-protokol yang dipilih oleh \textit{server}. Atribut ini akan menjadi salah satu masukan parameter yang dibutuhkan untuk konstruksi kelas \textit{WebSocket}.
%	
%	\item \textbf{readyState} \\ tipe: \textbf{unsigned short} \\ Menunjukan kondisi koneksi saat ini. Atribut ini memiliki beberapa konstanta yang menunjukan kondisi dari koneksi \textit{WebSocket}. Konstanta tersebut sebagai berikut:
%		\begin{itemize}
%			\item \textbf{CONNECTING} \\ nilai: 0 \\ Koneksi belum terbuka.
%			\item \textbf{OPEN} \\ nilai: 1 \\ Koneksi sudah terbuka dan siap untuk melakukan komunikasi.
%			\item \textbf{CLOSING} \\ nilai: 2 \\ Koneksi sedang dalam proses menutup.
%			\item \textbf{CLOSED} \\ nilai: 3 \\ Koneksi sudah tertutup atau tidak dapat dibuka.
%		\end{itemize}
%	
%	\item \textbf{url} \\ tipe: \textbf{DOMString} \\ \textit{URL} yang akan dituju oleh objek \textit{WebSocket}. Atribut ini akan menjadi salah satu masukan parameter untuk konstruksi kelas \textit{WebSocket}.
%\end{itemize}
%
%Kelas \textit{WebSocket} memiliki dua buah \textit{method}, yaitu:
%
%\begin{itemize}
%	\item \textbf{void close(in optional unsigned long code, in optional DOMString reason)} \\ Berfungsi untuk menutup suatu koneksi atau menghentikan proses koneksi. \\ \textbf{Parameter:} 
%		\begin{itemize}
%			\item \textbf{code} nilai numerik yang menunjukan kode status, yang menjelaskan mengapa suatu koneksi ditutup. Apabila parameter ini tidak tersedia, maka akan diasumsikan dengan nilai \textit{default} yaitu 1000 yang berarti transaksi selesai.
%			\item \textbf{reason} \textit{string} yang menjelaskan mengapa suatu koneksi ditutup. 
%		\end{itemize}
%	\textit{Method} ini dapat melemparkan eksepsi seperti berikut:
%		\begin{itemize}
%			\item \textbf{INVALID\_ACCESS\_ERR} parameter \textit{code} yang tidak valid.
%			\item \textbf{SYNTAX\_ERR} parameter \textit{reason} yang melebihi batas yang telah ditentukan.
%		\end{itemize}
%	
%	\item \textbf{void send(in DOMString data)} \\ Berfungsi untuk mengirimkan data ke \textit{server} melalui koneksi \textit{WebSocket}, dan menambah nilai dari \textit{bufferedAmount} sebanyak jumlah \textit{bytes} yang dibutuhkan untuk menampung data. \\ \textbf{Parameter} \\ Tipe data yang dikirimkan pada parameter dapat berbeda-beda, Beberapa tipe tersebut yaitu sebagai berikut:
%	
%		\begin{itemize}
%			\item \textbf{USVString} sebuah teks \textit{string} yang ditambahkan ke \textit{buffer} dalam format \textit{UTF-8}. Nilai dari \textit{bufferedAmount} akan bertambah sesuai dengan jumlah \textit{bytes} yang dibutuhkan untuk menyimpan \textit{UTF-8 string}.
%			
%			\item \textbf{ArrayBuffer} data biner yang disimpan pada \textit{fixed-length buffer}, dimana objek dari \textit{ArrayBuffer} dimanipulasi oleh objek \textit{TypedArray}.
%		\end{itemize}
%	\textit{Method} ini dapat melemparkan eksepsi seperti berikut:
%	\begin{itemize}
%		\item \textbf{INVALID\_STATE\_ERR} koneksi saat ini tidak terbuka.
%		\item \textbf{SYNTAX\_ERR} parameter \textit{data} tidak valid.
%	\end{itemize}
%\end{itemize}

%\subsection{CloseEvent}
%Kelas ini akan menangani koneksi \textit{WebSocket} yang ditutup. Objek \textit{CloseEvent} akan dikirim ke \textit{client} saat koneksi ditutup. Objek tersebut akan dikirimkan ke \textit{listener} yang ditunjukan oleh atribut \textit{onclose} milik objek \textit{WebSocket}.

%Konstruksi kelas ini yaitu:

%\begin{itemize}
%	\item \textbf{new CloseEvent(typeArg, closeEventInit);} \\ \textbf{Parameter:} 
%		\begin{itemize}
%			\item \textbf{typeArg} \\ tipe: \textbf{DOMString} \\ nama dari suatu \textit{event} yang akan dikirimkan.
%			\item \textbf{closeEventInit} bersifat pilihan, dan memiliki beberapa nilai sebagai berikut:
%				\begin{itemize}
%					\item \textit{"wasClean"} \\ tipe: \textbf{boolean} \\ menunjukan apakah koneksi sudah ditutup dengan baik atau belum.
%					\item \textit{"code"} \\ tipe: \textbf{unsigned short} \\ kode status yang menunjukan mengapa koneksi ditutup.
%					\item \textit{"reason"} \\ tipe: \textbf{DOMString} \\ teks yang menunjukan alasan mengapa koneksi ditutup oleh \textit{server}.
%				\end{itemize}
%		\end{itemize}
%\end{itemize}

%Berikut merupakan nilai-nilai dari kode status koneksi ditutup:
%
%\begin{itemize}
%	\item \textbf{0-999} \\ nama: - \\ \textit{Reserved}. Tidak digunakan.
%	
%	\item \textbf{1000} \\ nama: \textbf{Normal Closure} \\ Penutupan normal, yang berarti koneksi sudah menyelesaikan apapun tujuan dari koneksi tersebut.
%	
%	\item \textbf{1001} \\ nama: \textbf{Going Away} \\ \textit{Endpoint} menghilang karena kesalahan server atau \textit{browser} tidak lagi mengakses halaman yang sudah membuka koneksi.
%	
%	\item \textbf{1002} \\ nama: \textbf{Protocol Error} \\ \textit{Endpoint} menghentikan koneksi karena adanya kesalahan protokol.
%	
%	\item \textbf{1003} \\ nama: \textbf{Unsupported Data} \\ Koneksi dihentikan karena \textit{endpoint} menerima data dengan tipe yang tidak bisa diterima (contoh: \textit{text-only endpoint} menerima data biner).
%	
%	\item \textbf{1004} \\ nama: - \\ \textit{Reserved}. Makna dari kode tersebut akan dijelaskan di waktu yang akan datang.
%	
%	\item \textbf{1005} \\ nama: \textbf{No Status Recieved} \\ \textit{Reserved}. Menandakan bahwa tidak ada kode status yang tersedia.
%	
%	\item \textbf{1006} \\ nama: \textbf{Abnormal Closure} \\ \textit{Reserved}. Menandakan bahwa koneksi ditutup secara tidak normal (contoh: tidak ada \textit{close frame} yang dikirimkan).
%	
%	\item \textbf{1007} \\ nama: \textbf{Invalid frame payload data} \\ \textit{Endpoint} menghentikan koneksi karena pesan yang diterima berisi data yang tidak konsisten (contoh: data \textit{non-UTF-8} berada di dalam pesan teks).
%	
%	\item \textbf{1008} \\ nama: \textbf{Policy Violation} \\ \textit{Endpoint} menghentikan koneksi karena menerima pesan yang melanggar kebijakan. Kode status ini dapat digunakan apabila tidak ada kode status lain yang cocok atau digunakan untuk tidak menunjukan kebijakan lebih rinci.
%	
%	\item \textbf{1009} \\ nama: \textbf{Message too big} \\ \textit{Endpoint} menghentikan koneksi karena menerima \textit{frame} data yang terlalu besar.
%	
%	\item \textbf{1010} \\ nama: \textbf{Missing Extension} \\ \textit{Client} menghentikan koneksi karena \textit{server} tidak menangani satu atau beberapa ekstensi yang diminta oleh \textit{client}.
%	
%	\item \textbf{1011} \\ nama: \textbf{Internal Error} \\ \textit{Server} menghentikan koneksi karena mengalami kondisi tertentu yang menyebabkan tidak bisa memenuhi permintaan \textit{client}.
%	
%	\item \textbf{1012} \\ nama: \textbf{Service Restart} \\ \textit{Server} menghentikan koneksi karena harus mengulang kembali koneksi.
%	
%	\item \textbf{1013} \\ nama: \textbf{Try Again Later} \\ \textit{Server} menghentikan koneksi karena ada kondisi yang harus ditangani untuk sementara (contoh: \textit{overloaded}).
%	
%	\item \textbf{1014} \\ nama: \textbf{Bad Gateway} \\ \textit{Server} bertindak sebagai \textit{gateway} atau \textit{proxy} dan menerima respon yang tidak benar dari \textit{upstream server}.
%	
%	\item \textbf{1015} \\ nama: \textbf{TLS Handshake} \\ \textit{Reserved}. Menandakan bahwa koneksi ditutup karena gagal melakukan \textit{TLS handsake} (contoh: sertifikat \textit{server} tidak dapat diverifikasi).
%	
%	\item \textbf{1016-1999} \\ nama: - \\ \textit{Reserved}. Akan digunakan oleh standar \textit{WebSocket} di waktu yang akan datang.
%	
%	\item \textbf{2000-2999} \\ nama: - \\ \textit{Reserved}. Akan digunakan oleh ekstensi \textit{WebSocket}.
%	
%	\item \textbf{3000-3999} \\ nama: - \\ Tersedia untuk digunakan oleh \textit{libraries} dan \textit{frameworks}.
%	
%	\item \textbf{4000-4999} \\ nama: - \\ Tersedia untuk digunakan oleh aplikasi.
%\end{itemize}

%\subsection{MessageEvent}
%Kelas ini merepresentasikan pesan yang diterima oleh suatu objek tujuan. \textit{Constructor} dari kelas ini yaitu: 
%\begin{itemize}
%	\item \textbf{new MessageEvent(type, init);} \\
%	\textbf{Parameter:}
%	\begin{itemize}
%		\item \textbf{type} \\ Tipe pesan \textit{MessageEvent} yang akan dibuat.
%		\item \textbf{init} \\ Parameter ini dapat berupa beberapa nilai seperti berikut:
%		\begin{itemize}
%			\item \textbf{data} \\ Data yang akan diisi pada \textit{MessageEvent}. Dapat bernilai tipe data apapun.
%			\item \textbf{origin} \\ Merepresentasikan \textit{origin} dari suatu pemancar pesan.
%			\item \textbf{ports} \\ Sebuah \textit{array of MessagePort} yang merepresentasikan \textit{port} yang berhubungan dengan saluran pesan yang sedang dikirim.
%		\end{itemize}
%	\end{itemize}
%	Contoh implementasi:
%\begin{lstlisting}
%var myMessage = new MessageEvent('worker', {
%data : 'hello'
%});
%\end{lstlisting}
%\end{itemize}
%
%Beberapa properti yang dimiliki oleh kelas ini yaitu: 
%
%\begin{itemize}
%	\item \textbf{MessageEvent.data} \\ Merepresentasikan data yang dikirim oleh pemancar pesan. Parameter ini dapat berisi data dengan tipe data apapun. 
%	
%	Contoh implementasi:
%\begin{lstlisting}
%myWorker.onmessage = function(e) {
%result.textContent = e.data;
%console.log('Message received from worker');
%};
%\end{lstlisting}
%	
%	
%	\item \textbf{MessageEvent.source} \\ Merepresentasikan pemancar pesan atau sumber suatu pesan berasal. \\
%	Contoh implementasi:
%\begin{lstlisting}
%myWorker.onmessage = function(e) {
%result.textContent = e.data;
%console.log('Message received from worker');
%console.log(e.source);
%};
%	\end{lstlisting}
%\end{itemize}

%-------------REVISED LIMIT OF WEBSOCKETS---------

\section{Socket.io}
\label{sec:Socket.io}

\textit{Socket.io} merupakan salah satu teknologi yang memanfaatkan protokol \textit{WebSockets} \cite{rauch:11:socketio}. Teknologi ini memungkinkan sebuah aplikasi untuk melakukan komunikasi dua arah secara \textit{real-time}. \textit{Socket.io} dapat dijalankan di setiap \textit{platform, browser}, dan gawai.

Sebelum dapat menggunakan \textit{socket.io}, \textit{Node.js} harus sudah ter\textit{install} pada sistem komputer. Apabila hal tersebut sudah dilakukan, maka \textit{socket.io} dapat di\textit{install} dengan menggunakan \textit{command line tools} atau sejenisnya dengan melakukan langkah seperti berikut:
\begin{lstlisting}
npm install socket.io
\end{lstlisting}

Dengan begitu, aplikasi yang dibuat sudah dapat mengakses fitur-fitur yang dimiliki oleh \textit{socket.io}.

\textit{Socket.io} dibagi menjadi dua \textit{API}, yaitu \textit{Server API} dan \textit{Client API}. Subbab-subbab berikut menjelaskan kelas-kelas yang dimiliki \textit{Socket.io}.

\subsection{Server API}
Kelas-kelas yang ada pada \textit{Server API} digunakan untuk menangani proses yang terjadi dalam \textit{server}\cite{rauch:11:socketioserver}. Kelas-kelas tersebut adalah sebagai berikut:

\begin{enumerate}
	\item \textbf{Server} \\ 
	Kelas ini merupakan inti untuk dapat menangani proses yang terjadi dalam \textit{socket.io server}. Kelas ini memiliki konstruktor seperti berikut: 
	\begin{itemize}
		\item \textbf{new Server(httpServer[, options])} \\ 
		\textbf{Parameter:}
		\begin{itemize}
			\item \textbf{httpServer} \\ tipe: \textbf{http.Server} \\ \textit{Server} yang akan dituju.
			\item \textbf{options} \\ tipe: \textbf{Object} \\ Parameter ini dapat berupa berbagai jenis objek. Objek-objek tersebut yaitu sebagai berikut: 
			\begin{itemize}
				\item \textbf{path} \\ tipe: \textbf{String} \\ Nama dari path yang akan ditangkap oleh \textit{server} (contoh: \textit{/socket.io}).
				
				\item \textbf{serveClient} \\ tipe: \textbf{Boolean} \\ Menunjukan apakah \textit{server} akan melayani \textit{file} dari \textit{client} atau tidak.
				
%				\item \textbf{adapter} \\ tipe: \textbf{Adapter} \\ Objek yang akan mengatur beberapa \textit{socket} untuk menerima koneksi, dan mengirimkan pesan antara satu \textit{socket} dengan \textit{socket} lainnya.
%				
%				\item \textbf{origins} \\ tipe: \textbf{String} \\ \textit{Origins} yang diperbolehkan oleh \textit{server}.
				
				%\item \textbf{parser} \\ tipe: \textbf{Parser} \\ \textit{Parser} yang akan digunakan oleh \textit{server}.
			\end{itemize}
		\end{itemize}
	
%		Untuk dapat menggunakan fitur yang ada pada \textit{socket.io}, harus menambahkan modul \textit{socket.io} pada konstanta tertentu. Hal tersebut dapat dilakukan dengan dua cara, yaitu menggunakan kata kunci \textit{new} atau tanpa menggunakan kata kunci \textit{new}:
%		
%		\begin{itemize}
%			\item Menggunakan \textit{new}
%\begin{lstlisting}
%const Server = require('socket.io');
%const io = new Server();
%\end{lstlisting}
%			
%			\item Tanpa menggunakan \textit{new}
%\begin{lstlisting}
%const io = require('socket.io')();
%\end{lstlisting}
%\end{itemize}
%	
%	Contoh implementasi konstruktor:
%	
%\begin{lstlisting}
%const Server = require('socket.io');
%const http = require('http').createServer();
%		
%const io = new Server(http, {
%	path: '/test',
%	serveClient: false
%});
%\end{lstlisting}
	
%	\item \textbf{new Server(port[,options])} \\
%	\textbf{Parameter:}
%	\begin{itemize}
%		\item \textbf{port} \\ tipe: \textbf{Number} \\ Nomor \textit{port} yang akan dituju.
%		\item \textbf{options} \\ tipe: \textbf{Object} \\ Sama seperti konstruktor pertama, parameter ini dapat berupa berbagai jenis objek.
%	\end{itemize}
%	
%	Contoh implementasi konstruktor: 
%\begin{lstlisting}
%const Server = require('socket.io');
%const io = new Server(3000, {
%	path: '/test',
%	serveClient: false
%});
%\end{lstlisting}
	
%	\item \textbf{new Server(options)} \\ 
%	\textbf{Parameter:}
%	
%		\begin{itemize}
%			\item \textbf{options} \\ tipe: \textbf{Object} \\ Sama seperti konstruktor pertama, parameter ini dapat berupa berbagai jenis objek.
%		\end{itemize}
%	
%	Contoh implementasi konstruktor:
%	
%\begin{lstlisting}
%const Server = require('socket.io');
%const io = new Server({
%	path: '/test',
%	serveClient: false
%});
%\end{lstlisting}
	
	\end{itemize}

	Beberapa \textit{method} yang dimiliki oleh kelas ini yaitu sebagai berikut: 
	
		\begin{itemize}
%			\item \textbf{server.serveClient([value])} \\ 
%			\textbf{Parameter:}
%			\begin{itemize}
%				\item \textbf{value} \\ tipe: \textbf{Boolean}
%			\end{itemize}
%			\textbf{Kembalian:} \textit{Server} atau \textit{Boolean}. \\
%			Apabila parameter \textit{value} bernilai \textit{true}, maka \textit{server} akan menangani \textit{file} dari \textit{client}. Apabila tidak ada argumen pada \textit{method} ini, maka kembalian akan berupa status \textit{default} dari \textit{serveClient} saat ini (\textit{true}).
			
%			\item \textbf{server.path([value])} \\
%			\textbf{Parameter:}
%			\begin{itemize}
%				\item \textbf{value} \\ tipe: \textbf{String}
%			\end{itemize}
%			\textbf{Kembalian:} \textit{Server} atau \textit{String} \\
%			Parameter \textit{value} akan menetapkan nilai dari \textit{path} yang akan dituju. Secara \textit{default} nilai dari \textit{path} akan diisi dengan \textit{/socket.io}. Apabila tidak ada argumen pada \textit{method} ini, maka kembalian akan berupa nilai dari \textit{value} saat ini.
%			
%			Berikut contoh implementasi dari \textit{method} ini:
%\begin{lstlisting}
%const io = require('socket.io')();
%io.path('/myownpath');
%\end{lstlisting}
	
%			\item \textbf{server.adapter([value])} \\
%			\textbf{Parameter:}
%			\begin{itemize}
%				\item \textbf{value} \\tipe: \textbf{Adapter} \\ objek \textit{Adapter} yang akan digunakan.
%			\end{itemize}
%			\textit{Method} ini akan menentukan \textit{adapter} apa yang akan digunakan. Secara \textit{default} adapter yang akan digunakan merupakan objek \textit{adapter} yang berasal dari \textit{socket.io} yang bekerja berdasarkan memori. Apabila \textit{method} ini tidak menerima parameter, maka kembalian akan berupa \textit{adapter} saat ini (secara \textit{default}).
%			
%			Berikut contoh implementasi dari \textit{method} ini:
%\begin{lstlisting}
%const io = require('socket.io')(3000);
%const redis = require('socket.io-redis');
%io.adapter(redis({ host: 'localhost', port: 6379}));
%\end{lstlisting}
	
%			\item \textbf{server.origins([value])} \\ 
%			\textbf{Parameter:} 
%			\begin{itemize}
%				\item \textbf{value} \\tipe: \textbf{String} \\ Menunjukan \textit{origin} mana yang diizinkan oleh \textit{server}.
%			\end{itemize}
%			\textbf{Kembalian:} \textit{Server} atau \textit{String}
%			
%			\textit{Method} ini akan menetapkan \textit{origins} mana yang diizinkan oleh \textit{server}. Secara \textit{default}, \textit{origins} yang diizinkan dapat dari mana saja.
%			
%			Berikut contoh implementasi dari \textit{method} ini:
%\begin{lstlisting}
%const io = require('socket.io')();
%io.origins(['foo.example.com:443']);
%\end{lstlisting}
			
			\item \textbf{server.listen(port[, options])} \\
			\textbf{Parameter:} 
			\begin{itemize}
				\item \textbf{port} \\tipe: \textbf{Number} \\ Nomor yang akan digunakan untuk melakukan koneksi kepada \textit{server}
				\item \textbf{options} \\tipe: \textbf{Object} \\ Parameter ini dapat berupa berbagai jenis objek.
			\end{itemize}
			\textit{Method} ini akan melakukan koneksi kepada \textit{server} dengan menggunakan \textit{port} yang terdapat pada parameter.
			
%			\item \textbf{server.of(nsp)} \\
%			Berfungsi untuk menginisialisasi dan menerima \textit{namespace} yang didapat dari tanda pengenal \textit{nsp}. \\
%			\textbf{Parameter:}
%			\begin{itemize}
%				\item \textbf{nsp} \\tipe: \textbf{String} \\ \textit{Namespace}.
%			\end{itemize}
%			Contoh implementasi:
%\begin{lstlisting}
%const adminNamespace = io.of('/admin');
%\end{lstlisting}
			
%			\item \textbf{server.bind(engine)} \\
%			\textbf{Parameter:}
%			\begin{itemize}
%				\item \textbf{engine} \\tipe: \textbf{engine.Server}
%			\end{itemize}
%			\textbf{Kembalian:} \textit{Server}
%			
%			\textit{Method} ini akan menghubungkan \textit{server} dengan objek \textit{Server} dari \textit{engine.io}.
			
%			\item \textbf{server.close([callback])} \\
%			\textit{Method} ini akan menutup koneksi \textit{server socket.io}. Parameter \textit{callback} bersifat opsional dan akan dipanggi saat semua koneksi sudah ditutup. \\
%			\textbf{Parameter:}
%			\begin{itemize}
%				\item \textbf{callback} \\tipe: \textbf{Function} \\ Fungsi \textit{callback}.
%			\end{itemize}
%			Contoh implementasi:
%\begin{lstlisting}
%const Server = require('socket.io');
%const PORT   = 3030;
%const server = require('http').Server();
%	
%const io = Server(PORT);
%	
%// menutup server saat ini
%io.close(); 
%\end{lstlisting}
		
			
		\end{itemize}

	\item \textbf{Namespace} \\ 
	Kelas ini merepresentasikan kumpulan \textit{sockets} yang terhubung dalam lingkup yang diidentifikasi oleh nama \textit{path}. \textit{Client} akan selalu terhubung ke \textit{/} (\textit{namespace} utama), kemudian dapat terhubung ke \textit{namespace} lain saat berada dalam koneksi yang sama.
	
%	Beberapa properti yang dimiliki oleh kelas ini yaitu sebagai berikut:
%	\begin{itemize}
%		\item \textbf{namespace.name} \\tipe: \textbf{string} \\ Nama dari \textit{namespace} tertentu.
%		\item \textbf{namespace.adapter} \\tipe: \textbf{string} \\ \textit{Adapter} yang digunakan untuk \textit{namespace} tertentu.
%	\end{itemize}

	Beberapa \textit{method} yang dimiliki oleh kelas ini yaitu:
	\begin{itemize}
		\item \textbf{namespace.emit(eventName[, ...args])} \\
		Berfungsi untuk memancarkan suatu \textit{event} pada seluruh \textit{clients} yang terhubung. \\
		\textbf{Parameter:}
		\begin{itemize}
			\item \textbf{eventName} \\tipe: \textbf{String} \\ Nama dari \textit{event}.
			\item \textbf{args} \\ Argumen tambahan.
		\end{itemize}
		Contoh implementasi:
\begin{lstlisting}
const io = require('socket.io')();
		
// akan memancarkan event pada namespace utama (/)
io.emit('an event sent to all connected clients'); 
\end{lstlisting}
		
		\item \textbf{namespace.to(room)} \\ 
		Berfungsi untuk memancarkan \textit{event} kepada \textit{client} yang sudah bergabung dalam \textit{room} tertentu. \\ 
		\textbf{Parameter:}
		\begin{itemize}
			\item \textbf{room} \\tipe: \textbf{String} \\ Nama dari \textit{room}.
		\end{itemize}
		\textbf{Kembalian:} \textit{namespace}. \\
		Contoh implementasi:
\begin{lstlisting}
const io = require('socket.io')();
const adminNamespace = io.of('/admin');
	
adminNamespace.to('level1').emit('an event', { some: 'data' });
\end{lstlisting}
	
%		\item \textbf{namespace.clients(callback)} \\
%		Berfungsi untuk mendapatkan daftar \textit{ID clients} yang terhubung pada \textit{namespace} ini.
%		\textbf{Parameter:}
%		\begin{itemize}
%			\item \textbf{callback} \\ fungsi \textit{callback}
%		\end{itemize}
%		Contoh implementasi:
%\begin{lstlisting}
%const io = require('socket.io')();
%io.of('/chat').clients((error, clients) => {
%if (error) throw error;
%	
%// akan menampilkan id seperti [PZDoMHjiu8PYfRiKAAAF, 
%// Anw2LatarvGVVXEIAAAD]
%console.log(clients); 
%});
%\end{lstlisting}

	\end{itemize}
	
	\item \textbf{Socket} \\
	Kelas ini merupakan kelas yang mendasar untuk berinteraksi dengan \textit{browser} milik \textit{clients}. \textit{Socket} merupakan milik \textit{namespace} tertentu dan menggunakan kelas \textit{Client} untuk berkomunikasi. Dalam setiap \textit{namespace}, dapat ditentukan suatu \textit{room} yang dimana sebuah \textit{socket} dapat bergabung atau keluar. Kelas ini pun merupakan turunan dari \textit{EventEmitter} milik \textit{Node.js}. Kelas ini melakukan \textit{override} pada \textit{method emit} milik \textit{EventEmitter}, dan tidak memodifikasi \textit{method} lain.
	
	Beberapa properti yang dimiliki oleh kelas ini yaitu sebagai berikut:
	\begin{itemize}
		\item \textbf{socket.id} \\ Tanda pengenal unik untuk sesi saat ini, yang didapatkan dari kelas \textit{Client}.
		\item \textbf{socket.rooms} \\ Objek yang menandakan \textit{room} dari \textit{client} saat ini.
	\end{itemize}

	Beberapa \textit{method} yang dimiliki oleh kelas ini yaitu sebagai berikut:
	\begin{itemize}
		\item \textbf{socket.join(room[, callback])} \\
		Berfungsi untuk menambah \textit{client} ke \textit{room}. \\
		\textbf{Parameter:}
		\begin{itemize}
			\item \textbf{room} \\tipe: \textbf{String} \\ Nama \textit{room}.
			\item \textbf{callback} \\tipe: \textbf{Function} \\ Fungsi \textit{callback}.
		\end{itemize}
		\textbf{Kembalian:} \textit{Socket}.
		
%		Contoh implementasi:
%	\begin{lstlisting}
%io.on('connection', (socket) => {
%	socket.join('room 237', () => {
%	let rooms = Objects.keys(socket.rooms);
%	console.log(rooms); // [ <socket.id>, 'room 237' ]
%		
%	//akan memancarkan kepada seluruh user 
%	//yang berada di room 237
%	io.to('room 237', 'a new user has joined the room'); 
%	});
%});
%	\end{lstlisting}
		
%		\item \textbf{socket.leave(room[, callback])} \\
%		Berfungsi untuk menghapus \textit{client} dari suatu \textit{room}. \\
%		\textbf{Parameter:}
%		\begin{itemize}
%			\item \textbf{room} \\tipe: \textbf{String} \\ Nama \textit{room}.
%			\item \textbf{callback} \\tipe: \textbf{Function} \\ Fungsi \textit{callback}.
%		\end{itemize}
%		\textbf{Kembalian:} \textit{Socket}. \\
%		Contoh implementasi:
%	\begin{lstlisting}
%socket.leave('room 237');
%	\end{lstlisting}
		
	\end{itemize}
	
%	\item \textbf{Client} \\
%	Kelas ini merepresentasikan koneksi \textit{engine.io} yang masuk. Kelas ini dapat berhubungan dengan banyak \textit{socket} yang dimiliki oleh \textit{namespace} berbeda. \\
%	Properti yang dimiliki oleh kelas ini yaitu sebagai berikut:
%	\begin{itemize}
%		\item \textbf{client.conn} \\ Merepresentasikan koneksi \textit{engine.io} yang masuk.
%		\item \textbf{client.request} \\ Berfungsi untuk mendapatkan permintaan \textit{headers} seperti \textit{Cookie} atau \textit{User-Agent}.
%	\end{itemize}
	
\end{enumerate}

\subsection{Client API}
\textit{Client API} digunakan untuk menangani proses pengaturan koneksi yang terjadi pada bagian \textit{client}\cite{rauch:11:socketioclient}. Agar dapat menggunakan \textit{API} yang tersedia, \textit{client} harus menambahkan \textit{url} pada \textit{syntax script} didalam \textit{html} yang bersangkutan. Hal tersebut dapat dilakukan seperti berikut:

\begin{lstlisting}
	<script src="/socket.io/socket.io.js"></script>
\end{lstlisting}

Kelas-kelas yang ada pada \textit{Client API} yaitu sebagai berikut:
\begin{enumerate}
	\item \textbf{IO} \\
	Untuk dapat menggunakan fungsi yang ada pada \textit{IO}, dapat dilakukan langkah seperti berikut:
\begin{lstlisting}
	
//berfungsi untuk melayani file client
<script src="/socket.io/socket.io.js"></script>
\end{lstlisting}

%	Dengan langkah tersebut, \textit{socket.io} akan dapat menangani \textit{file} yang berasal dari \textit{client}. Selain langkah tersebut, ada satu langkah lagi yang dapat digunakan, yaitu sebagai berikut:
%\begin{lstlisting}
%const io = require('socket.io-client');
%\end{lstlisting}
	
	\textit{Method} yang dimiliki oleh kelas ini yaitu sebagai berikut:
	\begin{itemize}
		\item \textbf{io([url][, options])} \\
		Berfungsi untuk membuat objek baru dari kelas \textit{Manager} dengan \textit{url}, dan akan menggunakan objek kelas \textit{Manager} yang sudah ada untuk pemanggilan selanjutnya, apabila \textit{multiplex} pada parameter \textit{option} bernilai \textit{true}. Objek \textit{Socket} akan dikembalikan untuk \textit{namespace} yang sudah ditentukan oleh nama \textit{path} pada \textit{URL}, dengan nilai \textit{default} /. \\
		\textbf{Parameter:}
		\begin{itemize}
			\item \textbf{url} \\tipe: \textbf{String} \\ Nama \textit{URL}.
			\item \textbf{options} \\tipe: \textbf{Object} \\ Parameter ini dapat berupa beberapa jenis objek, seperti milik kelas \textit{Manager}
		\end{itemize}
		\textbf{Kembalian:} \textit{Socket}
		
	\end{itemize}
	
%	\item \textbf{Manager}
%	Konstruksi kelas ini yaitu sebagai berikut:
%	\begin{itemize}
%		\item \textbf{new Manager(url[, options])} \\
%		\begin{itemize}
%			\item \textbf{url} \\tipe: \textbf{String} \\ Nama \textit{URL}
%			\item \textbf{options} \\tipe: \textbf{Object} \\ Parameter ini dapat berupa berbagai jenis objek seperti berikut:
%			\begin{itemize}
%				\item \textbf{path} \\tipe: \textbf{String} \\ Nama \textit{path} yang dituju pada bagian \textit{server}.
%				\item \textbf{reconnection} \\tipe: \textbf{Boolean} \\ Menandakan apakah akan melakukan koneksi ulang secara otomatis atau tidak.
%				\item \textbf{timeout} \\tipe: \textbf{Number} \\ Menandakan waktu koneksi yang habis sebelum \textit{event connect\_error} dan \textit{connect\_timeout} terjadi.
%			\end{itemize}
%		\end{itemize}
%	\end{itemize}

%	Beberapa \textit{method} yang ada pada kelas ini yaitu sebagai berikut:
%	\begin{itemize}
%		\item \textbf{manager.timeout([value])} \\
%		Berfungsi untuk menentukan nilai \textit{timeout} untuk suatu koneksi.\\
%		\textbf{Parameter:}
%		\begin{itemize}
%			\item \textbf{value} \\tipe: \textbf{Number} \\ Nilai \textit{timeout}.
%		\end{itemize}
%		\textbf{Kembalian:} \textit{Number}
%		
%		\item \textbf{manager.open([callback])} \\
%		Apabila objek \textit{Manager} diinisiasi dengan nilai \textit{false} pada \textit{autoConnect}, maka dapat menggunakan \textit{method} ini untuk membuat percobaan koneksi baru.\\
%		\textbf{Parameter:}
%		\begin{itemize}
%			\item \textbf{callback} \\tipe: \textbf{Function} \\ Fungsi \textit{callback}.
%		\end{itemize}
%	\end{itemize}

%	Beberapa \textit{events} yang ada pada kelas ini yaitu sebagai berikut:
%	\begin{itemize}
%		\item \textbf{connect\_error} \\ Akan dipancarkan apabila ada kesalahan pada koneksi.
%		\item \textbf{connect\_timeout} \\ Akan dipancarkan apabila waktu koneksi telah habis.
%	\end{itemize}
	
	\item \textbf{Socket}
	
	Salah satu properti yang dimiliki oleh kelas ini yaitu:
	
	\begin{itemize}
		\item \textbf{socket.id} \\ \textit{Identifier} unik yang dimiliki oleh satu \textit{socket} untuk satu sesi. Properti ini akan mempunyai nilai segera setelah koneksi terhubung, dan akan diperbarui setelah melakukan koneksi ulang.
	\end{itemize}
	
	Salah satu \textit{method} yang dimiliki oleh kelas ini yaitu:
	
	\begin{itemize}
		
%		\item \textbf{socket.open()} \\ Berfungsi untuk membuka \textit{socket} secara manual.\\
%		\textbf{Kembalian:} \textit{Socket}
%		Contoh implementasi:
%\begin{lstlisting}
%const socket = io({
%autoConnect: false
%});
%	
%// ...
%socket.open();
%\end{lstlisting}
	
	\item \textbf{socket.emit(eventName[, ..args][, ack])} \\ 
	Berfungsi untuk memancarkan \textit{event} kepada \textit{socket} yang ditandai dengan nama dari \textit{event} tersebut.\\ 
	\textbf{Parameter:}
	\begin{itemize}
		\item \textbf{eventName} \\tipe: \textbf{String} \\ Nama \textit{event}.
		\item \textbf{args} \\ Argumen tambahan (opsional).
		\item \textbf{ack} \\tipe: \textbf{Function} \\ Fungsi tambahan (opsional).
	\end{itemize}
	Contoh implementasi:
\begin{lstlisting}
socket.emit('ferret', 'tobi', (data) => {
	console.log(data);
});
\end{lstlisting}
	
%	\item \textbf{socket.close()} \\ Berfungsi untuk menutup \textit{socket} secara manual.
	\end{itemize}

	Beberapa \textit{events} yang ada pada kelas ini yaitu sebagai berikut:
	\begin{itemize}
		\item \textbf{connect} \\ Akan dipancarkan apabila berhasil melakukan koneksi dan setelah melakukan koneksi ulang.
		\item \textbf{disconnect} \\ Akan dipancarkan apabila \textit{server} memutus koneksi atau \textit{client} memutus koneksi.
	\end{itemize}
\end{enumerate} 

%--------------------- END OF SOCKET.IO -----------------

 

\section{Canvas API}
\label{sec:Canvas API}
 
Canvas API merupakan salah satu elemen \textit{HTML5} yang digunakan untuk membuat gambar grafis dalam aplikasi web \cite{canvas}. Teknologi ini memiliki fitur untuk membuat komposisi foto dan membuat animasi. Untuk dapat menggunakan fitur-fitur yang ada pada \textit{Canvas API}, langkah yang harus dilakukan adalah sebagai berikut:
\begin{enumerate}
	\item Menambahkan \textit{tag <canvas>} pada \textit{file HTML}, dan menambahkan \textit{id} yang akan digunakan pada \textit{file JavaScript}.
\begin{lstlisting}
<canvas id="canvas"></canvas>
\end{lstlisting}
	
	\item Membuat variabel untuk mendapatkan konteks \textit{rendering} dan fungsi-fungsi menggambar agar dapat menampilkan sesuatu pada \textit{<canvas>}.
\begin{lstlisting}
// Variabel yang akan menampilkan sesuatu 
// pada <canvas> dengan id='canvas'
var canvas = document.getElementById('canvas'); 
	
// Variabel yang akan mendapatkan fungsi-fungsi menggambar 
var ctx = canvas.getContext('2d');
\end{lstlisting}
\end{enumerate}

Subbab-subbab berikut akan menjelaskan tentang beberapa elemen yang tersedia di \textit{Canvas API}.

%\subsection{HTMLCanvasElement}
%\textit{Interface} ini menyediakan beberapa properti dan \textit{method} untuk memanipulasi tata letak dan tampilan dari elemen \textit{canvas}.

%Beberapa properti yang dimiliki oleh \textit{HTMLCanvasElement} yaitu : 

%\begin{itemize}
%	\item \textbf{HTMLCanvasElement.height} \\ Merupakan bilangan integer positif yang merepresentasikan tinggi dari atribut \textit{HTML} pada elemen \textit{canvas} yang diinterpretasikan dalam piksel \textit{CSS}. Apabila atribut tidak didefinisikan, atau atribut diisi dengan nilai negatif, maka akan digunakan nilai \textit{default} yaitu 150.
%	
%	\item \textbf{HTMLCanvasElement.width} \\ Merupakan bilangan integer positif yang merepresentasikan lebar dari atribut \textit{HTML} pada elemen \textit{canvas} yang diinterpretasikan dalam piksel \textit{CSS}. Apabila atribut tidak didefinisikan, atau atribut diisi dengan nilai negatif, maka akan digunakan nilai \textit{default} yaitu 300.
%\end{itemize}

%Beberapa \textit{method} yang dimiliki oleh \textit{HTLMCanvasElement} yaitu : 

%\begin{itemize}
%	\item \textbf{HTMLCanvasElement.getContext(contextType, contextAttributes)} \\
%	\textit{Method} ini akan mengembalikan konteks menggambar pada \textit{canvas}, atau \textit{null} apabila \textit{identifier} konteks tidak didukung.
%	\textbf{Parameter:} 
%	\begin{itemize}
%		\item \textbf{contextType} \\tipe: \textbf{DOMString} \\ Berisi konteks yang menandakan konteks menggambar pada \textit{canvas}. Parameter ini dapat berupa berbagai jenis nilai seperti berikut:
%		\begin{itemize}
%			\item \textbf{"2d"} \\ Merepresentasikan konteks dua dimensi yang akan menciptakan objek \textit{CanvasRenderingContext2D}.
%			\item \textbf{"webgl"} \\ Merepresentasikan konteks tiga dimensi yang akan menciptakan objek \textit{WebGLRenderingContext}.
%			\item \textbf{"webgl2"} \\ Merepresentasikan konteks tiga dimensi yang akan menciptakan objek \textit{WebGL2RenderingContext}. Konteks ini hanya akan tersedia pada \textit{browser} yang dapat mengimplementasi \textit{WebGL} versi 2.
%			\item \textbf{bitmaprenderer} \\ Akan menciptakan objek \textit{ImageBitmapRenderingContext} yang akan mengganti konten dari \textit{canvas} dengan objek tersebut.
%		\end{itemize}
%	
%		\item \textbf{contextAttributes} \\ Berisi atribut dari parameter \textit{contextType}. Contoh atribut dari beberapa tipe konteks yaitu sebagai berikut:
%		\begin{itemize}
%			\item Atribut konteks \textit{2d}:
%			\begin{itemize}
%				\item \textbf{alpha} \\tipe: \textbf{boolean} \\ Menandakan apakah \textit{canvas} berisi \textit{alpha channel} atau tidak. Apabila bernilai \textit{false}, maka \textit{browser} akan menetapkan \textit{backdrop} untuk selalu bernilai \textit{opaque}, sehingga akan mempercepat proses menggambar.
%			\end{itemize}
%			
%			\item Atribut konteks WebGL:
%			\begin{itemize}
%				\item \textbf{depth} \\ tipe: \textbf{boolean} \\ Menandakan apakah \textit{buffer} untuk menggambar memiliki ukuran setidaknya 16 \textit{bits} atau tidak.
%				\item \textbf{antialias} \\tipe: \textbf{boolean} \\ Menandakan apakah dapat melakukan proses \textit{anti-aliasing} atau tidak.
%			\end{itemize}
%		\end{itemize}
%	\end{itemize}
%	\textbf{Kembalian:} Objek sesuai parameter \textit{contextType}.
%	
%\end{itemize}


%\subsection{CanvasRenderingContext2D}
%\textit{Interface} ini digunakan untuk menggambar persegi panjang, teks, gambar, dan objek-objek lain kedalam elemen \textit{canvas}. \textit{CanvasRenderingContext2D} menyediakan konteks \textit{2D rendering} untuk suatu elemen \textit{<canvas>}.
%Untuk mendapatkan objek dari \textit{interface} ini, harus memanggil \textit{getContext()} didalam elemen \textit{<canvas>}, dengan memberi \textit{''2d''} sebagai argumen. Berikut contoh penggunaannya :
%
%\begin{lstlisting}
%var canvas = document.getElementById('myCanvas');
%var ctx = canvas.getContext('2d');
%\end{lstlisting}

%Properti yang dimiliki oleh \textit{interfaces} ini terbagi menjadi beberapa bagian seperti berikut:
%\begin{enumerate}
%	\item \textbf{\textit{Fill} dan \textit{Stroke styles}}
%	\begin{itemize}
%		\item \textbf{CanvasRenderingContext2D.fillStyle} \\ Menentukan warna atau gaya yang akan digunakan pada bentuk tertentu. Nilai \textit{default} dari properti ini yaitu \#000 (\textit{'black'}). Properti ini dapat berisi:
%		\begin{itemize}
%			\item \textbf{color} \\tipe: \textbf{DOMString} 
%			\item \textbf{gradient} \\tipe: \textbf{CanvasGradient}
%			\item \textbf{pattern} \\tipe: \textbf{CanvasPattern}
%		\end{itemize}
%		Contoh implementasi:
%\begin{lstlisting}
%var canvas = document.getElementById('canvas');
%var ctx = canvas.getContext('2d');
%	
%//mengisi bentuk persegi dengan warna biru hanya pada sisinya
%ctx.strokeStyle = 'blue'; 
%
%//menggambar persegi tanpa ada warna didalam bentuknya
%ctx.strokeRect(10, 10, 100, 100);
%\end{lstlisting}
%		
%		\item \textbf{CanvasRenderingContext2D.strokeStyle} \\ Menentukan warna atau gaya yang akan digunakan pada garis sisi pada bentuk tertentu. Nilai \textit{default} pada properti ini yaitu \#000 (\textit{'black'}). properti ini dapat berupa:
%		\begin{itemize}
%			\item \textbf{color} \\tipe: \textbf{DOMString} 
%			\item \textbf{gradient} \\tipe: \textbf{CanvasGradient}
%			\item \textbf{pattern} \\tipe: \textbf{CanvasPattern}
%		\end{itemize}
%		 Contoh implementasi:
%\begin{lstlisting}
%var canvas = document.getElementById('canvas');
%var ctx = canvas.getContext('2d');
%
%//mengisi bentuk persegi dengan warna biru hanya pada sisinya
%ctx.strokeStyle = 'blue'; 
%
%//menggambar persegi tanpa ada warna didalam bentuknya
%ctx.strokeRect(10, 10, 100, 100);	
%\end{lstlisting}
%	\end{itemize}
%
%	\item \textbf{\textit{Line styles}} \\ 
%	\begin{itemize}
%		\item \textbf{CanvasRenderingContext2D.lineWidth} \\ Menentukan ketebalan dari garis. Nilai \textit{default} dari properti ini yaitu 1.0. Isi dari properti ini dapat berupa:
%		\begin{itemize}
%			\item \textbf{value} \\ Nilai yang menentukan ketebalan garis.
%		\end{itemize}
%		Contoh implementasi:
%\begin{lstlisting}
%var canvas = document.getElementById('canvas');
%var ctx = canvas.getContext('2d');
%	
%ctx.lineWidth = 15;
%\end{lstlisting}
%	
%		\item \textbf{CanvasRenderingContext2D.lineCap} \\ Menentukan jenis dari ujung suatu garis. Nilai dari properti ini dapat berupa:
%		\begin{itemize}
%			\item \textbf{butt} \\ Ujung dari garis memiliki bentuk rata.
%			\item \textbf{round} \\ Ujung dari garis memiliki bentuk bulat.
%			\item \textbf{square} \\ Ujung dari garis memiliki bentuk rata, ditambah kotak dengan ukuran lebar yang sama dan satu per delapan dari ketebalan garis tersebut.
%		\end{itemize}
%		Contoh implementasi:
%\begin{lstlisting}
%var canvas = document.getElementById('canvas');
%var ctx = canvas.getContext('2d');
%	
%ctx.lineCap = 'round';
%\end{lstlisting}
%	
%		\item \textbf{CanvasRenderingContext2D.lineJoin} \\ Menentukan bagaimana bentuk sudut dari kedua garis yang saling terhubung. Nilai dari atribut ini dapat berupa:
%		\begin{itemize}
%			\item \textbf{round} \\ Sudut memiliki bentuk bulat (melengkung).
%			\item \textbf{bevel} \\ Sudut memiliki bentuk rata.
%			\item \textbf{miter} \\ Sudut memiliki bentuk lancip.
%		\end{itemize}
%		Contoh implementasi:
%\begin{lstlisting}
%var canvas = document.getElementById('canvas');
%var ctx = canvas.getContext('2d');
%		
%ctx.lineJoin = 'round';
%\end{lstlisting}
%	\end{itemize}
%
%	\item \textbf{Text styles} 
%	\begin{itemize}
%		\item \textbf{CanvasRenderingContext2D.font} \\ Menentukan jenis teks yang akan digunakan. Nilai \textit{default} dari properti ini yaitu 10px \textit{sans-serif}.
%	\end{itemize}
%	
%\end{enumerate}

%\textit{Method} yang dimiliki oleh \textit{interfaces} ini terbagi menjadi beberapa bagian seperti berikut:
%\begin{enumerate}
%	\item \textbf{Menggambar \textit{rectangles}}
%	\begin{itemize}
%		\item \textbf{CanvasRenderingContext2D.clearRect(x, y, width, height)} \\ 
%		\textit{Method} ini akan menghapus gambar sebelumnya dengan membentuk suatu persegi. \textit{Method} ini akan menggambar koordinat titik awal (\textit{x}, \textit{y}) dengan lebar dan tinggi yang sudah ditentukan oleh \textit{width} dan \textit{height}. 
%		
%		\textbf{Parameter:}
%		\begin{itemize}
%			\item \textbf{x} \\ Koordinat x yang menandakan titik awal persegi.
%			\item \textbf{y} \\ Koordinat y yang menandakan titik awal persegi.
%			\item \textbf{width} \\ Lebar persegi.
%			\item \textbf{height} \\ tinggi persegi.
%		\end{itemize}
%		Contoh implementasi:
%\begin{lstlisting}
%var canvas = document.getElementById('canvas');
%var ctx = canvas.getContext('2d');
%		
%ctx.fillRect(25, 25, 100, 100);
%	
%// Akan menghapus bagian dalam persegi yang 
%// sudah digambar sebelumnya
%ctx.clearRect(45, 45, 60, 60); 
%	\end{lstlisting}
		
%		\item \textbf{CanvasRenderingContext2D.fillRect(x, y, width, height)} \\ 
%		\textit{Method} ini akan menggambar persegi dengan warna tertentu didalam bentuknya. \textit{Method} ini akan menggambar koordinat titik awal (\textit{x}, \textit{y}) dengan lebar dan tinggi yang sudah ditentukan oleh \textit{width} dan \textit{height}. 
%		
%		\textbf{Parameter:}
%		\begin{itemize}
%			\item \textbf{x} \\ Koordinat x yang menandakan titik awal persegi.
%			\item \textbf{y} \\ Koordinat y yang menandakan titik awal persegi.
%			\item \textbf{width} \\ Lebar persegi.
%			\item \textbf{height} \\ tinggi persegi.
%		\end{itemize}
%		Contoh implementasi:
%\begin{lstlisting}
%var canvas = document.getElementById('canvas');
%var ctx = canvas.getContext('2d');
%	
%ctx.fillStyle = 'green';
%ctx.fillRect(10, 10, 100, 100);
%\end{lstlisting}
	
%	\item \textbf{CanvasRenderingContext2D.strokeRect(x, y, width, height)} \\ 
%	\textit{Method} ini akan menggambar persegi tanpa ada warna didalam bentuknya, namun akan memberi warna tertentu pada garis sisi persegi tersebut. \textit{Method} ini akan menggambar koordinat titik awal (\textit{x}, \textit{y}) dengan lebar dan tinggi yang sudah ditentukan oleh \textit{width} dan \textit{height}. 
%	
%	\textbf{Parameter:}
%	\begin{itemize}
%		\item \textbf{x} \\ Koordinat x yang menandakan titik awal persegi.
%		\item \textbf{y} \\ Koordinat y yang menandakan titik awal persegi.
%		\item \textbf{width} \\ Lebar persegi.
%		\item \textbf{height} \\ tinggi persegi.
%	\end{itemize}
%	Contoh implementasi:
%\begin{lstlisting}
%var canvas = document.getElementById('canvas');
%var ctx = canvas.getContext('2d');
%	
%ctx.strokeStyle = 'green';
%ctx.strokeRect(10, 10, 100, 100);
%\end{lstlisting}
%	\end{itemize}

%	\item \textbf{Paths} \\
%	\textit{Paths} merupakan kumpulan beberapa titik yang terhubung oleh garis yang dapat membentuk lengkungan, garis, atau bentuk tertentu. Untuk dapat membuat bentuk tertentu menggunakan \textit{paths}, dapat dilakukan langkah berikut:
%	\begin{enumerate}
%		\item Membuat \textit{path}.
%		\item Menggunakan beberapa fungsi \textit{path} untuk menggambar suatu \textit{path}.
%		\item Menutup \textit{path} yang sudah digambar.
%		\item Setelah \textit{path} ditutup, maka bentuk tersebut dapat diberi warna didalam bentuk tersebut maupun di garis sisinya.
%	\end{enumerate}
	
%	Berikut merupakan beberapa \textit{method} untuk membuat \textit{path}:
%	\begin{itemize}
%		\item \textbf{CanvasRenderingContext2D.beginPath()} \\ 
%		\textit{Method} ini digunakan untuk memulai suatu \textit{path} baru dengan mengosongkan \textit{list} dari \textit{sub-paths} sebelumnya.
%		
%		\item \textbf{CanvasRenderingContext2D.moveTo(x, y)} \\
%		\textit{Method} ini digunakan untuk memindahkan titik awal dari \textit{sub-path} yang baru ke koordinat (\textit{x}, \textit{y}). \\
%		\textbf{Parameter:}
%		\begin{itemize}
%			\item \textbf{x} \\ Koordinat x yang menandakan titik pada posisi sumbu x.
%			\item \textbf{y} \\ Koordinat y yang menandakan titik pada posisi sumbu y.
%		\end{itemize}
%		
%		\item \textbf{CanvasRenderingContext2D.lineTo(x, y)} \\
%		\textit{Method} ini digunakan untuk menghubungkan titik sebelumnya pada \textit{sub-path} dengan koordinat (\textit{x}, \textit{y}).
%		\textbf{Parameter:}
%		\begin{itemize}
%			\item \textbf{x} \\ Koordinat x yang menandakan akhir dari garis.
%			\item \textbf{y} \\ Koordinat y yang menandakan akhir dari garis.
%		\end{itemize}
%		
%		\item \textbf{CanvasRenderingContext2D.closePath()} \\
%		\textit{Method} ini berfungsi untuk memindahkan posisi titik saat ini kembali ke posisi awal \textit{sub-path}. Apabila bentuk yang dibuat oleh \textit{path} sudah ditutup, atau hanya memiliki satu titik, maka \textit{method} ini tidak akan berfungsi.
%		
%		\item \textbf{CanvasRenderingContext2D.arc(x, y, radius, startAngle, endAngle, anticlockwise)} \\
%		\textit{Method} ini akan menambah lengkungan pada \textit{path} dengan titik tengah berada pada posisi (\textit{x}, \textit{y}), memiliki \textit{radius} dan dimulai dari sudut \textit{startAngle} dan berakhir di sudut \textit{endAngle}, dengan cara menggambar sesuai arah \textit{anticlockwise}. \\
%		\textbf{Parameter:}
%		\begin{itemize}
%			\item \textbf{x} \\ Koordinat x dari titik tengah lengkungan.
%			\item \textbf{y} \\ Koordinat y dari titik tengah lengkungan.
%			\item \textbf{radius} \\ Radius lengkungan.
%			\item \textbf{startAngle} \\ Derajat awal dari lengkungan, diukur searah jarum jam dari posisi sumbu x positif dalam radian.
%			\item \textbf{endAngle} \\ Derajat akhir dari lengkungan, diukur searah jarum jam dari posisi sumbu x positif dalam radian.
%			\item \textbf{anticlockwise} \\ Cara menggambar lengkungan. Apabila bernilai \textit{true}, maka akan digambar dengan cara berlawanan arah jarum jam.
%		\end{itemize}
%		
%		Contoh implementasi:
%\begin{lstlisting}
%var canvas = document.getElementById('canvas');
%var ctx = canvas.getContext('2d');
%	
%// langkah berikut akan membuat gambar segitiga
%ctx.beginPath();
%ctx.moveTo(20, 20);
%ctx.lineTo(200, 20);
%ctx.lineTo(120, 120);
%ctx.closePath(); 
%ctx.stroke();
%\end{lstlisting}
%		
%	\end{itemize} 

%\end{enumerate}

\subsection{Animation}
Dengan menggunakan \textit{JavaScript} untuk mengontrol elemen \textbf{<canvas>}, hal tersebut akan sangat membantu dalam membuat animasi interaktif. Subbab ini akan membahas tentang pembuatan animasi didalam \textit{Canvas API}.

\textbf{Langkah Dasar Animasi}

Ada beberapa langkah yang dibutuhkan untuk menggambar suatu \textit{frame}.

\begin{enumerate}
	\item \textit{Clear the canvas} \\
	Gambar atau bentuk yang berada dalam \textit{canvas} pada \textit{state} sebelumnya harus dihapus terlebih dahulu. Cara yang paling mudah untuk melakukan hal tersebut adalah menggunakan \textit{method} \textit{clearRect()}.
	
	\item \textit{Save the canvas state} \\
	Apabila akan dilakukan perubahan pada gaya, transformasi, atau hal lainnya yang mengakibatkan kondisi \textit{canvas} saat ini, maka kondisi \textit{canvas} yang asli harus disimpan terlebih dahulu. Hal tersebut dilakukan agar kondisi \textit{canvas} yang asli digunakan setiap suatu \textit{frame} digambar.
	
	\item \textit{Draw animated shapes} \\
	Langkah ini melakukan proses \textit{rendering} pada \textit{frame} yang telah digambar.
	
	\item \textit{Restore the canvas state} \\
	Apabila kondisi \textit{canvas} telah disimpan sebelumnya, maka kondisi tersebut harus dikembalikan lagi sebelum menggambar \textit{frame} baru.
\end{enumerate}

Untuk dapat melihat suatu gambar atau bentuk yang bergerak didalam \textit{canvas}, diperlukan cara menjalankan fungsi menggambar selama periode waktu tertentu. Berikut merupakan beberapa cara yang dapat dilakukan:

\begin{itemize}
	\item \textbf{setInterval(func, delay[, param1, param2, ...])} \\
	\textbf{Parameter:}
	\begin{itemize}
		\item \textbf{func} \\
		Suatu fungsi yang akan dieksekusi setiap \textit{delay} \textit{milliseconds}.
		
		\item \textbf{delay} \\
		Waktu dalam \textit{milliseconds}. \textit{Timer} akan melakukan \textit{delay} diantara waktu selama menjalankan fungsi tertentu.
		
		\item \textbf{param1,...,paramN} \\
		Parameter tambahan untuk fungsi apabila waktu \textit{timer} telah habis.
	\end{itemize}
	
	\textbf{Kembalian:} \\ 
	\textbf{intervalID} identifikasi unik milik suatu interval.

	\textit{Method} ini akan memanggil suatu fungsi atau mengeksekusi kode tertentu selama rentang waktu yang telah ditentukan. Pemanggilan fungsi atau eksekusi kode dilakukan dengan penundaan selama waktu \textit{delay} tertentu.
	
	\item \textbf{clearInterval(intervalID)} \\
	\textbf{Parameter:}
	\begin{itemize}
		\item \textbf{intervalID} identifikasi milik suatu interval yang akan diberhentikan.
	\end{itemize}

	\textit{Method} ini akan memberhentikan suatu fungsi yang berjalan berulang-ulang selama rentang waktu tertentu, yang dipanggil oleh \textit{method} \textbf{setInterval()}.
	
	
	\item \textbf{requestAnimationFrame(callback)} \\
	\textbf{Parameter:} \\
	\begin{itemize}
		\item \textbf{callback} \\
		Parameter fungsi yang akan dipanggil saat harus memperbarui animasi dan menggambar \textit{frame} selanjutnya.
	\end{itemize}

	\textit{Method} ini akan memberitahu \textit{browser} bahwa akan dilakukan suatu animasi, dan melakukan \textit{request} kepada \textit{browser} untuk memanggil fungsi tertentu untuk memperbarui animasi sebelum melakukan gambar ulang selanjutnya.
\end{itemize} 


\subsection{canvasRenderingContext2D} 

\textit{Interface} ini digunakan untuk menggambar persegi panjang, teks, gambar, dan objek-objek lain kedalam elemen \textit{canvas}. \textit{CanvasRenderingContext2D} menyediakan konteks \textit{2D rendering} untuk suatu elemen \textit{<canvas>}.
Untuk mendapatkan objek dari \textit{interface} ini, harus memanggil \textit{getContext()} didalam elemen \textit{<canvas>}, dengan memberi \textit{''2d''} sebagai argumen. Berikut contoh penggunaannya :

\begin{lstlisting}
var canvas = document.getElementById('myCanvas');
var ctx = canvas.getContext('2d');
\end{lstlisting}

Salah satu properti yang dimiliki oleh \textit{interface} ini adalah:
\begin{itemize}
	\item \textbf{CanvasRenderingContext2D.globalCompositeOperation} \\
	Properti ini akan menetapkan tipe operasi komposisi yang akan digunakan saat menggambar suatu bentuk baru kedalam \textit{canvas}. Tipe tersebut berupa \textit{String} yang akan menentukan komposisi atau \textit{blending mode} yang akan digunakan. 
	
	Berikut contoh tipe yang dapat digunakan:
	\begin{itemize}
		\item \textbf{destination-over} Bentuk yang baru akan digambar pada posisi dibelakang konten \textit{canvas} yang telah ada sebelumnya.
	\end{itemize}
\end{itemize}

Beberapa \textit{method} yang dimiliki oleh \textit{interface} ini adalah sebagai berikut:

\begin{itemize}
	\item \textbf{CanvasRenderingContext2D.clearRect(x, y, width, height)} \\ 
	\textit{Method} ini akan menghapus gambar sebelumnya dengan membentuk suatu persegi. \textit{Method} ini akan menggambar koordinat titik awal (\textit{x}, \textit{y}) dengan lebar dan tinggi yang sudah ditentukan oleh \textit{width} dan \textit{height}. 
	
	\textbf{Parameter:}
	\begin{itemize}
		\item \textbf{x} \\ Koordinat x yang menandakan titik awal persegi.
		\item \textbf{y} \\ Koordinat y yang menandakan titik awal persegi.
		\item \textbf{width} \\ Lebar persegi.
		\item \textbf{height} \\ tinggi persegi.
	\end{itemize}
%	Contoh implementasi:
%	\begin{lstlisting}
%	var canvas = document.getElementById('canvas');
%	var ctx = canvas.getContext('2d');
%	
%	ctx.fillRect(25, 25, 100, 100);
%	
%	// Akan menghapus bagian dalam persegi yang 
%	// sudah digambar sebelumnya
%	ctx.clearRect(45, 45, 60, 60); 
%	\end{lstlisting}
	
	\item \textbf{CanvasRenderingContext2D.drawImage(image, dx, dy)} \\
	\textbf{Parameter:}
	\begin{itemize}
		\item \textbf{image} elemen yang akan digambar kedalam \textit{context} tertentu.
		\item \textbf{dx} koordinat \textit{x} pada \textit{canvas} untuk menempatkan \textit{image} di pojok kiri atas.
		\item \textbf{dy} koordinat \textit{y} pada \textit{canvas} untuk menempatkan \textit{image} di pojok kiri atas.
	\end{itemize}

	\textit{Method} ini akan menyediakan cara untuk menggambar suatu elemen \textit{image} pada \textit{canvas}.
	
	\item \textbf{CanvasRenderingContext2D.save()} \\ 
	\textit{Method} ini akan menyimpan seluruh \textit{state} dari \textit{canvas} dengan menaruh \textit{state} tersebut kedalam suatu \textit{stack} yang sudah diatur didalam elemen \textit{<canvas>}.
	
	\item \textbf{CanvasRenderingContext2D.restore()} \\
	\textit{Method} ini akan mengembalikan \textit{state} yang baru saja disimpan dengan mengeluarkannya dari tumpukan paling atas suatu \textit{stack}. Apabila \textit{stack} tersebut kosong, maka \textit{method} ini tidak melakukan apapun.
\end{itemize}

\section{jQuery}
\label{sec:jQuery}

jQuery merupakan pustaka JavaScript yang menyediakan fitur-fitur untuk mengatur berbagai elemen HTML ~\cite{jQuery}. Pustaka ini memiliki fitur-fitur seperti memanipulasi berkas HTML, menangani suatu \textit{event}, dan mengatur jalannya animasi. jQuery dapat menyediakan fitur-fitur untuk menangani berbagai hal tersebut yang berjalan di berbagai \textit{browser} yang berbeda.

Subbab-subbab berikut akan menjelaskan beberapa \textit{method} yang dimiliki oleh jQuery.

\subsection{.submit(handler)}
\textit{Method} ini akan menyambungkan \textit{handler} suatu \textit{event} dengan \textit{"submit"} yang merupakan \textit{event} dari JavaScript, atau memancarkan \textit{event} tersebut kepada elemen tertentu ~\cite{jQueryAPI}. 

\textbf{Parameter:}
\begin{itemize}
	\item \textbf{handler}, Fungsi yang akan dieksekusi setiap suatu \textit{event} dipancarkan.
\end{itemize}

Berikut merupakan contoh penggunaan dari \textit{method} ini:

\begin{lstlisting}
// HTML
<form id="target">
<input type="text" value="Hello there">
<input type="submit" value="Go">
</form>

// JQUERY
$( "#target" ).submit(function( event ) {
alert( "Handler for .submit() called." );
event.preventDefault();
});
\end{lstlisting}

\subsection{.val()}
\textit{Method} ini akan mendapatkan nilai dari elemen-elemen \textit{form} seperti \textbf{input}, \textit{select}, dan \textit{textarea} ~\cite{jQueryAPI}. \\
\textbf{Kembalian:} \textit{String, Number, Array}.

\subsection{.html()}
\textit{Method} ini akan mendapatkan konten HTML dari elemen pertama yang ada didalam kumpulan elemen-elemen yang sesuai ~\cite{jQueryAPI}. \\
\textbf{Kembalian:} \textit{String}

\subsection{.preventDefault()}
Apabila \textit{method} ini dieksekusi, maka aksi \textit{default} dari suatu \textit{event} tidak akan dieksekusi ~\cite{jQueryAPI}.

\section{Content Template element}
\label{sec:template}

HTML Content Template (<template>) element adalah suatu mekanisme untuk menyimpan konten milik \textit{client} agar konten tersebut tidak dimuat pada saat memuat halaman, dimana konten tersebut dapat dipakai pada saat \textit{runtime} dengan menggunakan JavaScript. 

Berikut merupakan contoh penggunaan dari elemen <template>:

\begin{lstlisting}
<html>
<head>
<title>Home</title>
</head>
<body>
<p>Hello World!</p>

<template id="stage">

</template>

<template id="home_page">
<p>Hello from template element!</p>
</template>

</body>
</html>
\end{lstlisting}

Apabila halaman HTML ini dimuat, maka yang akan ditunjukan dihalaman adalah teks "Hello World!". Seluruh elemen yang ada didalam elemen <template> tidak akan dimuat dihalaman. Agar seluruh isi dari elemen <template> dimuat, maka diperlukan JavaScript. Berikut merupakan contoh kode JavaScript untuk memuat elemen <template>.

\begin{lstlisting}
<script>
// variable yang menyimpan elemen dari <template> stage
var bg = $("#stage");

// variable ini menyimpan seluruh isi dari template home_page
var home = $("#home_page").html();

// variable bg akan memuat template dari home_page
bg.html(home);

</script>
\end{lstlisting}

