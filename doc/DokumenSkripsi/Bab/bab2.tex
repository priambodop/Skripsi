%versi 2 (8-10-2016)
\chapter{Landasan Teori}
\label{chap:teori}

Pada bab ini akan dijelaskan landasan teori mengenai \textit{WebSockets API}, \textit{Socket.io}, \textit{Node.js}, \textit{Express.js}, dan \textit{Canvas API}.

\section{WebSockets}
\label{sec:WebSockets} 

\subsection{Interface}
	\begin{itemize}
		\item \textit{Websocket}
	\end{itemize}

\subsection{WebSocket Events}

\textit{WebSockets API} bekerja berdasarkan \textit{events}(kejadian). Kode-kode yang ada pada aplikasi akan memperhatikan suatu \textit{events} pada objek \textit{WebSocket} untuk mengatasi apabila ada data-data yang masuk dan perubahan pada status koneksi. Aplikasi pada \textit{client} tidak perlu melakukan \textit{poll} terhadap \textit{server} untuk memperbarui data. \textit{Events} dan pesan-pesan lainnya akan diterima secara bersamaan saat server mengirimnya.

Sebuah objek \textit{WebSocket} dapat mengirimkan empat \textit{events} yang berbeda, yaitu : 
	
	\begin{itemize}
		\item Open
		\item Message
		\item Error
		\item Close
	\end{itemize}

\subsection{WebSocket Methods}
Suatu objek \textit{WebSocket} memiliki dua \textit{method}, yaitu \textit{send()} dan \textit{close()}.

\begin{itemize}
	\item send() \\
	Setelah aplikasi melakukan koneksi menggunakan \textit{WebSocket}, maka aplikasi tersebut dapat mengirimkan metode \textit{send()} selama koneksi tetap terbuka. Sebuah aplikasi akan menggunakan \textit{send()} untuk mengirimkan pesan dari \textit{client} menuju \textit{server}. 
	
	
	\item close()
	
	
\end{itemize}

\section{Socket.io}
\label{sec:Socket.io}

\textit{Socket.io} merupakan \textit{JavaScript library} yang digunakan pada aplikasi web untuk melakukan koneksi secara \textit{realtime}. Teknologi ini memiliki dua bagian \textit{library}: bagian \textit{client} yang dijalankan pada \textit{web browser}, dan bagian \textit{server} yang jalankan untuk \textit{Node.js}. \textit{Socket.io} memiliki fitur-fitur yang beragam, seperti melakukan \textit{broadcast} ke beberapa \textit{sockets}, dan menyimpan data yang berhubungan dengan masing-masing \textit{client}.

\subsection{Connection}
Koneksi yang dimulai oleh \textit{Socket.io} dilakukan dengan cara \textit{handshake}. Hal tersebut merupakan bagian yang penting dalam protokol. \textit{Handshake} hanya dilakukan hanya pada saat memulai koneksi, pesan-pesan atau hal lain dalam protokol akan dikirimkan melalui \textit{socket}. 


\subsection{Messages}
Apabila koneksi telah dilakukan, seluruh komunikasi antara \textit{client} dan \textit{server} akan menggunakan pesan(\textit{message}) melalui \textit{socket}. Pesan yang akan dikirimkan harus diubah kedalam format yang sudah dispesifikasi oleh \textit{socket.io}.

Format yang sudah dispesifikasi oleh \textit{socket.io} bertujuan untuk menentukan jenis pesan dan data yang dikirimkan dalam pesan tersebut. Format pesan yang sudah dispesifikasi yaitu seperti berikut : 

[type] : [id] : [endpoint] (: [data])

\begin{itemize}
	\item \textit{type}, merupakan satu digit angka integer yang menunjukan jenis pesan yang akan dikirim.
	\item \textit{id}, merupakan identitas pesan yang terdiri dari beberapa digit angka.
	\item \textit{endpoint}, merupakan \textit{socket} tujuan yang akan menerima pesan yang sedang dikirim. Apabila tidak ada \textit{endpoint}, maka pesan akan dikirimkan ke \textit{default socket}.
	\item \textit{data}, merupakan data yang akan dikirim ke \textit{socket} tertentu. Pada kasus \textit{messages}, data akan dikirimkan dalam bentuk \textit{plain text}, sementara pada kasus \textit{events} akan dikirimkan dalam bentuk \textit{JSON}.
\end{itemize}

\section{Node.js}
\label{sec:Node.js}




\section{Express.js}
\label{sec:Express.js}

\section{Canvas API}
\label{sec:Canvas API}
 
Akan dipaparkan bagaimana menggunakan template ini, termasuk petunjuk singkat membuat referensi, gambar dan tabel.
Juga hal-hal lain yang belum terpikir sampai saat ini. 
 
\dtext{15-16}

\subsection{Tabel}  
Berikut adalah contoh pembuatan tabel. 
Penempatan tabel dan gambar secara umum diatur secara otomatis oleh \LaTeX{}, perhatikan contoh di file bab2.tex untuk melihat bagaimana cara memaksa tabel ditempatkan sesuai keinginan kita.

Perhatikan bawa berbeda dengan penempatan judul gambar gambar, keterangan tabel harus diletakkan di atas tabel!!
Lihat Tabel~\ref{tab:contoh1} berikut ini:

\begin{table}[H] %atau h saja untuk "kira kira di sini"
	\centering 
	\caption{Tabel contoh}
	\label{tab:contoh1}
	\begin{tabular}{cccc}
		\toprule
		& $v_{start}$ & $\mathcal{S}_{1}$ & $v_{end}$\\

		\midrule
		$\tau_{1}$ & 1 & 12& 20\\
		$\tau_{2}$ & 1 &  & 20\\
		$\tau_{3}$ & 1 & 9 & 20\\
		$\tau_{4}$ & 1 &  & 20\\

		\bottomrule
		
	\end{tabular} 
\end{table}
Tabel~\ref{tab:cthwarna1} dan Tabel~\ref{tab:cthwarna2} berikut ini adalah tabel dengan sel yang berwarna dan ada dua tabel yang bersebelahan. 
\begin{table}[H]
	\begin{minipage}[c]{0.49\linewidth}
		\centering
		\caption{Tabel bewarna(1)}
		\label{tab:cthwarna1}
		\begin{tabular}{ccccc}
			\toprule
			 & $v_{start}$ & $\mathcal{S}_{2}$ & $\mathcal{S}_{1}$ & $v_{end}$\\
			
			\midrule
			$\tau_{1}$ & 1 & 5 \cellcolor{green}& 12& 20\\
			$\tau_{2}$ & 1 & 8 \cellcolor{green}& & 20\\
			$\tau_{3}$ & 1 & 2/8/17 \cellcolor{green}& 9 & 20\\
			$\tau_{4}$ & 1 & \cellcolor{red}& & 20\\
			
			\bottomrule

		\end{tabular}
	\end{minipage}
	\begin{minipage}[c]{0.49\linewidth}
		
		\centering 
		\caption{Tabel bewarna(2)}
		\label{tab:cthwarna2}
		\begin{tabular}{ccccc}
			\toprule
			 & $v_{start}$ & $\mathcal{S}_{1}$ & $\mathcal{S}_{2}$ & $v_{end}$\\
			
			\midrule
			$\tau_{1}$ & 1 & 12& 5 \cellcolor{red} &20\\
			$\tau_{2}$ & 1 &  &  8 \cellcolor{green} &20\\
			$\tau_{3}$ & 1 & 9 & 2/8/17 \cellcolor{green} &20\\
			$\tau_{4}$ & 1 &   & \cellcolor{red} &20\\
			
			\bottomrule
		
		\end{tabular}
	\end{minipage}
\end{table}

 
\subsection{Kutipan}
\label{subs:kutipan} 
Berikut contoh kutipan dari berbagai sumber, untuk keterangan lebih lengkap, silahkan membaca file referensi.bib yang disediakan juga di template ini.
Contoh kutipan:
\begin{itemize}
	\item Buku:~\cite{berg:08:compgeom} 
	\item Bab dalam buku:~\cite{kreveld:04:GIS}
	\item Artikel dari Jurnal:~\cite{buchin:13:median}
	\item Artikel dari prosiding seminar/konferensi:~\cite{kreveld:11:median}
	\item Skripsi/Thesis/Disertasi:~\cite{lionov:02:animasi}~\cite{wiratma:10:following}~\cite{wiratma:22:later}
	\item Technical/Scientific Report:~\cite{kreveld:07:watertight}
	\item RFC (Request For Comments):~\cite{RFC1654}
	\item Technical Documentation/Technical Manual:~\cite{Z.500}~\cite{unicode:16:stdv9}~\cite{google:16:and7}
	\item Paten:~\cite{webb:12:comm}
	\item Tidak dipublikasikan:~\cite{wiratma:09:median}~\cite{lionov:11:cpoly}
	\item Laman web:~\cite{erickson:03:cgmodel}  
	\item Lain-lain:~\cite{agung:12:tango}
\end{itemize}    
  
\subsection{Gambar}

Pada hampir semua editor, penempatan gambar di dalam dokumen \LaTeX{} tidak dapat dilakukan melalui proses {\it drag and drop}.
Perhatikan contoh pada file bab2.tex untuk melihat bagaimana cara menempatkan gambar.
Beberapa hal yang harus diperhatikan pada saat menempatkan gambar:
\begin{itemize}
	\item Setiap gambar {\bf harus} diacu di dalam teks (gunakan {\it field} {\sc label})
	\item {\it Field} {\sc caption} digunakan untuk teks pengantar pada gambar. Terdapat dua bagian yaitu yang ada di antara tanda $[$ dan $]$ dan yang ada di antara tanda $\{$ dan $\}$. Yang pertama akan muncul di Daftar Gambar, sedangkan yang kedua akan muncul di teks pengantar gambar. Untuk skripsi ini, samakan isi keduanya.
	\item Jenis file yang dapat digunakan sebagai gambar cukup banyak, tetapi yang paling populer adalah tipe {\sc png} (lihat Gambar~\ref{fig:ularpng}), tipe {\sc jpg} (Gambar~\ref{fig:ularjpg}) dan tipe {\sc pdf} (Gambar~\ref{fig:ularpdf})
	\item Besarnya gambar dapat diatur dengan {\it field} {\sc scale}.
	\item Penempatan gambar diatur menggunakan {\it placement specifier} (di antara tanda  $[$ dan $]$ setelah deklarasi gambar.
	Yang umum digunakan adalah {\bf H} untuk menempatkan gambar {\bf sesuai} penempatannya di file .tex atau  {\bf h} yang berarti "kira-kira" di sini. \\
	Jika tidak menggunakan {\it placement specifier}, \LaTeX{} akan menempatkan gambar secara otomatis untuk menghindari bagian kosong pada dokumen anda.
	Walaupun cara ini sangat mudah, hindarkan terjadinya penempatan dua gambar secara berurutan. 	
	\begin{itemize}
		\item Gambar~\ref{fig:ularpng} ditempatkan di bagian atas halaman, walaupun penempatannya dilakukan setelah penulisan 3 paragraf setelah penjelasan ini.
		\item Gambar~\ref{fig:ularjpg} dengan skala 0.5 ditempatkan di antara dua buah paragraf. Perhatikan penulisannya di dalam file bab2.tex!
		\item Gambar~\ref{fig:ularpdf} ditempatkan menggunakan {\it specifier} {\bf h}.
	\end{itemize}
\end{itemize}
 
\dtext{17-18}
\begin{figure} 
	\centering  
	\includegraphics[scale=1]{ular-png}  
	\caption[Gambar {\it Serpentes} dalam format png]{Gambar {\it Serpentes} dalam format png} 
	\label{fig:ularpng} 
\end{figure} 

\dtext{19-20}
\begin{figure}[H]
	\centering  
	\includegraphics[scale=0.5]{ular-jpg}  
	\caption[Ular kecil]{Ular kecil} 
	\label{fig:ularjpg} 
\end{figure} 
\dtext{21-22}

\begin{figure}[ht] 
	\centering  
	\includegraphics[scale=1]{ular-pdf}  
	\caption[ {\it Serpentes} betina]{ {\it Serpentes} jantan} 
	\label{fig:ularpdf} 
\end{figure} 
 
