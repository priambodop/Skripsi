\chapter{Implementasi dan Pengujian}
\label{chap:implementasi}

Bab ini terdiri dari dua bagian, yaitu Implementasi Perangkat Lunak dan Pengujian Perangkat Lunak. Bagian implementasi berisi penjelasan lingkungan implementasi dan hasil implementasi. Sedangkan bagian pengujian berisi hasil pengujian fungsional dan eksperimental terhadap perangkat lunak yang telah dibangun.

\section{Implementasi}

\subsection{Lingkungan Implementasi}

Lingkungan implementasi terbagi menjadi dua bagian, yaitu lingkungan perangkat keras dan lingkungan perangkat lunak. Kedua bagian tersebut akan dijelaskan sebagai berikut:
\begin{enumerate}
	\item \textbf{Lingkungan Perangkat Keras} \\
	\begin{enumerate}
		\item \textbf{Komputer}: Macbook Pro (Retina, 13 inch, Mid 2014)
		\item \textbf{Prosesor}: 2.6 GHz Intel Core i5
		\item \textbf{Memori}: 8 GB 1600 MHz DDR3
		\item \textbf{Grafis}: Intel Iris 1536 MB
	\end{enumerate}

	\item \textbf{Lingkungan Perangkat Lunak} \\
	Lingkungan perangkat lunak terbagi menjadi dua bagian yaitu, lingkungan pada \textit{server} dan \textit{client}. Kedua bagian tersebut akan dijelaskan sebagai berikut:
	\begin{enumerate}
		\item \textbf{Server} \\
		\begin{enumerate}
			\item \textbf{Heroku} \\
			Heroku merupakan \textit{cloud platform} yang menyediakan fitur yang dapat membantu pengguna untuk dapat memiliki alamat domain. Spesifikasi Heroku yang digunakan oleh Finger For Life akan dijelaskan sebagai berikut:
			
			\begin{tabular}{ |p{1.5cm}|p{1.5cm}|p{2cm}|p{1.5cm}|p{1.5cm}|p{1.5cm}|p{1.5cm}|}
			\hline
			Dyno Type & Sleeps & Professional Features & Memory (RAM) & CPU Share & Dedicated & Compute \\ \hline
			free & yes & no & 512 MB & 1x & no & 1x-4x \\ \hline
				
			\end{tabular}
		
			\item \textbf{Node.js}
			 
		\end{enumerate}
		
		\item \textbf{Client}
		\begin{enumerate}
			\item \textbf{Express.js}
			
			\item \textbf{Canvas API}
			
			\item \textbf{jQuery}
			
			\item \textbf{The Content Template element}
		\end{enumerate}
	\end{enumerate}
	
\end{enumerate}

\subsection{Hasil Implementasi}


\section{Pengujian}

\subsection{Pengujian Fungsional}

\subsection{Pengujian Eksperimental}