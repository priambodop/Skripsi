%versi 2 (8-10-2016)
\chapter{Kode Program \textit{Server}}
\label{lamp:B}

%terdapat 2 cara untuk memasukkan kode program
% 1. menggunakan perintah \lstinputlisting (kode program ditempatkan di folder yang sama dengan file ini)
% 2. menggunakan environment lstlisting (kode program dituliskan di dalam file ini)
% Perhatikan contoh yang diberikan!!
%
% untuk keduanya, ada parameter yang harus diisi:
% - language: bahasa dari kode program (pilihan: Java, C, C++, PHP, Matlab, C#, HTML, R, Python, SQL, dll)
% - caption: nama file dari kode program yang akan ditampilkan di dokumen akhir
%
% Perhatian: Abaikan warning tentang textasteriskcentered!!
%

Kode Program \textit{Server}
\begin{lstlisting}
#!/usr/bin/env node

/**
* Module dependencies.
*/

var app = require('../app');
var http = require('http');
var socketIO = require('socket.io');
var {Users} = require('./utils/users');

/**
* Get port from environment and store in Express.
*/
var port = normalizePort(process.env.PORT || '3000');
app.set('port', port);

/**
* Create HTTP server.
*/
var server = http.createServer(app);

/**
* Create Socket.io Server
*/
var io = socketIO(server);

/**
* Create Users Object
*/
var users = new Users();

/**
* Listen to socket.io connection
*/
io.on('connection', function(socket){

socket.on('hostJoinRoom', (msg) => {
	socket.join(msg.room);
	users.removeUser(msg.id);
	users.addUser(msg.id, msg.room);
}),

socket.on('sendingTheWinner', (winner) => {
	var user = users.getUser(socket.id);
	io.to(user.room).emit('getTheWinner', winner);

	io.to(user.room).emit('toWinnerPage', winner);

}),

socket.on('goBackHome', (msg) => {
	var user = users.getUser(socket.id);
	users.removeRoom(user.room);
	io.to(user.room).emit('backHome', msg);
}),

socket.on('ping', (message) => {
	socket.emit('pong', message);
}),

//When a room is already full
//redirecting to the character page
socket.on('roomFull', (msg) => {
	io.to(msg.room).emit('toCharPage');
}),

//When a user is selecting a character
//And the character which is clicked is shown.
socket.on('selectingChar', (msg) => {
	var user = users.getUser(socket.id);
	io.to(user.room).emit('charSelecting', msg);
}),

socket.on('sendingChar', (msg) => {
	var user = users.getUser(socket.id);
	io.to(user.room).emit('charSent', msg);
}),

socket.on('charIsReady', (msg) => {
	var user = users.getUser(socket.id);
	io.to(user.room).emit('startTheGame', msg);
}),

socket.on('stepClicked', (message) => {
	var user = users.getUser(socket.id);
	io.to(user.room).emit('moveThePlayer', socket.id);
}),

//debug purpose
socket.on('charMobile', (msg) => {
	socket.emit('CharMobileAccepted', msg);
}),

// When a user requesting to join the game
socket.on('requestToJoin', function(msg){
	var check = users.isRoomExist(msg.room);
	var roomList = users.getUserList(msg.room);

	if (check === true) {
		if (roomList.length === 3) {
		socket.emit('joinRejected', 'The room is already full');
		}
	else{
		socket.join(msg.room);
		users.removeUser(msg.id);
		users.addUser(msg.id, msg.room);

	io.to(user.room).emit('requestAccepted', msg);
	socket.emit('joinSucceed', 'Welcome to the game !');
	}

	}else{
	socket.emit('joinRejected', 'The room is not exist');
	}
	});
});

/**
* Listen on provided port, on all network interfaces.
*/

server.listen(port, (req, res) => {
	console.log("Listening to " + port);
});


/**
* Normalize a port into a number, string, or false.
*/

function normalizePort(val) {
	var port = parseInt(val, 10);

	if (isNaN(port)) {
	// named pipe
	return val;
	}

	if (port >= 0) {
	// port number
	return port;
	}

	return false;
}

\end{lstlisting}