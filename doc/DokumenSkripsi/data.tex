%_____________________________________________________________________________
%=============================================================================
% data.tex v10 (22-01-2017) dibuat oleh Lionov - T. Informatika FTIS UNPAR
%
% Perubahan pada versi 10 (22-01-2017)
%	- Penambahan overfullrule untuk memeriksa warning
%  	- perubahan mode buku menjadi 4: bimbingan, sidang(1), sidang akhir dan 
%     buku final
%	- perbaikan perintah pada beberapa bagian
%  	- perubahan pengisian tulisan "daftar isi" yang error
%  	- penghilangan lipsum dari file ini
%_____________________________________________________________________________
%=============================================================================

%=============================================================================
% 								PETUNJUK
%=============================================================================
% Ini adalah file data (data.tex)
% Masukkan ke dalam file ini, data-data yang diperlukan oleh template ini
% Cara memasukkan data dijelaskan di setiap bagian
% Data yang WAJIB dan HARUS diisi dengan baik dan benar adalah SELURUHNYA !!
% Hilangkan tanda << dan >> jika anda menemukannya
%=============================================================================

%_____________________________________________________________________________
%=============================================================================
% 								BAGIAN 0
%=============================================================================
% Entri untuk memperbaiki posisi "DAFTAR ISI" jika tidak berada di bagian 
% tengah halaman. Sayangnya setiap sistem menghasilkan posisi yang berbeda.
% Isilah dengan 0 atau 1 (e.g. \daftarIsiError{1}). 
% Pemilihan 0 atau 1 silahkan disesuaikan dengan hasil PDF yang dihasilkan.
%=============================================================================
%\daftarIsiError{0}   
\daftarIsiError{1}   
%=============================================================================

%_____________________________________________________________________________
%=============================================================================
% 								BAGIAN I
%=============================================================================
% Tambahkan package2 lain yang anda butuhkan di sini
%=============================================================================
\usepackage{booktabs} 
\usepackage{longtable}
\usepackage{amssymb}
\usepackage{todo}
\usepackage{verbatim} 		%multiline comment
\usepackage{pgfplots}
\usepackage{subcaption} %package tambahan untuk bab 5
%\overfullrule=3mm % memperlihatkan overfull 
%=============================================================================

%_____________________________________________________________________________
%=============================================================================
% 								BAGIAN II
%=============================================================================
% Mode dokumen: menetukan halaman depan dari dokumen, apakah harus mengandung 
% prakata/pernyataan/abstrak dll (termasuk daftar gambar/tabel/isi) ?
% - final 		: hanya untuk buku skripsi, dicetak lengkap: judul ina/eng, 
%   			  pengesahan, pernyataan, abstrak ina/eng, untuk, kata 
%				  pengantar, daftar isi (daftar tabel dan gambar tetap 
%				  opsional dan dapat diatur), seluruh bab dan lampiran.
%				  Otomatis tidak ada nomor baris dan singlespacing
% - sidangakhir	: buku sidang akhir = buku final - (pengesahan + pernyataan +
%   			  untuk + kata pengantar)
%				  Otomatis ada nomor baris dan onehalfspacing 
% - sidang 		: untuk sidang 1, buku sidang = buku sidang akhir - (judul 
%				  eng + abstrak ina/eng)
%				  Otomatis ada nomor baris dan onehalfspacing
% - bimbingan	: untuk keperluan bimbingan, hanya terdapat bab dan lampiran
%   			  saja, bab dan lampiran yang hendak dicetak dapat ditentukan 
%				  sendiri (nomor baris dan spacing dapat diatur sendiri)
% Mode default adalah 'template' yang menghasilkan isian berwarna merah, 
% aktifkan salah satu mode di bawah ini :
%=============================================================================
%\mode{bimbingan} 		% untuk keperluan bimbingan
%\mode{sidang} 			% untuk sidang 1
%\mode{sidangakhir} 	% untuk sidang 2 / sidang pada Skripsi 2(IF)
\mode{final} 			% untuk mencetak buku skripsi 
%=============================================================================

%_____________________________________________________________________________
%=============================================================================
% 								BAGIAN III
%=============================================================================
% Line numbering: penomoran setiap baris, nomor baris otomatis di-reset ke 1
% setiap berganti halaman.
% Sudah dikonfigurasi otomatis untuk mode final (tidak ada), mode sidang (ada)
% dan mode sidangakhir (ada).
% Untuk mode bimbingan, defaultnya ada (\linenumber{yes}), jika ingin 
% dihilangkan, isi dengan "no" (i.e.: \linenumber{no})
% Catatan:
% - jika nomor baris tidak kembali ke 1 di halaman berikutnya, compile kembali
%   dokumen latex anda
% - bagian ini hanya bisa diatur di mode bimbingan
%=============================================================================
%\linenumber{no} 
\linenumber{yes}
%=============================================================================

%_____________________________________________________________________________
%=============================================================================
% 								BAGIAN IV
%=============================================================================
% Linespacing: jarak antara baris 
% - single	: otomatis jika ingin mencetak buku skripsi, opsi yang 
%			     disediakan untuk bimbingan, jika pembimbing tidak keberatan 
%			     (untuk menghemat kertas)
% - onehalf	: otomatis jika ingin mencetak dokumen untuk sidang
% - double 	: jarak yang lebih lebar lagi, jika pembimbing berniat memberi 
%             catatan yg banyak di antara baris (dianjurkan untuk bimbingan)
% Catatan: bagian ini hanya bisa diatur di mode bimbingan
%=============================================================================
\linespacing{single}
%\linespacing{onehalf}
%\linespacing{double}
%=============================================================================

%_____________________________________________________________________________
%=============================================================================
% 								BAGIAN V
%=============================================================================
% Tidak semua skripsi memuat gambar dan/atau tabel. Untuk skripsi yang tidak 
% memiliki gambar dan/atau tabel, maka tidak diperlukan Daftar Gambar dan/atau 
% Daftar Tabel. Sayangnya hal tsb sulit dilakukan secara manual karena 
% membutuhkan kedisiplinan pengguna template.  
% Jika tidak ingin menampilkan Daftar Gambar dan/atau Daftar Tabel, karena 
% tidak ada gambar atau tabel atau karena dokumen dicetak hanya untuk 
% bimbingan, isi dengan "no" (e.g. \gambar{no})
%=============================================================================
\gambar{yes}
%\gambar{no}
\tabel{yes}
%\tabel{no}  
%=============================================================================

%_____________________________________________________________________________
%=============================================================================
% 								BAGIAN VI
%=============================================================================
% Pada mode final, sidang da sidangkahir, seluruh bab yang ada di folder "Bab"
% dengan nama file bab1.tex, bab2.tex s.d. bab9.tex akan dicetak terurut, 
% apapun isi dari perintah \bab.
% Pada mode bimbingan, jika ingin:
% - mencetak seluruh bab, isi dengan 'all' (i.e. \bab{all})
% - mencetak beberapa bab saja, isi dengan angka, pisahkan dengan ',' 
%   dan bab akan dicetak terurut sesuai urutan penulisan (e.g. \bab{1,3,2}). 
% Catatan: Jika ingin menambahkan bab ke-3 s.d. ke-9, tambahkan file 
% bab3.tex, bab4.tex, dst di folder "Bab". Untuk bab ke-10 dan 
% seterusnya, harus dilakukan secara manual dengan mengubah file skripsi.tex
% Catatan: bagian ini hanya bisa diatur di mode bimbingan
%=============================================================================
\bab{all}
%=============================================================================

%_____________________________________________________________________________
%=============================================================================
% 								BAGIAN VII
%=============================================================================
% Pada mode final, sidang dan sidangkhir, seluruh lampiran yang ada di folder 
% "Lampiran" dengan nama file lampA.tex, lampB.tex s.d. lampJ.tex akan dicetak 
% terurut, apapun isi dari perintah \lampiran.
% Pada mode bimbingan, jika ingin:
% - mencetak seluruh lampiran, isi dengan 'all' (i.e. \lampiran{all})
% - mencetak beberapa lampiran saja, isi dengan huruf, pisahkan dengan ',' 
%   dan lampiran akan dicetak terurut sesuai urutan (e.g. \lampiran{A,E,C}). 
% - tidak mencetak lampiran apapun, isi dengan "none" (i.e. \lampiran{none})
% Catatan: Jika ingin menambahkan lampiran ke-C s.d. ke-I, tambahkan file 
% lampC.tex, lampD.tex, dst di folder Lampiran. Untuk lampiran ke-J dan 
% seterusnya, harus dilakukan secara manual dengan mengubah file skripsi.tex
% Catatan: bagian ini hanya bisa diatur di mode bimbingan
%=============================================================================
\lampiran{all}
%=============================================================================

%_____________________________________________________________________________
%=============================================================================
% 								BAGIAN VIII
%=============================================================================
% Data diri dan skripsi/tugas akhir
% - namanpm		: Nama dan NPM anda, penggunaan huruf besar untuk nama harus 
%				  benar dan gunakan 10 digit npm UNPAR, PASTIKAN BAHWA 
%				  BENAR !!! (e.g. \namanpm{Jane Doe}{1992710001}
% - judul 		: Dalam bahasa Indonesia, perhatikan penggunaan huruf besar, 
%				  judul tidak menggunakan huruf besar seluruhnya !!! 
% - tanggal 	: isi dengan {tangga}{bulan}{tahun} dalam angka numerik, 
%				  jangan menuliskan kata (e.g. AGUSTUS) dalam isian bulan.
%			  	  Tanggal ini adalah tanggal dimana anda akan melaksanakan 
%				  sidang ujian akhir skripsi/tugas akhir
% - pembimbing	: pembimbing anda, lihat daftar dosen di file dosen.tex
%				  jika pembimbing hanya 1, kosongkan parameter kedua 
%				  (e.g. \pembimbing{\JND}{} ), \JND adalah kode dosen
% - penguji 	: para penguji anda, lihat daftar dosen di file dosen.tex
%				  (e.g. \penguji{\JHD}{\JCD} )
% !!Lihat singkatan pembimbing dan penguji anda di file dosen.tex!!
% Petunjuk: hilangkan tanda << & >>, dan isi sesuai dengan data anda
%=============================================================================
\namanpm{Priambodo Pangestu}{2013730055}
\tanggal{<<tanggal>>}{<<bulan>>}{2018}
\pembimbing{\PAN}{}    
\penguji{<<penguji 1>>}{<<penguji 2>>} 
%=============================================================================

%_____________________________________________________________________________
%=============================================================================
% 								BAGIAN IX
%=============================================================================
% Judul dan title : judul bhs indonesia dan inggris
% - judulINA: judul dalam bahasa indonesia
% - judulENG: title in english
% Petunjuk: 
% - hilangkan tanda << & >>, dan isi sesuai dengan data anda
% - langsung mulai setelah '{' awal, jangan mulai menulis di baris bawahnya
% - gunakan \texorpdfstring{\\}{} untuk pindah ke baris baru
% - judul TIDAK ditulis dengan menggunakan huruf besar seluruhnya !!
%=============================================================================
\judulINA{Pemanfaatan Smartphone Sebagai Pengendali Permainan Berbasis Web }
\judulENG{Utilization of Smartphone as Web-Based Game Controllers}
%_____________________________________________________________________________
%=============================================================================
% 								BAGIAN X
%=============================================================================
% Abstrak dan abstract : abstrak bhs indonesia dan inggris
% - abstrakINA: abstrak bahasa indonesia
% - abstrakENG: abstract in english 
% Petunjuk: 
% - hilangkan tanda << & >>, dan isi sesuai dengan data anda
% - langsung mulai setelah '{' awal, jangan mulai menulis di baris bawahnya
%=============================================================================
\abstrakINA{Socket.io merupakan sebuah pustaka yang menyediakan fitur untuk melakukan komunikasi secara \textit{real-time} dan dua arah antara \textit{client} dan \textit{server}. Dengan menggunakan Socket.io, \textit{client} dapat mengirimkan pesan kepada \textit{server} dan menerima respon tanpa harus melakukan \textit{polling}, yang berarti proses pengecekan secara berulang terhadap \textit{server} untuk mengetahui apakah \textit{server} masih tersambung atau tidak. Fitur-fitur yang dimiliki oleh Socket.io dapat dimanfaatkan untuk mengembangkan aplikasi web yang membutuhkan komunikasi \textit{real-time}. Salah satu pemanfaatan pustaka Socket.io adalah permainan berbasis web.

Permainan berbasis web yang akan dibangun dinamakan Finger For Life. Permainan ini memanfaatkan teknologi \textit{smartphone} dan \textit{PC}, dimana \textit{smartphone} akan berperan sebagai pengendali didalam permainan, dan \textit{PC} akan berperan sebagai \textit{console} yang akan menyediakan permainan. Untuk memainkan permainan, \textit{smartphone} harus terkoneksi ke \textit{PC} melalui \textit{browser}. Oleh karena itu, fitur yang dimiliki oleh Socket.io digunakan untuk melakukan koneksi antara \textit{smartphone} dan \textit{PC}.

Dengan penggunaan Socket.io didalam pengembangan aplikasi Finger For Life, diharapkan ukuran \textit{latency} yang dihasilkan pada saat memainkan permainan akan sangat kecil. \textit{Latency} merupakan jarak waktu yang dihasilkan pada saat suatu konten atau data dikirimkan dari \textit{client} menuju \textit{server}, maupun sebaliknya. Ukuran \textit{latency} yang dihasilkan akan sangat berpengaruh pada saat tombol yang ada di \textit{smartphone} ditekan, dengan respon yang diberikan oleh \textit{PC} berdasarkan aksi tersebut. Semakin kecil jumlah \textit{latency} yang dihasilkan maka akan semakin cepat respon yang diberikan.

%Pengujian dari aplikasi permainan berbasis web Finger For Life dilakukan oleh beberapa responden dengan \textit{smartphone} dan \textit{browser} yang beragam. Berdasarkan hasil pengujian, aplikasi dapat berjalan dengan baik dan memberikan respon yang tepat pada seluruh responden. Hasil pengujian aplikasi Finger For Life membuktikan bahwa penggunaan Socket.io menghasilkan \textit{latency} yang rendah pada saat permainan dimainkan.
}
\abstrakENG{Socket.io is a library that enables real-time, bidirectional communication between the client and the server. With this API, client can send messages to the server and receive responses without having to poll the server, which means check continuously to the server to see whether the server still connected or not. With these features, Socket.io can be used to build a real-time communication web application such as games.

The web-based games which utilize the features of Socket.io is Finger For Life. This game use smartphone as the controller and PC as the console. To be able to play, users need to connect the smartphone to the PC through web browser. Therefore, Socket.io is used to connect a smartphone to the PC.

The use of Socket.io in the developing Finger For Life web application is expected to decrease the sum of latency of playing the web-based game. Latency is a time interval when the content or the data is being sent from client to the server and back. The sum of latency can affect how fast the response from the PC when the users click the button on the smartphone. The lower the latency, the faster the response can be sent.}
%=============================================================================

%_____________________________________________________________________________
%=============================================================================
% 								BAGIAN XI
%=============================================================================
% Kata-kata kunci dan keywords : diletakkan di bawah abstrak (ina dan eng)
% - kunciINA: kata-kata kunci dalam bahasa indonesia
% - kunciENG: keywords in english
% Petunjuk: hilangkan tanda << & >>, dan isi sesuai dengan data anda.
%=============================================================================
\kunciINA{Socket.io, pemanfaatan, \textit{smartphone}, \textit{PC}, pengendali, permainan, web, \textit{browser}, \textit{latency}}
\kunciENG{Socket.io, utilization, smartphone, PC, controller, game, web, browser, latency}
%=============================================================================

%_____________________________________________________________________________
%=============================================================================
% 								BAGIAN XII
%=============================================================================
% Persembahan : kepada siapa anda mempersembahkan skripsi ini ...
% Petunjuk: hilangkan tanda << & >>, dan isi sesuai dengan data anda.
%=============================================================================
\untuk{Dipersembahkan kepada Teknik Informatika UNPAR, keluarga tercinta, teman-teman, dan diri sendiri}
%=============================================================================

%_____________________________________________________________________________
%=============================================================================
% 								BAGIAN XIII
%=============================================================================
% Kata Pengantar: tempat anda menuliskan kata pengantar dan ucapan terima 
% kasih kepada yang telah membantu anda bla bla bla ....  
% Petunjuk: hilangkan tanda << & >>, dan isi sesuai dengan data anda.
%=============================================================================
\prakata{Puji syukur kepada Tuhan Yang Maha Esa atas seluruh berkat yang diberikan kepada penulis sehingga dapat menyelesaikan skripsi dengan judul \textbf{Pemanfaatan Smartphone sebagai Pengendali Permainan Berbasis Web} dengan baik dan tepat waktu. Penulis juga berterima kasih kepada pihak-pihak yang telah memberikan dukungan dan bantuan kepada penulis dalam menyelesaikan skripsi ini. Penulis ingin mengucapkan terima kasih kepada:
\begin{enumerate}
	\item Kedua orang tua penulis, Bapak Permadi dan Ibu Heni Herliana Nurhayati yang selalu memberikan dukungan selama pengerjaan skripsi ini.
	\item Adik penulis, Mulyo Raharjo Pambudi yang selalu menghibur dan menemani penulis selama pengerjaan skripsi.
	\item Bapak Pascal Alfadian sebagai dosen pembimbing yang telah membimbing penulis hingga dapat menyelesaikan skripsi ini.
	\item Renaldi Nugroho, Gabriel Radewa, dan Reza Zacky yang telah membantu penulis dalam pengujian dan menemukan \textit{bug} pada aplikasi Finger For Life.
	\item Teman-Teman Kosan Ilham yang telah menghibur penulis dan membantu dalam pengujian aplikasi Finger For Life.
	\item Teman-teman Teknik Informatika UNPAR yang telah membantu dalam pengerjaan tugas akhir ini.
	\item Teman-teman Unit Kegiatan Mahasiswa POTRET UNPAR yang membantu dalam pengujian pada aplikasi Finger For Life.
	\item Pihak-pihak lain yang telah membantu penulisan skripsi ini, yang terus memberikan doa dan semangat kepada penulis.
\end{enumerate}
Akhir kata, penulis berharap agar skripsi ini dapat bermanfaat bagi pembaca yang hendak melakukan penelitian dan pengembangan yang terkait dengan skripsi ini.} 
%=============================================================================

%_____________________________________________________________________________
%=============================================================================
% 								BAGIAN XIV
%=============================================================================
% Tambahkan hyphen (pemenggalan kata) yang anda butuhkan di sini 
%=============================================================================
\hyphenation{ma-te-ma-ti-ka}
\hyphenation{fi-si-ka}
\hyphenation{tek-nik}
\hyphenation{in-for-ma-ti-ka}
%=============================================================================

%_____________________________________________________________________________
%=============================================================================
% 								BAGIAN XV
%=============================================================================
% Tambahkan perintah yang anda buat sendiri di sini 
%=============================================================================
\renewcommand{\vtemplateauthor}{lionov}
\pgfplotsset{compat=newest}
%=============================================================================
